%======================================================================================================================
% SLO: definiraj strukturo dokumenta
% ENG: define file structure
%======================================================================================================================
\documentclass[a4paper, 12pt]{book}
\usepackage[T1]{fontenc}

%======================================================================================================================
% SLO: Odkomentiraj "\SLOtrue " za izbiro slovenskega jezika
% ENG: Uncomment "\SLOfalse" to chose English languagge
%======================================================================================================================
\newif\ifSLO
\newif\ifTRACKEXIST
\newif\ifTRACKCS
\newif\ifPROGRAMMM

% ---------------------------------------------------------------------------------------------------------------------
% IMPORTANT: Adjust the thesis language, your study program and course within this block
%  --------------------------------------------------------------------------------------------------------------------
% switch language
\SLOtrue % Enables Slovenian language
%\SLOfalse  % Enables English language

% switch programs: Computer science and Multimedia. Set to false if the program is in Multimedia
\PROGRAMMMfalse
%\PROGRAMMMtrue

% switch on if your program is divided into tracks CS and DS, otherwise leave it false
% CAUTION: if you were first enrolled into your program before school year 2019/2020, your program is not divided
% into tracks. In any case, be absolutely sure you select the correct variant. IF IN DOUBT, always contact the
% student office to advise you.
%
\TRACKEXISTfalse
%\TRACKEXISTtrue

% default course name is "Computer science" if your course name is "Data science", set the following switch to false
\TRACKCStrue % uncomment if the thesis is from course "Information science"
%\TRACKCSfalse % uncomment if the thesis is from course "Data Science"
% ---------------------------------------------------------------------------------------------------------------------
% End of language, program and course adjustment
% ---------------------------------------------------------------------------------------------------------------------


%======================================================================================================================
% SLO: vključi oblikovanje in pakete
% ENG: include design and packages
%======================================================================================================================
%----------------------------------------------------------------
% SLO: LaTeX paketi
% ENG: LateX packages
%----------------------------------------------------------------
% SLO: omogoča uporabo slovenskih (latinskih) črk kodiranih v formatu UTF-8
% ENG: enables the use of slovene (latin) caracters encoded in the UFT-8 format
\usepackage[utf8]{inputenc}
%\inputencoding{utf8}
% SLO: naloži, med drugim, slovenske delilne vzorce
% ENG: loads, among others, slovene dividing patterns
\usepackage[slovene,english]{babel}
% SLO: poskrbi za postavitev strani
% ENG: takes care of the page layout
\usepackage{fancyhdr}
% SLO: za vlaganje slik različnih formatov
% ENG: for loading figures of different formats
\usepackage{graphicx}
\usepackage{caption}
\captionsetup[figure]{labelfont=bf} % SLO: napis "Slika #" v krepkem tisku
									% ENG: wirte "Figure #" caption in bold
\captionsetup[table]{labelfont=bf} % SLO: napis "Tabela #" v krepkem tisku
								   % ENG: wirte "Table #" caption in bold
% SLO: za pisanje psevdokode
% ENG: for writing pseudocode
\usepackage{algorithm}
\usepackage{algorithmic}
\floatname{algorithm}{\footnotesize Algorithm} % SLO: napis "Algoritem #" v krepkem tisku
											   % ENG: write "Algorithm #" caption in bold
% SLO: poveže reference slik/tabel in slike/tabele znotraj dokumenta
% ENG: links image/table references with the images/tables within the document
\usepackage[pdfa]{hyperref}
% SLO: pri kliku na referenco slike/tabele se postavi na vrh slike/tabele
% ENG: when clicking the image/table reference, position the focus on top of the image/table
\usepackage[all]{hypcap}
% SLO: omogoča, med drugim, definicjo in uporebo barve
% ENG: enables, among others, the definition and use of colors
\usepackage{xcolor}
%----------------------------------------------------------------
% SLO: dodatni paketi
% ENG: additional packages
%----------------------------------------------------------------
% SLO: omogoča večjo manipulacijo nad tabelami
% ENG: allows for greater manipulation of tables
\usepackage{booktabs}
% SLO: naloži dodatne simbole
% ENG: loads additional symbols
\usepackage{amssymb}
% SLO: omogoča, med drugim, sklicevanje na formule z eqref
% ENG: enables, among others, equation referencing with eqref
\usepackage{amsmath}
% SLO: omogoča komentiranje večjega dela teksta
% ENG: enables the commenting of larger text parts
\usepackage{verbatim}
% SLO: omogoča rotacijo PDF strani v ležeč položaj
% ENG: enables the rotation of a PDF page to landscape
\usepackage{pdflscape}
% SLO: omogoča barvanje vrstic in stolpcev tabel
% ENG: enables coloring of table rows and columns
\usepackage{colortbl}
\usepackage{url}
\usepackage{tabularx}



%================================================================
% SLO: nastavitve dokumenta
% ENG: document properties
%================================================================
% SLO: prilagoditev robov za tisk
% ENG: margin adjustments for printing
\addtolength{\marginparwidth}{-20pt}
\addtolength{\oddsidemargin}{40pt}
\addtolength{\evensidemargin}{-40pt}
% SLO: razmik med vrsticami
% ENG: line spacing
\renewcommand{\baselinestretch}{1.3}
% SLO: postavitev strani
% ENG: page layout
\renewcommand{\chaptermark}[1]{\markboth{\MakeUppercase{\thechapter.\ #1}}{}}
\renewcommand{\sectionmark}[1]{\markright{\MakeUppercase{\thesection.\ #1}}}
\renewcommand{\headrulewidth}{0.5pt} % Header rule
\renewcommand{\footrulewidth}{0pt} % Footer rule
%
\fancypagestyle{frontmatter}{%
	\fancyhf{} % Clear all headers and footers first
	\fancyhead[LE, RO]{\sl \thepage}
	%\fancyhead[LO]{\sl \rightmark}
	%\fancyhead[RE]{\sl \leftmark}
}
\fancypagestyle{mainmatter}{%
  	\fancyhf{} % Clear all headers and footers first
	\fancyhead[LE,RO]{\sl \thepage}
	\fancyhead[LO]{\sl \rightmark}
	\fancyhead[RE]{\sl \leftmark}
}
% SLO: font za ime avtorja
% ENG: font for author name
\newcommand{\authorfont}{\Large}
% SLO: font za naslov diplomskega dela
% ENG: font for thesis title
\newcommand{\titlefont}{\LARGE\bf}
% SLO: globina kazala
% ENG: content depth
\setcounter{tocdepth}{1}
% SLO: definiraj ukaz za prazno stran
% ENG: define the command for empty page
\newcommand{\clearemptydoublepage}{\newpage{\pagestyle{empty}\cleardoublepage}}

% course title
\newcommand{\trackname}{!undefined!}

\newcommand{\BibTeX}{{\sc Bib}\TeX}




%----------------------------------------------------------------------------------------------------------------------
% |||||||||||||||||||||| USTREZNO POPRAVI |||||||||||||||||||||||
% |||||||||||||||||||||| EDIT ACCORDINGLY |||||||||||||||||||||||
%----------------------------------------------------------------------------------------------------------------------
\newcommand{\ttitle}{Neprekinjena integracija in dostava poslovno kritičnih aplikacij}
\newcommand{\ttitleEn}{Continuous integration and delivery for business critical applications}
\newcommand{\tsubject}{\ttitle}
\newcommand{\tsubjectEn}{\ttitleEn}
\newcommand{\tauthor}{Jakob Maležič}
\newcommand{\temail}{jm6421@student.uni-lj.si}
\newcommand{\myyear}{2023}
\newcommand{\tkeywords}{neprekinjena integracija, neprekinjena dostava, poslovno kritične aplikacije}
\newcommand{\tkeywordsEn}{continuous integration, continuous deployment, business critical applications}
\newcommand{\mysupervisor}{doc.~dr.\ Nejc Ilc}
\newcommand{\mycosupervisor}{dr.\ Tadej Justin}

% include formatted front pages

%----------------------------------------------------------------
% SLO: definiraj metapodatke za datoteko master_thesis.tex
% ENG: define metadata for the file master_thesis.tex
%----------------------------------------------------------------
%----------------------------------------------------------------
%	HYPERREF SETUP
% SLO: ustrezno popravi e-mail
% ENG: edit the e-mail accordingly
%----------------------------------------------------------------
\hypersetup{pdftitle={\ttitle}}
\hypersetup{pdfsubject=\ttitleEn}
\hypersetup{pdfauthor={\tauthor, \temail}}
\hypersetup{pdfkeywords=\tkeywordsEn}

%----------------------------------------------------------------
% define medatata
% SLO: ustrezno popravi e-mail
% ENG: edit the e-mail accordingly
%----------------------------------------------------------------
\def\Title{\ttitle}
\def\Author{\tauthor, \temail}
\def\Subject{\ttitleEn}
\def\Keywords{\tkeywordsEn}
\def\Org{Univerza v Ljubljani, Fakulteta za računalništvo in informatiko}

%%%%%%%%%%%%%%%%%%%%%%%%%%%%%%%%%%%%%%%%
% \convertDate converts D:20080419103507+02'00' to 2008-04-19T10:35:07+02:00
%%%%%%%%%%%%%%%%%%%%%%%%%%%%%%%%%%%%%%%%
\def\convertDate{%
    \getYear
}

{\catcode`\D=12
 \gdef\getYear D:#1#2#3#4{\edef\xYear{#1#2#3#4}\getMonth}
}
\def\getMonth#1#2{\edef\xMonth{#1#2}\getDay}
\def\getDay#1#2{\edef\xDay{#1#2}\getHour}
\def\getHour#1#2{\edef\xHour{#1#2}\getMin}
\def\getMin#1#2{\edef\xMin{#1#2}\getSec}
\def\getSec#1#2{\edef\xSec{#1#2}\getTZh}
\def\getTZh +#1#2{\edef\xTZh{#1#2}\getTZm}
\def\getTZm '#1#2'{%
    \edef\xTZm{#1#2}%
    \edef\convDate{\xYear-\xMonth-\xDay T\xHour:\xMin:\xSec+\xTZh:\xTZm}%
}

\expandafter\convertDate\pdfcreationdate


%%%%%%%%%%%%%%%%%%%%%%%%%%%%%%%%%%%%%%%%
% get pdftex version string
%%%%%%%%%%%%%%%%%%%%%%%%%%%%%%%%%%%%%%%%
\newcount\countA
\countA=\pdftexversion
\advance \countA by -100
\def\pdftexVersionStr{pdfTeX-1.\the\countA.\pdftexrevision}

%%%%%%%%%%%%%%%%%%%%%%%%%%%%%%%%%%%%%%%%
% XMP data
%%%%%%%%%%%%%%%%%%%%%%%%%%%%%%%%%%%%%%%%
\usepackage{xmpincl}
\includexmp{pdfa-1b}

%%%%%%%%%%%%%%%%%%%%%%%%%%%%%%%%%%%%%%%%
% pdfInfo
%%%%%%%%%%%%%%%%%%%%%%%%%%%%%%%%%%%%%%%%
\pdfinfo{%
    /Title    (\ttitle)
    /Author   (\tauthor, \temail)
    /Subject  (\ttitleEn)
    /Keywords (\tkeywordsEn)
    /ModDate  (\pdfcreationdate)
    /Trapped  /False
}

%================================================================
% SLO: razno
% ENG: other
%================================================================
% SLO: nastavitev sklicevanj
% ENG: hyper referencing setup
\definecolor{black}{rgb}{0,0,0}
\hypersetup{
	colorlinks = true,
	linkcolor = black,
	citecolor = black,
	urlcolor = black
}

%----------------------------------------------------------------
% SLO: dodaj poti do datotek s slikami
% ENG: add paths to files containing figures
%----------------------------------------------------------------
\graphicspath{
	{figures/}
	{tables/}
}
%----------------------------------------------------------------
% SLO: moji paketi
% ENG: my packages
%----------------------------------------------------------------
% ...
%----------------------------------------------------------------
% SLO: moji konstrukti
% ENG: my constructs
%----------------------------------------------------------------
\newtheorem{izrek}{Izrek}[chapter]
\newtheorem{trditev}{Trditev}[izrek]
\newenvironment{dokaz}{\emph{Dokaz.}\ }{\hspace{\fill}{$\Box$}}

\newcommand{\CcImageCc}[1]{%
	\includegraphics[scale=#1]{cc-licenca/cc_cc_30.pdf}%
}
\newcommand{\CcImageBy}[1]{%
	\includegraphics[scale=#1]{cc-licenca/cc_by_30.pdf}%
}
\newcommand{\CcImageSa}[1]{%
	\includegraphics[scale=#1]{cc-licenca/cc_sa_30.pdf}%
}

%================================================================
% SLO: začetne strani magistrskega dela
% ENG: fist pages of the master's thesis
%================================================================
\begin{document}
% SLO: prepreči težave s številkami strani v kazalu
% ENG: prevents problems with the page numbers in the contents page
\renewcommand{\thepage}{}

%----------------------------------------------------------------
% Language-dependent formatting
%----------------------------------------------------------------
\ifSLO
    % SLO: definiraj slovensko besedo za kazalo
    \renewcommand{\contentsname}{Kazalo}

    % SLO: naslovnica
    % select the course title if it exist
\ifTRACKEXIST
    \ifTRACKCS
        \renewcommand{\trackname}{Računalništvo in Informatika}
    \else
        \renewcommand{\trackname}{Podatkovne Vede}
    \fi
\fi

\ifPROGRAMMM
    \thispagestyle{empty}
    \begin{center}
        {\large\sc Univerza v Ljubljani\\Fakulteta za računalništvo in informatiko\\
            Fakulteta za elektrotehniko}
    	   \vskip 10em
    	   {\authorfont \tauthor \par}
    	   {\titlefont \ttitle \par}
        {\vskip 2em \textsc{MAGISTRSKO DELO\\[2mm]
        MAGISTRSKI ŠTUDIJSKI PROGRAM DRUGE STOPNJE\\MULTIMEDIJA
        }\par}
        \vfill\null
        {\large \textsc{Mentor}: \mysupervisor \par}
   	    {\large \textsc{Somentor}: \mycosupervisor \par}
        {\vskip 2em \large Ljubljana, \myyear \par}
   \end{center}
\else
    \thispagestyle{empty}
	\begin{center}
            {\large\sc Univerza v Ljubljani\\Fakulteta za računalništvo in informatiko}
    	   \vskip 10em
    	   {\authorfont \tauthor \par}
    	   {\titlefont \ttitle \par}
        {\vskip 2em \textsc{MAGISTRSKO DELO\\[2mm]
        ŠTUDIJSKI PROGRAM DRUGE STOPNJE\\RAČUNALNIŠTVO IN INFORMATIKA
        \ifTRACKEXIST
            \\Smer: \trackname
        \fi
        }\par}
        \vfill\null
        {\large \textsc{Mentor}: \mysupervisor \par}
   	    {\large \textsc{Somentor}: \mycosupervisor \par}
        {\vskip 2em \large Ljubljana, \myyear \par}
   \end{center}
\fi  \clearemptydoublepage
    % SLO: avtorske pravice
    \thispagestyle{empty}
\vspace*{\fill}
{\noindent\footnotesize
{\sc Avtorske pravice}. 
\ifPROGRAMMM
    Rezultati magistrskega dela so intelektualna lastnina avtorja, Fakultete za ra\-ču\-nal\-niš\-tvo in informatiko ter Fakultete za Elektrotehniko Univerze v Ljubljani. Za objavljanje ali izkoriščanje rezultatov ma\-gi\-str\-ske\-ga dela je potrebno pisno soglasje avtorja, Fakultete za ra\-ču\-nal\-niš\-tvo in informatiko, Fakultete za Elektrotehniko ter mentorja\footnote{V dogovorju z mentorjem lahko kandidat magistrsko delo s pripadajočo izvorno kodo izda tudi pod drugo licenco, ki ponuja določen del pravic vsem: npr. Creative Commons, GNU GPL. V tem primeru na to mesto vstavite opis licence, na primer tekst~\cite{licence}.}.
\else
    Rezultati magistrskega dela so intelektualna lastnina avtorja in Fakultete za ra\-ču\-nal\-niš\-tvo in informatiko Univerze v Ljubljani. Za objavljanje ali izkoriščanje rezultatov ma\-gi\-str\-ske\-ga dela je potrebno pisno soglasje avtorja, Fakultete za ra\-ču\-nal\-niš\-tvo in informatiko ter mentorja\footnote{V dogovorju z mentorjem lahko kandidat magistrsko delo s pripadajočo izvorno kodo izda tudi pod drugo licenco, ki ponuja določen del pravic vsem: npr. Creative Commons, GNU GPL. V tem primeru na to mesto vstavite opis licence, na primer tekst~\cite{licence}.}.
\fi    
}
\begin{center}
{\footnotesize{\sc \copyright \myyear\ \tauthor}}
\end{center}  \clearemptydoublepage
    % SLO: izjava o avtorstvu (ni več del vezane izdaje, ločena oddaja)
    % SLO: zahvala
    \thispagestyle{empty}

\begin{center}
{\Large \textbf{\sc Zahvala}}
\end{center}
\vspace{0.5cm}

{\it\noindent
Rad bi se zahvalil svoji družini, še posebej svojim staršem, za vzpodbudo in podporo tekom mojega študija, zaročenki Maši za potrpežljivost in pomoč pri pisanju, mentorjema doc. dr. Nejcu Ilcu in dr. Tadeju Justinu za mentorstvo in nasvete, sodelavcu Roku Koleša za vso pomoč pri reševanju izzivov, in podjetju Medius, da mi je omogočilo pisanje magistrske naloge ter priskrbelo zanimivo temo.

\vspace{0.5cm} \hfill \tauthor, \myyear
} \clearemptydoublepage
    % SLO: posvetilo
    \thispagestyle{empty}\mbox{}{\vskip0.20\textheight}\mbox{}\hfill\begin{minipage}{0.55\textwidth}%

Vsem rožicam tega sveta.\\\\
\textit{''The only reason for time is so that everything doesn't happen at once.''}
\flushright --- Albert Einstein
\normalfont\end{minipage} \clearemptydoublepage
\else

    % ENG: title page ENG
    % select the course title if it exist
\ifTRACKEXIST
    \ifTRACKCS
        \renewcommand{\trackname}{Computer and Information Science}
    \else
        \renewcommand{\trackname}{Data Science}
    \fi
\fi

\ifPROGRAMMM
    \thispagestyle{empty}
	\begin{center}
        {\large\sc University of Ljubljana\\Faculty of Computer and Information Science\\
        Faculty of Electrical Engineering}
    	\vskip 10em
    	{\authorfont \tauthor \par}
    	{\titlefont \ttitleEn \par}
        {\vskip 2em \textsc{MASTER'S THESIS\\[2mm]
        THE 2nd CYCLE MASTER'S STUDY PROGRAMME\\MULTIMEDIA
        }\par}
        \vfill\null
        {\large \textsc{Supervisor}: \mysupervisor \par}
   	    {\large \textsc{Co-supervisor}:  \mycosupervisor \par}
        {\vskip 2em \large Ljubljana, \myyear \par}
   \end{center}
\else
    \thispagestyle{empty}
	\begin{center}
        {\large\sc University of Ljubljana\\Faculty of Computer and Information Science}
    	\vskip 10em
    	{\authorfont \tauthor \par}
    	{\titlefont \ttitleEn \par}
        {\vskip 2em \textsc{MASTER'S THESIS\\[2mm]
        THE 2nd CYCLE MASTER'S STUDY PROGRAMME\\COMPUTER AND INFORMATION SCIENCE
        \ifTRACKEXIST
            \\Track: \trackname
        \fi
        }\par}
        \vfill\null
        {\large \textsc{Supervisor}: \mysupervisor \par}
   	    {\large \textsc{Co-supervisor}:  \mycosupervisor \par}
        {\vskip 2em \large Ljubljana, \myyear \par}
    \end{center}
\fi  \clearemptydoublepage
    % ENG: title page SLO
    % select the course title if it exist
\ifTRACKEXIST
    \ifTRACKCS
        \renewcommand{\trackname}{Računalništvo in Informatika}
    \else
        \renewcommand{\trackname}{Podatkovne Vede}
    \fi
\fi

\ifPROGRAMMM
    \thispagestyle{empty}
    \begin{center}
        {\large\sc Univerza v Ljubljani\\Fakulteta za računalništvo in informatiko\\
            Fakulteta za elektrotehniko}
    	   \vskip 10em
    	   {\authorfont \tauthor \par}
    	   {\titlefont \ttitle \par}
        {\vskip 2em \textsc{MAGISTRSKO DELO\\[2mm]
        MAGISTRSKI ŠTUDIJSKI PROGRAM DRUGE STOPNJE\\MULTIMEDIJA
        }\par}
        \vfill\null
        {\large \textsc{Mentor}: \mysupervisor \par}
   	    {\large \textsc{Somentor}: \mycosupervisor \par}
        {\vskip 2em \large Ljubljana, \myyear \par}
   \end{center}
\else
    \thispagestyle{empty}
	\begin{center}
            {\large\sc Univerza v Ljubljani\\Fakulteta za računalništvo in informatiko}
    	   \vskip 10em
    	   {\authorfont \tauthor \par}
    	   {\titlefont \ttitle \par}
        {\vskip 2em \textsc{MAGISTRSKO DELO\\[2mm]
        ŠTUDIJSKI PROGRAM DRUGE STOPNJE\\RAČUNALNIŠTVO IN INFORMATIKA
        \ifTRACKEXIST
            \\Smer: \trackname
        \fi
        }\par}
        \vfill\null
        {\large \textsc{Mentor}: \mysupervisor \par}
   	    {\large \textsc{Somentor}: \mycosupervisor \par}
        {\vskip 2em \large Ljubljana, \myyear \par}
   \end{center}
\fi  \clearemptydoublepage
    % ENG: copyright
    \thispagestyle{empty}
\vspace*{\fill}
{\noindent\footnotesize
{\sc Copyright}. 
\ifPROGRAMMM
    The results of this master's thesis are the intellectual property of the author, the Faculty of Computer and Information Science and the Faculty of Electrical Engineering, University of Ljubljana. For the publication or exploitation of the master's thesis results, a written consent of the author, the Faculty of Computer and Information Science, the Faculty of Electrical Engineering and the supervisor is necessary.
    \footnote{V dogovorju z mentorjem lahko kandidat magistrsko delo s pripadajočo izvorno kodo izda tudi pod drugo licenco, ki ponuja določen del pravic vsem: npr. Creative Commons, GNU GPL. V tem primeru na to mesto vstavite opis licence, na primer tekst~\cite{licence}.}
\else
    The results of this master's thesis are the intellectual property of the author and the Faculty of Computer and Information Science, University of Ljubljana. For the publication or exploitation of the master's thesis results, a written consent of the author, the Faculty of Computer and Information Science, and the supervisor is necessary.
    \footnote{V dogovorju z mentorjem lahko kandidat magistrsko delo s pripadajočo izvorno kodo izda tudi pod drugo licenco, ki ponuja določen del pravic vsem: npr. Creative Commons, GNU GPL. V tem primeru na to mesto vstavite opis licence, na primer tekst~\cite{licence}.}
\fi    
}
\begin{center}
{\footnotesize{\sc \copyright \myyear\ \tauthor}}
\end{center}  \clearemptydoublepage
    % ENG: declaration of authorship (not part of paper edition, turn in separately)
    % ENG: acknowledgements
    \thispagestyle{empty}

\begin{center}
{\Large \textbf{\sc Acknowledgments}}
\end{center}
\vspace{0.5cm}

{\it\noindent
Worth mentioning in the acknowledgment is everyone who contributed to your thesis.

\vspace{0.5cm} \hfill \tauthor, \myyear
} \clearemptydoublepage
    % ENG: dedication
    \thispagestyle{empty}\mbox{}{\vskip0.20\textheight}\mbox{}\hfill\begin{minipage}{0.55\textwidth}%

To all the flowers of this world.\\\\
\textit{''The only reason for time is so that everything doesn't happen at once.''}
\flushright --- Albert Einstein
\normalfont\end{minipage} \clearemptydoublepage
\fi

%----------------------------------------------------------------
% SLO: kazalo
% ENG: contents
%----------------------------------------------------------------
\begingroup
	\hypersetup{colorlinks=true,linkcolor=black}
	\def\thepage{}
	\tableofcontents{}
	\clearemptydoublepage
\endgroup


\ifSLO
    % SLO: seznam kratic
    \chapter*{Seznam uporabljenih kratic}

\begin{tabular}{l|l|l}
  {\bf kratica} & {\bf angleško} & {\bf slovensko} \\ \hline
  % after \\: \hline or \cline{col1-col2} \cline{col3-col4} ...
  {\bf CRM} & Customer Relationship Management & upravljanje odnosov s strankami \\
  {\bf SCM} & Supply Chain Management & upravljanje oskrbovalne verige \\
  {\bf ERP} & Enterprise Resource Planning & upravljanje podjetniških virov \\
  {\bf BI} & Business Intelligence & poslovna inteligenca \\
  {\bf REST} & Representational state transfer & predstavitveni prenos stanja \\
  {\bf IaC} & Infrastructure as Code & infrastuktura kot koda \\
  {\bf API} & Application Programming Interface & aplikacijski programski vmesnik \\
  {\bf NPM} & Node Package Manager & upravitelj paketov Node \\
  {\bf PNPM} & Performant Node Package Manager & učinkovit upravitelj paketov Node \\
  {\bf CI/CD} & Continuos Integration and Deployment & neprekinjena integracija in  dostava \\
  
\end{tabular} \clearemptydoublepage
    % SLO: glavne strani diplomskega dela
\else
    % ENG: list of acronmys
    \chapter*{List of used acronmys}

\begin{tabular}{l|l|l}
  {\bf acronym} & {\bf meaning}  \\ \hline
  % after \\: \hline or \cline{col1-col2} \cline{col3-col4} ...
  {\bf CA} & classification accuracy \\
  {\bf DBMS} & database management system \\
  {\bf SVM} & support vector machine \\
  ... & ... \\
\end{tabular} \clearemptydoublepage
\fi

\frontmatter
\pagestyle{frontmatter}
\setcounter{page}{1} %
\renewcommand{\thepage}{}       % preprecimo težave s številkami strani v kazalu

% Include extended abstract [Razširjeni povzetek v slovenščini-- le za dela pisana v angleščini]
\ifSLO
    % include Slovenian abstract
    %---------------------------------------------------------------
% SLO: slovenski povzetek
% ENG: slovenian abstract
%---------------------------------------------------------------
\selectlanguage{slovene} % Preklopi na slovenski jezik
\addcontentsline{toc}{chapter}{Povzetek}
\chapter*{Povzetek}

\noindent\textbf{Naslov:} \ttitle
\bigskip

V magistrski nalogi predstavimo proces neprekinjene integracije in dostave (ang. continuous integration and delivery - CI/CD), ki predstavlja pomemben del razvoja poslovno kritičnih aplikacij. V svetu obstaja veliko aplikacij in programov, ki so namenjeni prav tej nalogi. Vendar vsak izmed njih ni primeren za uporabo pri razvoju poslovno kritičnih aplikaciji. V tej nalogi smo najprej izpostavili tehnologije, ki se splošno uporabljajo v okviru procesa CI/CD. Le-te smo preverili in izbrali primerne za vpeljavo v projekte poslovno kritičnih aplikacij. Dodatno smo izpostavili njihove prednosti in pomanjkljivosti pri uporabi. Na podlagi ugotovitev smo razvili komponento, ki omogoča enostavno povezovanje vseh izbranih tehnologij in skuša odpraviti izpostavljene pomanjkljivosti.

Razvili smo komponento za vzpostavitev CI/CD z orodjem GitLab CI/CD imenovano Medius CD. S komponento smo poenotili cevovode CI/CD, zmanjšali podvojenost konfiguracije in kode ter olajšali vzdrževanje. Razvita komponenta je tudi zelo prilagodljiva različnim zahtevam naročnikov in omogoča enostavno vzpostavitev cevovoda CI/CD tako v razvojnem okolju kot tudi v okolju naročnika.

Med razvojem smo odkrili potrebo po vtičniku za integrirano okolje IntelliJ, ki bi preverjal konfiguracijske datoteke orodja GitLab CI/CD. Zato smo razvili odprtokoden vtičnik Gitlab Template Lint, ki v integriranem okolju prikazuje napake in združeno vsebino konfiguracijskih datotek. Razvit vtičnik smo objavili na tržnico vtičnikov JetBrains in na ta način omogočili njegovo uporabo velikemu številu razvijalcev DevOps po vsem svetu.

V sklopu naloge smo koncepte in funkcionalnosti komponente ponazorili s pomočjo študij primerov iz resničnih projektov poslovno kritičnih aplikacij, ki uporabljajo različne programske jezike. S tem smo pokazali praktično uporabnost razvite rešitve ter njeno sposobnost prilagajanja različnim scenarijem in programskim jezikom v poslovnem okolju.


% Opišemo poslovno kritične aplikacije in proces neprekinjene integracije in dostave. Predstavimo izbrane tehnologije in aplikacije, ki jih v nadaljevanju uporabimo za vzpostavitev cevovoda neprekinjene integracije in dostave. Poseben poudarek namenimu razvoju komponente Medius CD, ki olajšuje vzpostavitev cevovoda neprekinjene integracije in dostave z orodjem Gitlab CI/CD.

% Razvita komponenta je sestavljena iz dveh ključnih delov: prvi del vključuje pripravljene predloge za uporabo v okviru projektov GitLab, medtem ko drugi del obsega skripte v programskem jeziku Bash. Te skripte se uporabljajo za izvajanje bolj kompleksnih nalog v okviru opredeljenih predlog. Ena izmed glavnih prednosti razvite komponente je njena visoka prilagodljivost, saj omogoča prilagajanje različnim zahtevam projektov. Dodana konfiguracijska datoteka igra ključno vlogo pri določanju specifičnih zahtev za posamezen projekt.

% Razvijemo komponento Medius CD, ki olajša vzpostavitev cevovoda neprekinjene integracije in dostave na platformi Gitlab. Komponento razdelimo na dva dela: predloge Gitlab, ki jih lahko enostavno pouporabimo v projektih in skripte v programskem jeziku Bash, ki jih predloge uporabljajo za izvajanje bolj kompleksnih nalog. Razvita komponenta omogoča visoko prilagodljivost različnim projektnim zahtevam, ki jih je mogoče definirati v konfiguracijski datoteki. Slednje pokažemo na primerih projektov iz resničnega sveta. 

% V vzorcu je predstavljen postopek priprave magistrskega dela z uporabo okolja \LaTeX. Vaš povzetek mora sicer vsebovati približno 100 besed, ta tukaj je odločno prekratek. Dober povzetek vključuje: (1) kratek opis obravnavanega problema, (2) kratek opis vašega pristopa za reševanje tega problema in (3) (najbolj uspešen) rezultat ali prispevek magistrske naloge.

\subsection*{Ključne besede}
\textit{\tkeywords}
\clearemptydoublepage

    % include English abstract
     %---------------------------------------------------------------
% SLO: angleški povzetek
% ENG: english abstract
%---------------------------------------------------------------
\selectlanguage{english} % Preklopi na angleški jezik
\addcontentsline{toc}{chapter}{Abstract}
\chapter*{Abstract}

\noindent\textbf{Title:} \ttitleEn
\bigskip

In this master's thesis we present the process of continuous integration and delivery (CI/CD), which is a crucial part of developing business-critical applications. There are many applications and tools worldwide designed for this purpose. However, not all of them are suitable for developing business-critical applications. We first highlighted the technologies commonly used within the CI/CD process. We examined them and selected those suitable for implementation in business-critical application projects. Additionally, we emphasized their advantages and drawbacks in their usage. Based on these findings, we developed a component that facilitates the seamless integration of the selected technologies and aims to address the identified shortcomings.

We developed a component for establishing CI/CD using the GitLab CI/CD tool, named Medius CD. With this component, we standardized CI/CD pipelines, reduced configuration and code duplication, and simplified maintenance. The developed component is highly adaptable to various customer requirements and enables easy setup of CI/CD pipelines in both the development environment and the customer's environment.

During development, we identified the need for a plugin for the integrated development environment IntelliJ IDEA that verifies the configuration files of the GitLab CI/CD tool. Therefore, we developed an open-source plugin called Gitlab Template Lint, which displays errors and consolidated content of configuration files in the integrated environment. We published the plugin on the JetBrains plugin marketplace, thus enabling its use to a large number of DevOps developers worldwide.

As part of the assignment, we illustrated the concepts and functionalities of the component through case studies from real projects of business-critical applications that use different programming languages. This demonstrated the practical usability of the developed solution and its ability to adapt to various scenarios and programming languages in the business environment.

% We describe business-critical applications and their characteristics. We present the process of continuous integration and delivery, which is an important part of developing business-critical applications. We describe the technologies that allow us to introduce best development practices and use them within the CI/CD process. 

% We develop a component for establishing CI/CD with the GitLab CI/CD tool, called Medius CD. With this component, we standardize CI/CD pipelines, reduce configuration and code duplication, and simplify maintenance. The developed component is highly adaptable to various customer requirements and enables easy setup of CI/CD pipelines in both the development environment and the customer's environment.

% As part of the task, we illustrate the concepts and functionalities of the component with case studies from real projects of business-critical applications that use different programming languages. This demonstrates the practical usability of the developed solution and its ability to adapt to various scenarios in the business environment.

% During development, we discover the need for a plugin for the integrated development environment IntelliJ IDEA that checks the configuration files of the GitLab CI/CD tool. Therefore, we develop an open-source plugin called Gitlab Template Lint, which displays errors and consolidated content of configuration files in the integrated development environment. We publish the developed plugin on the JetBrains plugin marketplace, thus enabling its use by a large number of DevOps developers worldwide.

\subsection*{Keywords}
\textit{\tkeywordsEn}
\clearemptydoublepage
\else
    % include English abstract
     %---------------------------------------------------------------
% SLO: angleški povzetek
% ENG: english abstract
%---------------------------------------------------------------
\selectlanguage{english} % Preklopi na angleški jezik
\addcontentsline{toc}{chapter}{Abstract}
\chapter*{Abstract}

\noindent\textbf{Title:} \ttitleEn
\bigskip

In this master's thesis we present the process of continuous integration and delivery (CI/CD), which is a crucial part of developing business-critical applications. There are many applications and tools worldwide designed for this purpose. However, not all of them are suitable for developing business-critical applications. We first highlighted the technologies commonly used within the CI/CD process. We examined them and selected those suitable for implementation in business-critical application projects. Additionally, we emphasized their advantages and drawbacks in their usage. Based on these findings, we developed a component that facilitates the seamless integration of the selected technologies and aims to address the identified shortcomings.

We developed a component for establishing CI/CD using the GitLab CI/CD tool, named Medius CD. With this component, we standardized CI/CD pipelines, reduced configuration and code duplication, and simplified maintenance. The developed component is highly adaptable to various customer requirements and enables easy setup of CI/CD pipelines in both the development environment and the customer's environment.

During development, we identified the need for a plugin for the integrated development environment IntelliJ IDEA that verifies the configuration files of the GitLab CI/CD tool. Therefore, we developed an open-source plugin called Gitlab Template Lint, which displays errors and consolidated content of configuration files in the integrated environment. We published the plugin on the JetBrains plugin marketplace, thus enabling its use to a large number of DevOps developers worldwide.

As part of the assignment, we illustrated the concepts and functionalities of the component through case studies from real projects of business-critical applications that use different programming languages. This demonstrated the practical usability of the developed solution and its ability to adapt to various scenarios and programming languages in the business environment.

% We describe business-critical applications and their characteristics. We present the process of continuous integration and delivery, which is an important part of developing business-critical applications. We describe the technologies that allow us to introduce best development practices and use them within the CI/CD process. 

% We develop a component for establishing CI/CD with the GitLab CI/CD tool, called Medius CD. With this component, we standardize CI/CD pipelines, reduce configuration and code duplication, and simplify maintenance. The developed component is highly adaptable to various customer requirements and enables easy setup of CI/CD pipelines in both the development environment and the customer's environment.

% As part of the task, we illustrate the concepts and functionalities of the component with case studies from real projects of business-critical applications that use different programming languages. This demonstrates the practical usability of the developed solution and its ability to adapt to various scenarios in the business environment.

% During development, we discover the need for a plugin for the integrated development environment IntelliJ IDEA that checks the configuration files of the GitLab CI/CD tool. Therefore, we develop an open-source plugin called Gitlab Template Lint, which displays errors and consolidated content of configuration files in the integrated development environment. We publish the developed plugin on the JetBrains plugin marketplace, thus enabling its use by a large number of DevOps developers worldwide.

\subsection*{Keywords}
\textit{\tkeywordsEn}
\clearemptydoublepage
    % include Slovenian abstract
    %---------------------------------------------------------------
% SLO: slovenski povzetek
% ENG: slovenian abstract
%---------------------------------------------------------------
\selectlanguage{slovene} % Preklopi na slovenski jezik
\addcontentsline{toc}{chapter}{Povzetek}
\chapter*{Povzetek}

\noindent\textbf{Naslov:} \ttitle
\bigskip

V magistrski nalogi predstavimo proces neprekinjene integracije in dostave (ang. continuous integration and delivery - CI/CD), ki predstavlja pomemben del razvoja poslovno kritičnih aplikacij. V svetu obstaja veliko aplikacij in programov, ki so namenjeni prav tej nalogi. Vendar vsak izmed njih ni primeren za uporabo pri razvoju poslovno kritičnih aplikaciji. V tej nalogi smo najprej izpostavili tehnologije, ki se splošno uporabljajo v okviru procesa CI/CD. Le-te smo preverili in izbrali primerne za vpeljavo v projekte poslovno kritičnih aplikacij. Dodatno smo izpostavili njihove prednosti in pomanjkljivosti pri uporabi. Na podlagi ugotovitev smo razvili komponento, ki omogoča enostavno povezovanje vseh izbranih tehnologij in skuša odpraviti izpostavljene pomanjkljivosti.

Razvili smo komponento za vzpostavitev CI/CD z orodjem GitLab CI/CD imenovano Medius CD. S komponento smo poenotili cevovode CI/CD, zmanjšali podvojenost konfiguracije in kode ter olajšali vzdrževanje. Razvita komponenta je tudi zelo prilagodljiva različnim zahtevam naročnikov in omogoča enostavno vzpostavitev cevovoda CI/CD tako v razvojnem okolju kot tudi v okolju naročnika.

Med razvojem smo odkrili potrebo po vtičniku za integrirano okolje IntelliJ, ki bi preverjal konfiguracijske datoteke orodja GitLab CI/CD. Zato smo razvili odprtokoden vtičnik Gitlab Template Lint, ki v integriranem okolju prikazuje napake in združeno vsebino konfiguracijskih datotek. Razvit vtičnik smo objavili na tržnico vtičnikov JetBrains in na ta način omogočili njegovo uporabo velikemu številu razvijalcev DevOps po vsem svetu.

V sklopu naloge smo koncepte in funkcionalnosti komponente ponazorili s pomočjo študij primerov iz resničnih projektov poslovno kritičnih aplikacij, ki uporabljajo različne programske jezike. S tem smo pokazali praktično uporabnost razvite rešitve ter njeno sposobnost prilagajanja različnim scenarijem in programskim jezikom v poslovnem okolju.


% Opišemo poslovno kritične aplikacije in proces neprekinjene integracije in dostave. Predstavimo izbrane tehnologije in aplikacije, ki jih v nadaljevanju uporabimo za vzpostavitev cevovoda neprekinjene integracije in dostave. Poseben poudarek namenimu razvoju komponente Medius CD, ki olajšuje vzpostavitev cevovoda neprekinjene integracije in dostave z orodjem Gitlab CI/CD.

% Razvita komponenta je sestavljena iz dveh ključnih delov: prvi del vključuje pripravljene predloge za uporabo v okviru projektov GitLab, medtem ko drugi del obsega skripte v programskem jeziku Bash. Te skripte se uporabljajo za izvajanje bolj kompleksnih nalog v okviru opredeljenih predlog. Ena izmed glavnih prednosti razvite komponente je njena visoka prilagodljivost, saj omogoča prilagajanje različnim zahtevam projektov. Dodana konfiguracijska datoteka igra ključno vlogo pri določanju specifičnih zahtev za posamezen projekt.

% Razvijemo komponento Medius CD, ki olajša vzpostavitev cevovoda neprekinjene integracije in dostave na platformi Gitlab. Komponento razdelimo na dva dela: predloge Gitlab, ki jih lahko enostavno pouporabimo v projektih in skripte v programskem jeziku Bash, ki jih predloge uporabljajo za izvajanje bolj kompleksnih nalog. Razvita komponenta omogoča visoko prilagodljivost različnim projektnim zahtevam, ki jih je mogoče definirati v konfiguracijski datoteki. Slednje pokažemo na primerih projektov iz resničnega sveta. 

% V vzorcu je predstavljen postopek priprave magistrskega dela z uporabo okolja \LaTeX. Vaš povzetek mora sicer vsebovati približno 100 besed, ta tukaj je odločno prekratek. Dober povzetek vključuje: (1) kratek opis obravnavanega problema, (2) kratek opis vašega pristopa za reševanje tega problema in (3) (najbolj uspešen) rezultat ali prispevek magistrske naloge.

\subsection*{Ključne besede}
\textit{\tkeywords}
\clearemptydoublepage


  %  \cleardoublepage
    \let\oldthesection=\thesection %Special section numbering for this chapter - remember default one
    \let\oldthesubsection=\thesubsection
    \renewcommand{\thesection}{\Roman{section}} %Special section numbering for this chapter
    \renewcommand{\thesubsection}{\thesection.\Roman{subsection}}

    % set roman page numbering
    \pagenumbering{roman}
    % set slovene language
    \selectlanguage{slovene}
    % insert extended abstract
     \chapter{Razširjeni povzetek}
 
 To je primer razširjenega povzetka v slovenščini, ki je obvezen za naloge pisane v angleščini. Razširjeni povzetek mora vsebovati vse glavne elemente dela napisanega v angleščini skupaj s kratkim uvodom in povzetkom glavnih elementov metode, glavnih eksperimentalnih rezultatov in glavnih ugotovitev. Razširjeni povzetek naj bo strukturiran v podpoglavja (spodaj je naveden le okvirni primer in je nezavezujoč).
 Čez palec navadno razširjeni povzetek nanese okoli 10 odstotkov obsega celotnega dela. 
 
 \section{Kratek pregled sorodnih del}
 
 \section{Predlagana metoda}
 
 \section{Eksperimentalna evaluacija}
 
 \section{Sklep}
 
poljuben tekst  poljuben tekst  poljuben tekst  poljuben tekst  poljuben tekst  poljuben tekst  poljuben tekst  poljuben tekst  poljuben tekst  poljuben tekst  poljuben tekst  poljuben tekst  poljuben tekst  poljuben tekst  poljuben tekst  poljuben tekst  poljuben tekst  poljuben tekst  poljuben tekst  poljuben tekst  poljuben tekst  poljuben tekst  poljuben tekst  poljuben tekst  poljuben tekst  poljuben tekst  poljuben tekst  poljuben tekst  poljuben tekst  poljuben tekst  poljuben tekst  poljuben tekst  poljuben tekst  poljuben tekst  poljuben tekst  poljuben tekst  poljuben tekst  poljuben tekst  poljuben tekst  poljuben tekst  poljuben tekst  poljuben tekst  poljuben tekst  poljuben tekst  poljuben tekst  poljuben tekst  poljuben tekst  poljuben tekst  poljuben tekst  poljuben tekst  poljuben tekst  poljuben tekst  poljuben tekst  poljuben tekst  poljuben tekst  poljuben tekst  poljuben tekst  poljuben tekst  poljuben tekst  poljuben tekst  poljuben tekst  poljuben tekst  poljuben tekst  poljuben tekst  poljuben tekst  poljuben tekst  poljuben tekst  poljuben tekst  poljuben tekst  poljuben tekst  poljuben tekst  poljuben tekst  poljuben tekst  poljuben tekst  poljuben tekst  poljuben tekst  poljuben tekst  poljuben tekst  poljuben tekst  poljuben tekst  poljuben tekst  poljuben tekst  poljuben tekst  poljuben tekst  poljuben tekst  poljuben tekst  poljuben tekst  poljuben tekst  poljuben tekst  poljuben tekst 


    \let\thesection=\oldthesection % Restore default section numbering
    \let\thesubsection=\oldthesubsection
\fi

%----------------------------------------------------------------
% SLO: Preklopi izbrani jezik
% ENG: Switch to chosen language
%----------------------------------------------------------------
\ifSLO
    \selectlanguage{slovene} % Preklopi na slovenski jezik
\else
    \selectlanguage{english}  % Switch to english language
\fi

% SLO: vklopi številčenje poglavji, ponastavi številčenje strani in uporabi arabske številkami za številčenje strani
% ENG: turns on chapter numbering, resets page numbering and uses arabic numerals for page numbers
\mainmatter
\pagestyle{mainmatter}
\setcounter{page}{1}
\pagestyle{fancy}


%================================================================
% ENG: main pages of the thesis
%================================================================

%----------------------------------------------------------------
% Poglavje 1 - Uvod in motivacija
%----------------------------------------------------------------


\chapter{Uvod}
\label{ch:uvod}

Pri razvoju večje programske opreme ali aplikacije je programje navadno razdeljeno na več zaključenih enot, ki jih
različne razvojne skupine ali razvojna podjetja neodvisno razvijajo.\ Za slednje je pred delitvijo nalog vnaprej
izbrano in specificirano tudi produkcijsko okolje.\ Na ta način si lahko razvojne skupine, za preverjanje realizacije
izvedenega dela, pripravijo razvojno okolje, ki ima čim bolj podobno infrastrukturo, kot je na voljo v produkcijskem
okolju.\ Samo tako so posamezne skupine lahko dovolj samozavestne, da bo njihovo programje pravilno delovalo tudi v
produkcijskem okolju.

\section{Motivacija}
\label{sec:motivacija}

Poseben primer produkcijskega okolja predstavlja zaprto produkcijsko okolje stranke, ki se velikokrat pojavi pri
poslovno kritičnih aplikacijah. Poslovno kritične aplikacije so tiste aplikacije, ki so ključne za delovanje
poslovnega procesa podjetja~\cite{Hinchey2010, Syng2016}. V takšnem primeru je zaradi varnosti produkcijsko okolje
pogosto razvijalcem nedostopno, kot tudi celotno omrežje stranke. S tem je dostava programske kode v omrežje
stranke s strani razvijalcev nemogoča. Za takšen primer razvijanja programske opreme je potrebno celoten proces
neprekinjene dostave prilagoditi in upoštevati morebitne dodatne zaplete pri izdaji programske opreme v produkcijsko
okolje.

\section{Cilji}
\label{sec:cilji}

V magistrski nalogi bomo predstavili strategije in tehnologije, ki nam omogočajo postavitev takšnega sistema. Slednje
bomo opisali na realnih projektih in pripravili orodja za izvajanje integracijskega ter dostavnega procesa. Bolj
podrobno bomo opisali tudi proces neprekinjene integracije in dostave v okolje, kjer nimamo dostopa do nobenega izmed
programskih okolij in nimamo možnosti odriva programske kode v strankino omrežje.


%----------------------------------------------------------------
% Poglavje 2 - Teorija
%----------------------------------------------------------------

\chapter{Teoretična osnova}
\label{ch:teoretična-osnova}
V tem poglavju bomo predstavili sorodna dela, ki so nam bila osnova in navdih za raziskovanje področja. Na kratko bomo opisali kaj posamezna dela obravnavajo in zakaj so za nas pomembna. Nato pojasnimo kaj so poslovno kritične aplikacije in navedemo nekaj primerov takšnih aplikacij. Sledi predstavitev neprekinjene integracije in neprekinjene dostave in razlaga zakaj predstavljata ključen del razvojnega procesa. Na koncu poglavja obravnavamo še pomembnost strukture projekta, njen vpliv na razvojni proces in dva načina upravljanja z različicami izvorne kode.

\section{Pregled sorodnih del}
\label{sec:pregled-sorodnih-del}

Zanimanje in motivacijo za vpeljevanje strategij razvoja in operacij smo našli v~\cite{smith_2020}, kjer so predstavljene tehnike vpeljevanja strategij v realnih delovnih okoljih. Avtor daje velik poudarek motivaciji in razlogom zakaj je vpeljava takšnih strategij potrebna in kako ugotoviti, katera orodja zares potrebujemo. Predstavljen je tudi psihološki vidik uporabe takšnih orodji na zaposlene.

V~\cite{Ebert2016} so opisani posamezni koraki neprekinjene integracije in dostave ter orodja, ki se v posameznih korakih uporabljajo. Opisan je namen posameznih orodji in kako pripomorejo k bolj učinkovitemu delovanju integracijskega procesa. 

V~\cite{Chen2016} je podrobno opisana neprekinjena dostava po posameznih korakih v cevovodu na primeru velikega podjetja. V prispevku so opisane prednosti in izzivi vzpostavitve takšnega sistema.

V~\cite{KOCEVAR_2019} so predstavljene raziskave na področju oblačnega računalništva in uporaba principov razvoja ter operacij na oblačni programski rešitvi. Na osnovi tega, so opisane dve arhitekturni zasnovi. V rezultatih se izkaže, da je proces namestitve nove programske opreme veliko lažji in hitrejši, če ga avtomatiziramo in upoštevamo principe razvoja in operacij. Opiše tudi slabosti in prednosti tehnologije Kubernetes, ki jo bomo pri implementaciji tudi sami uporabljali.

Podobno je v~\cite{OKORN_2022} predstavljena neprekinjena integracija in dostava v okolju Kubernetes s poudarkom na različnih orodjih za vzpostavitev takšnega sistema. Opisana je implementacija z različnimi orodji, med katerimi je tudi Gitlab CI/CD, ki ga uporabimo tudi sami.


\section{Poslovno kritične aplikacije}
\label{sec:poslovno-kritične-aplikacije}
Poslovno kritične aplikacije so ključne za delovanje podjetja. Odpoved ali prekinitev delovanja takšnih aplikacij močno vpliva na poslovanje podjetja in potencialno povzroči veliko finančno škodo ali škoduje ugledu podjetja. Kritičnost aplikacije, ki jo neko podjetje uporablja, je odvisna od podjetja in narave dela, ki ga to podjetje opravlja. Primeri poslovno kritičnih aplikacij so~\cite{Shen2015}:
\begin{itemize}
    \item Aplikacije za elektronsko poslovanje: spletne aplikacije, ki omogočajo podjetjem, da tržijo svoje izdelke in storitve preko spleta.
    \item Aplikacije za upravljanje odnosov s strankami (Customer Relationship Management - CRM): aplikacije, ki pomagajo podjetjem upravljati poslovanje s strankami in hranijo podatke o strankah.
    \item Upravljanje oskrbovalne verige (Supply Chain Management - SCM): sistemi, ki pomagajo podjetju upravljati z dobavo in prodajo, kot tudi z beleženjem inventarja in logistiko.
    \item Upravljanje podjetniških virov (Enterprise Resource Planning - ERP): aplikacije, ki jih podjetja uporabljajo za upravljanje dnevnih poslovnih aktivnosti kot so: vodenje računov, dobavljanje, upravljanje projektov,  upravljanje s tveganji in nadzor dobavne verige.
    \item Sistemi poslovne inteligence (Business Intelligence - BI): to so aplikacije, ki pomagajo podjetju zbirati, hraniti in analizirati podatke za sklepanje boljših poslovnih odločitev.
    \item Orodja za komunikacijo: aplikacije, ki zaposlenim omogočajo sporazumevanje po digitalnih kanalih, kot so na primer: elektronska pošta, sistemi za neposredno sporočanje in aplikacije za video konference.
\end{itemize}


Pri razvoju poslovno kritičnih aplikacij so zato pomembne naslednje lastnosti~\cite{Sarkis2004}:
\begin{itemize}
    \item Zanesljivost: aplikacija mora biti vedno na voljo in delovati konsistentno brez napak in zaustavitev.
    \item Skalabilnost: aplikacija mora biti sposobna obvladovati velike količine prometa in podpreti rastoče zahteve podjetja.
    \item Varnost: poslovno kritične aplikacije pogosto hranijo občutljive podatke, na primer podatke o uporabnikih ali finančne podatke. Aplikacija mora zato biti varna in zaščitena pred kibernetskimi napadi, nepooblaščenim dostopi ter drugim tveganji, da prepreči puščanje občutljivih podatkov.
    \item Zmogljivost: aplikacija mora zato biti odzivna in delovati dobro tudi pod velikimi obremenitvami, saj počasna ali neodzivna aplikacija negativno vpliva na produktivnosti in odvrača uporabnike.
    \item Lahko vzdrževanje: aplikacija mora biti lahka za vzdrževanje in posodabljanje kot tudi jasno ter dobro dokumentirana, da ostane zanesljiva in zmogljiva.
    \item Dobra uporabniška izkušnja: dober uporabniški vmesnik in izkušnja lahko uporabnikom olajšata razumevanje in uporabo aplikacije, kar privablja nove uporabnike in poveča njihovo produktivnost.
    % \item Integracija: Aplikacija mora nuditi dober programski vmesnik, da se lahko druge aplikacije nanj integrirajo.
    \item Prilagodljivost: podjetja imajo različne potrebe, zato je pomembno, da je aplikacijska koda prožna in prilagodljiva, da se lahko hitro prilagodi spremembam ali novim zahtevam.
\end{itemize}

% Podjetja pogosto razdelijo svoje aplikacije v skupine, glede na velikost posledic, ki bi jih njihova odpoved prinesla. V večini primerov se delijo na: aplikacije s kritično nalogo, poslovno kritične aplikacije in ne-kritične aplikacije. Ločimo jih po škodi, ki jih povzroči njihov izpad.

% \subsection{Aplikacije s kritično nalogo}
% Odpovedi aplikacije s kritično nalogo, največkrat privede do napake pri doseganju nekega pomembnega cilja. Na primer: reševanje življenj, preprečevanje resnih poškodb, transport nujnih stvari ipd.

% Primerjava poslovno kritičnih z aplikacijami s kritično nalogo.

\section{Neprekinjena integracija}
\label{sec:neprekinjena-integracija}
Neprekinjena integracija je praksa razvoja programske opreme, pri kateri programerji redno združujejo svoje spremembe kode v centralni repozitorij. Ta koda se nato avtomatsko zgradi, preizkusi in objavi. Cilj neprekinjene integracije je čim prej avtomatsko odkriti napake, tako da lahko programerji posvečajo več časa pisanju kode kot iskanju napak. Avtomatska gradnja, preizkušanje in objavljanje kode tudi pomaga ekipam, da objavijo posodobitve pogosteje in z večjim zaupanjem, kar je še posebej pomembno v razvojnih okoljih, kjer se aplikacijska koda hitro spreminja. Za sistem neprekinjene integracije so potrebni trije glavni elementi~\cite{Fowler2006}:
\begin{itemize}
    \item Centralni repozitorij, kamor razvijalci oddajo spremembe svoje kode.
    \item Orodje za avtomatizacijo gradnje, ki je odgovorno za avtomatsko gradnjo in objavljanje posodobljene kode.
    \item Orodje za preizkušanje, ki izvaja avtomatsko testiranje na posodobljeni kodi in s tem zagotavlja, da posodobljena koda deluje pravilno in dosega zahtevane standarde kakovosti.
\end{itemize}

% Travassos, G. H., Fernandes, E. B., de O. Costa, R. F., & Maldonado, J. C. (2016). Continuous integration: Best practices, patterns, and anti-patterns. Journal of Systems and Software, 120, 1-16.

% Neprekinjena integracija pri razvoju programske opreme predstavlja večkratno dnevno združevanje delovnih različič programerjev v skupno celoto \cite{Fowler2006}.
% Fowler, Martin (1 May 2006). "Continuous Integration". Retrieved 9 January 2014.

\subsection{Razlogi za neprekinjeno integracijo}
Ko razvijalec želi dodati nekaj novega v programsko kodo aplikacije, si v svojem okolju ustvari kopijo temeljne programske kode, ki je v tistem trenutku aktualna. Medtem, ko razvijalec v svojem okolju razvija nove funkcionalnosti ali pripravlja popravke, lahko ostali razvijalci spreminjajo temeljno programsko kodo. Tako se lokalna kopija razvijalca čedalje bolj razlikuje od temeljne programske kode. Ostali razvijalci lahko v temeljno programsko kodo dodajo nove funkcionalnosti, nove knjižnice ali druge vire, ki lahko ustvarijo dodatne odvisnosti in potencialne konflikte. Dalj časa kot razvijalec svoje lokalne kopije ne združi s temeljno programsko kodo, večje je tveganje integracijskih konfliktov in napak pri združevanju kode~\cite{Duvall2007}. Preden razvijalec svoje spremembe doda v glavno vejo, mora najprej posodobiti svojo lokalno kopijo, da pridobi vse spremembe, ki so bile v vmesnem času dodane v glavno vejo. Več kot je bilo sprememb dodanih v vmesnem času, več dela ima razvijalec, preden svoje spremembe lahko doda v glavno vejo. Če razvijalec predolgo odlaša z oddajo kode, se njegova kopija lahko zelo razlikuje od kode v glavni veji in za integracijo svoje kode porabi več časa, kot ga je za razvoj sprememb. Temu rečemo tudi integracijski pekel~\cite{Cunningham2009}.

Vpeljava neprekinjene integracije lahko prinese številne koristi pri razvoju programske opreme, vključno z izboljšano kakovostjo kode, hitrejšim objavljanjem posodobitev in zmanjšanjem tveganja napak pri integraciji in objavi. Vendar pa vpeljava neprekinjene integracije predstavlja tudi nekaj izzivov, kot je potreba po namestitvi in vzdrževanju infrastrukture za neprekinjeno integracijo in potreba po zagotavljanju učinkovitega postopka preizkušanja programske kode. 

\section{Neprekinjena dostava}
\label{sec:neprekinjena-dostava}

Neprekinjena dostava je metoda razvoja in izdelave, ki temelji na stalnem izboljševanju procesov in izdelkov ter zagotavljanju neprekinjene dostave izdelkov ali storitev. To pomeni, da se procesi in izdelki izboljšujejo in nadgrajujejo neprekinjeno ter dostava izdelkov ali storitev ne povzroči prekinitev ali zastoja v procesu. 

Cilj neprekinjene dostave je zmanjšanje časa med pisanjem kode in njene dostave končnim uporabnikom, obenem pa zagotavljanje visoke kakovosti in zanesljivosti. Neprekinjena dostava ima zato širši pomen, ki vključuje neprekinjeno integracijo in avtomatsko testiranje, obenem pa tudi izdajanje novih verzij in avtomatsko objavo aplikacije v testno ali produkcijsko okolje. Pomemben del pa je tudi sistem za spremljanje napak in težav z zmogljivostjo, ki razvojni ekipi pošilja povratne informacije o delovanju aplikacije in jim na ta način pomaga identificirati in odpraviti napake~\cite{Humble2010}.

\subsection{Dostava izvorne kode}
\label{subsec:dostava-izvorne-kode}
Ključen del neprekinjene dostave je tudi dostava izvorne kode stranki ali končnim uporabnikom. Proces prenosa kode mora biti dobro zasnovan, da zagotovimo, da se izvorna koda pravilno in učinkovito prenese. Dostava izvorne kode sicer ni nujna za vsak projekt in je odvisna od pogodbe med podjetjem in naročnikom ter poslovnega modela podjetja, ki je kodo napisalo. Vseeno pa se večkrat izkaže kot pomembna zahteva pri poslovno kritičnih aplikacijah, ker so bistvene za poslovanje podjetja in si zato podjetje želi ostati neodvisno od podjetja, ki je kodo napisalo.

% Skupaj lahko dostava izvorne kode stranki zagotovi večjo neodvisnost in prilagodljivost pri uporabi programske opreme, pa tudi priložnost prilagoditve programske opreme lastnim specifičnim potrebam in zahtevam (Humble et al., 2010).

Dostava izvorne kode stranki je lahko koristna tako za stranko kot tudi za podjetje, ki je napisalo izvorno kodo. Za stranko je koristna, ker ni več odvisna od podjetja iz vidika posodobitev, vzdrževanja in podpore, vendar za to lahko najame drugo podjetje ali pa ta del prevzame sama. To je lahko zelo koristno predvsem takrat, ko podjetje, ki je ustvarilo izvorno kodo, preneha s svojim poslovanjem ali pa ni zmožno zagotoviti podpore. Dostava izvorne kode pa je koristna tudi za podjetje, ki je izvorno kodo naredilo predvsem iz dveh vidikov~\cite{Humble2010}:
\begin{itemize}
    \item Prenos odgovornosti: z dostavo izvorne kode stranki lahko podjetje prenese del odgovornosti vzdrževanja programske opreme na stranko. To je še posebej koristno, če stranka načrtuje prilagoditve ali spremembe programske opreme, saj podjetje ni odgovorno za morebitne težave ali napake, ki bi lahko nastale zaradi teh sprememb.
    \item Zmanjšanje bremena vzdrževanja in podpore: če ima stranka dostop do izvorne kode, lahko sama odpravi težave in jih popravi, namesto, da bi se zanašala na podjetje za podporo.
\end{itemize}

\subsection{Neprekinjena namestitev}
\label{subsec:neprekinjena-namestitev}
Najširši obseg pa predstavlja neprekinjena namestitev, ki je praksa razvoja programske opreme, pri kateri se spremembe kode avtomatsko zgradijo, preizkusijo, oddajo naročniku in tudi objavijo v produkcijsko okolje brez posredovanja človeka. Predstavlja še en korak naprej od neprekinjene dostave, pri kateri se koda zgradi in preizkusi, vendar jo mora človek ročno objaviti v produkcijo.

Cilj neprekinjene namestitve je, da se nove funkcionalnosti in popravki čim hitreje dostavijo končnim uporabnikom. Da bi to dosegli, uporabljamo neprekinjeno integracijo, neprekinjeno dostavo in prakse neprekinjene namestitve, kot je proces upravljanja in rezervacije računalniških virov z uporabo datotek, ki jih računalnik zna prebrati - infrastruktura kot koda (Infrastructure as Code - IaC). Neprekinjena namestitev lahko pomaga podjetju, da pogosteje objavi nove funkcionalnosti in posodobitve in izboljša učinkovitost in hitrost razvojnega procesa aplikacije~\cite{Wittig2016, Humble2014}.

% Programerjem omogoča testiranje aplikacije v živo? Quality assurance?

\section{Struktura projekta}
\label{sec:struktura-projekta}
Struktura projekta opisuje kako je projekt organiziran in razmerja med različnimi deli projekta. Dobro strukturiran projekt pripomore k boljšemu razumevanju, vzdrževanju in razvijanju programske kode projekta. Obstaja veliko načinov strukturiranja projektov, seveda pa je struktura projekta odvisna od specifičnih potreb in ciljev projekta. Pri strukturiranju projekta so ključni deli:
\begin{itemize}
    \item Struktura imenika: Opisuje kako so datoteke in imeniki razporejeni na datotečnem sistemu. Dobra struktura imenika združuje odvisne datoteke in vire ter olajša iskanje datotek.
    \item Moduli in odvisnosti: V večini projektov je programska koda logično razdeljena na več modulov ali komponent, kjer je vsak izmed modulov zadolžen za le en del projekta. Ti moduli so lahko tudi odvisni med sabo, zato jih je potrebno skrbno organizirati in poskrbeti, da ne pride do krožnih odvisnosti.
    \item Skripti za gradnjo in namestitev projekta: Skripti opisujejo, kako projekt zgraditi in namestiti na testno in produkcijsko okolje. Ti skripti so po navadi del neprekinjene namestitve in morajo zato biti dobro organizirani in enostavni za razumevanje.
    \item Dokumentacija: Dobra dokumentacija je pomemben del vsakega projekta. Ta lahko vključuje uporabniška navodila, dokumentacijo aplikacijskega programskega vmesnika (Application programming interface - API) in druge dokumente, ki pomagajo razvijalcem pri razvoju ter uporabnikom pri razumevanju delovanja projekta.
\end{itemize}

Na projektno strukturo močno vpliva tudi sistem za spremljanje različic kode, ki ga uporabljamo, saj lahko kodo shranjujemo v enem skupnem ali pa jo razdelimo v več ločenih repozitorijev. Izbira repozitorijske strukture je ključna za strukturo projekta, saj ima vsaka repozitorijska struktura svoje prednosti in slabosti. Najbolj pogosti repozitorijski strukturi sta: \textit{monorepo} in \textit{polirepo}~\cite{Wiki_Architectural_pattern, Kokrehel2022}.

\subsection{Monorepo}
\label{subsec:monorepo}
Monorepo ali monolitni repozitorij je poimenovanje organizacije in upravljanje različic izvorne kode z enim repozitorijem. Ta lahko vsebuje več projektov, paketov ali modulov in razvijalcem omogoča širok dostop do izvorne kode, skupnih orodji in skupne množice odvisnosti na enem mestu. Prednosti monolitnega repozitorija so~\cite{Jaspan2018, Shakikhanli2022}:
\begin{itemize}
    \item Boljša preglednost izvorne kode: Razvijalci lažje najdejo relevantno dokumentacijo, primere implementacij in uporabe, kar pozitivno vpliva na hitrost razvijanja in kvaliteto kode. 
    \item Enostavne odvisnosti: Vsi paketi in moduli v repozitoriju imajo lahko enako različico in ni potrebno skrbeti katere verzije so med seboj kompatibilne.
    \item Lažje spreminjanje odvisnih delov kode: Pri popravljanju in posodabljanju kode lahko popravimo vse dele kode, ki so med seboj odvisni in vse skupaj oddamo v repozitorij kot eno spremembo.
\end{itemize} 

Slabosti monolitnega repozitorija so~\cite{Harry2017}:
\begin{itemize}
    \item Velikost repozitorija: Monolitni repozitorij lahko skozi razvoj zasede veliko prostora na podatkovnem sistemu. To lahko oteži prenos izvorne kode iz repozitorija in ostale operacije sistema za upravljanje različic.
    \item Kompleksen cevovod za neprekinjeno dostavo: Kompleksnost konfiguracije cevovoda za neprekinjeno dostavo se poveča, saj je potrebno vse korake izvesti na vseh modulih in projektih. Da ohranimo učinkovitost, pa nočemo vedno izvajati vseh korakov za vse module, saj to podaljša izvajanje cevovoda.
    \item Veliki zahtevki za združitev vej: Ker repozitorij vsebuje vse odvisne dele kode, je ob posodobitvah lahko posodobljenih veliko vrstic izvorne kode. To lahko povzroči slabo preglednost sprememb v pregledu sprememb zahtevka za združitev.
    \item Orodja za upravljanje: Za učinkovito upravljanje monolitnega repozitorija so velikokrat potrebna dodatna programska orodja, kot so orodja za gradnjo aplikacije in upravitelji paketov. To lahko poveča kompleksnost razvijanja aplikacije.
\end{itemize}

\subsection{Polirepo}
\label{subsec:polirepo}
Polirepo je poimenovanje organizacije in načina upravljanja z različicami izvorne kode z več repozitoriji. Ti vsebujejo vse pakete in komponente izvorne kode, ki skupaj tvorijo celotno aplikacijo. Prednosti uporabe več repozitorijev so~\cite{Shakikhanli2022}:
\begin{itemize}
    \item Poenostavljeno upravljanje: S posameznimi moduli ali paketi v ločenih repozitorijih je lažje slediti spremembam in ugotoviti, katere spremembe spadajo k posameznim projektom.
    \item Ločitev odgovornosti: Shranjevanje posameznih projektov v ločenih repozitorijih lahko pomaga izolirati kodo in zmanjša tveganje za nastanek sporov med različnimi deli projekta.
    \item Fleksibilnost: ločeni repozitoriji omogočijo različnim ekipam razvijalcev, da se posvetijo vsak svojemu delu projekta brez medsebojnih motenj.
    \item Velikost repozitorijev: Ker je celoten projekt razdeljen na več repozitorijev, so ti repozitoriji manjši in zato lažji za prenos in delo na lokalnih računalnikih, še posebej, če razvijalec dela samo na enem delu projekta.
    \item Potencialno hitrejše gradnje: S posameznimi moduli ali paketi v ločenih repozitorijih jih je morda mogoče graditi neodvisno, kar lahko skrajša skupni čas gradnje.
\end{itemize}

Slabosti uporabe več repozitorijev so~\cite{Shakikhanli2022}:
\begin{itemize}
    \item Upravljanje različic in določevanje odvisnosti: Potrebno je določiti, kako se bodo spreminjale različice modulov ali paketov v posameznih repozitorijih in kako bodo določene odvisnosti med njimi.
    \item Oteženo spreminjanje odvisnih delov kode: Po spremembi enega dela kode, se velikokrat zgodi, da je potrebno popraviti še kodo v odvisnih modulih ali paketih. To pomeni, da je treba v vsakem izmed odvisnih repozitorijev dodati spremembo in mogoče popraviti tudi verzijo.
\end{itemize}

Oba pristopa imata prednosti in slabosti in tisto, kar en razvijalec šteje za prednost, lahko drugemu predstavlja slabost. Na primer, nekateri razvijalci lahko vidijo enostavnost upravljanja enega samega repozitorija kot prednost uporabe monolitnega repozitorija, medtem ko drugi lahko zaradi potencialne povečane zapletenosti vidijo to kot slabost. Podobno lahko nekateri razvijalci vidijo sposobnost enostavnega deljenja kode in virov med projekti kot prednost uporabe več repozitorijev, medtem ko drugi lahko vidijo potrebo po upravljanju več repozitorijev kot slabost. Zato je izbira med uporabo monolitnega repozitorija ali več repozitorijev največkrat stvar osebnega okusa in je odvisna od specifičnih potreb ter narave projekta kot tudi delovnih navad podjetja.


%----------------------------------------------------------------
% Poglavje 3 - Tehnologije
%----------------------------------------------------------------

\chapter{Uporabljene tehnologije in aplikacije}
\label{ch:uporabljene-tehnologije-in-aplikacije}
V tem poglavju bomo predstavili tehnologije in aplikacije, ki so bile uporabljene za postavitev cevovoda neprekinjene integracije in dostave. Opisali bomo repozitorije za nadzor različic in orodja za gradnjo, ki so bila uporabljena, pa tudi aplikacijo Gitlab in konfiguracijske strežnike. Prav tako bomo podrobneje obravnavali način zagotavljanja varnosti aplikacije in opisali pomožna orodja in njihov namen.

\section{Repozitoriji za nadzor različic}
\label{sec:repozitoriji-za-nadzor-različic}
Za razvoj aplikacij uporabljamo več repozitorijev, ki služijo različnim namenom. Izvorna koda se nahaja v repozitoriju s sistemom za nadzor različic, ki razvijalcem omogoča preprost pregled sprememb kode in združevanje sprememb iz različnih vej. Interne knjižnice, od katerih so aplikacije odvisne, so v obliki artefaktov ali paketov naložene v repozitorij, katerega upravlja centralni upravitelj repozitorijev. Aplikacije po gradnji največkrat zapakiramo v sliko Docker, ki so shranjene v registru slik. V nadaljevanju predstavimo, katere implementacije posameznih repozitorijev smo uporabili pri implementaciji cevovoda za neprekinjeno integracijo in dostavo.

\subsection{Git}
\label{subsec:git}
Git\footnote{\url{https://git-scm.com/}} je sistem za upravljanje različic, ki se pogosto uporablja pri razvoju programske opreme. Omogoča, da več razvijalcev naenkrat dela na enaki izvorni kodi in jim pomaga slediti spremembam, ki so bile narejene~\cite{loeliger2012version}.

Uporaba sistema za upravljanje različic Git deluje tako, da ima vsak razvijalec kopijo celotnega repozitorija na svojemu računalniku vključno z vso zgodovino vseh sprememb, ki so bile narejene na izvorni kodi. Ko razvijalec razvija aplikacijo, spreminja zgolj kodo, ki jo ima shranjeno na svojem računalniku. Če želi spremembe oddati na oddaljen repozitorij, mora svoje spremembe s potrditvijo najprej dodati v svoje lokalno verzionirano skladišče, nato pa jih lahko naloži še na oddaljen repozitorij~\cite{loeliger2012version}.

Drugi razvijalci lahko nato prenesejo spremembe iz oddaljenega repozitorija v svoje lokalno skladišče in združijo spremembe s svojo kopijo. Če so na oddaljenem repozitoriju nove spremembe kode na enakih delih kot spremembe v skladišču razvijalca, mora razvijalec, ki želi združiti kodo s svojo lokalno kopijo, izbrati katere spremembe bo obdržal. Git vsebuje orodja, ki ta postopek združevanja poenostavijo~\cite{loeliger2012version}.

Za neprekinjeno integracijo in dostavo aplikacij je Git pomemben predvsem zaradi sistema za sledenje spremembam, ki omogoča, da celotna ekipa razvijalcev naenkrat sodeluje pri razvoju. Pri neprekinjeni integraciji je zelo pomembno, da lahko hitro in enostavno zaznamo in testiramo spremembe, ki so bile narejene, kot tudi, da lahko združimo spremembe. Hkrati omogoča, da v primeru napak pri testiranju ali drugih težav enostavno in hitro vrnemo aplikacijo na zadnje delujoče stanje~\cite{Dingare2022}.

\subsection{Nexus}
\label{subsec:nexus}
Apache Nexus\footnote{\url{https://www.sonatype.com/products/nexus-repository}} je eden od najbolj priljubljenih odprtokodnih upravljalcev repozitorijev, ki se uporabljajo za gostovanje in upravljanje artefaktov, kot so datoteke tipa JAR in ostale binarne komponente, ki jih projekti potrebujejo za svojo gradnjo in delovanje. Poleg strežbe repozitorijev z interno razvitimi artefakti, se pogosto uporablja kot nadomestni strežnik za zunanje repozitorije, kot je na primer centralni repozitorij Maven\footnote{\url{https://mvnrepository.com/repos/central}}. Takšen način uporabe zmanjša število zunanjih odvisnosti, ki jih je potrebno prenesti s spleta, pohitri gradnjo aplikacij in izboljša njihovo varnost z nadzorom artefaktov, ki so na voljo. Slednje omejuje z nastavljivimi pravili, kot npr. da morajo biti vsi artefakti digitalno podpisani ali pa da morajo biti odobreni s strani administratorja~\cite{Varanasi2019}.

Nexus ponuja tudi uporabniški vmesnik, s katerim je enostavno poiskati in prenesti artefakte kot tudi ročno naložiti nove artefakte v repozitorij. Prav tako izpostavlja aplikacijski programski vmesnik z arhitekturnim slogom REST, ki se lahko uporablja za avtomatizacijo podobnih opravil.

Za proces neprekinjene integracije in dostave aplikacij je Nexus pomemben zaradi več razlogov:
\begin{itemize}
    \item Pomaga poenotiti proces razvijanja aplikacije tako, da nudi enotno in konsistentno mesto za shranjevanje zgrajenih artefaktov.
    \item Poveča učinkovitost in hitrost z zagotavljanjem hitrega in zanesljivega načina dostopa do zgrajenih artefaktov.
    \item Izboljša varnost z nadzorom dostopa do artefaktov in dodatnimi pravili, ki omejujejo dostop.
    \item Olajša sledljivost s hranjenjem različnih različic artefaktov, ki so bili zgrajeni in nameščeni.
\end{itemize}

\subsection{Register slik Docker}
\label{subsec:register-slik-docker}
Register slik Docker je sistem za centralizirano hranjene in distribucijo slik Docker. Služi kot sistem za upravljanje različic slik, v katerega lahko slike shranjujemo, jih delimo z drugimi in jih prenašamo~\cite{Anwar2018}.

Obstaja veliko javnih registrov, do katerih dostop ni omejen, kot so: Docker Hub\footnote{\url{https://hub.docker.com/}}, ki je privzeti register in ga vzdržuje Docker, Quay\footnote{\url{https://quay.io/}} in IBM Cloud container registry\footnote{\url{https://www.ibm.com/cloud/container-registry}}. Na lastni infrastrukturi pa lahko postavimo tudi privatni register, do katerega omejimo dostop in s tem zagotovimo boljšo varnost~\cite{Anwar2018}.

V kontekstu neprekinjene integracije nam register slik Docker omogoča hranjenje slik Docker, ki jih proces CI zgradi. Te slike lahko potem uporabimo za postavitev testnega okolja pred oddajo v produkcijo. V kontekstu neprekinjene dostave pa nam register omogoča upravljanje z različnimi različicami aplikacijskih slik Docker in v primeru težav hitro vračanje aplikacije na prejšnjo različico.

\section{Orodja za gradnjo}
\label{sec: orodja-za-gradnjo}

\subsection{Maven}
\label{subsec:maven}

Maven\footnote{\url{https://maven.apache.org/}} je orodje za avtomatizacijo gradnje, testiranja in namestitve projektov napisanih v programskem jeziku Java. Uporablja deklarativen pristop, s katerim razvijalci naštejejo vse odvisnosti, vtičnike in ostale informacije, potrebne za gradnjo projekta, v datoteko \texttt{pom.xml}~\cite{Varanasi2019}.

Orodje je pomembno za neprekinjeno integracijo in dostavo aplikacij, ker omogoča konsistenten in ponovljiv način gradnje, testiranja in nameščanja aplikacije. Z orodjem lahko razvijalci opišejo vse potrebne korake za gradnjo projekta v eni konfiguracijski datoteki, ki jo v procesu CI/CD lahko nato uporabimo za avtomatizacijo gradnje in namestitve projekta.

\subsection{Gradle}
\label{subsec:gradle}
Gradle\footnote{\url{https://gradle.org/}} je prav tako kot Maven odprtokodno orodje za gradnjo projektov, vendar je bolj fleksibilno. Tako kot Maven, se tudi Gradle primarno uporablja za projekte spisane v programskem jeziku Java, vendar se lahko uporablja tudi za gradnjo projektov v drugih programskih jezikih, kot na primer C++ ali Python. Za definicijo skripta za gradnjo projekta uporablja domensko-specifičen jezik, ki je osnovan na jeziku Groovy in omogoča bolj zgoščene in ekspresivno napisane skripte, v primerjavi s skripti v formatu XML, ki ga uporablja Maven~\cite{Ikkink2015}.

Podobno kot Maven, je Gradle pomemben za neprekinjeno integracijo in dostavo aplikacije, ker omogoča avtomatizacijo gradnje in namestitve projekta. Gradle podpira tudi gradnjo več projektov hkrati in inkrementalno gradnjo ter je zato primeren tudi za velike in kompleksne projekte. Znan je tudi po svojem sistemu predpomnenja, ki z beleženjem vhodov in izhodov gradnje ter njihovo ponovno uporabo, pohitri gradnjo projekta. To je še posebej uporabno za proces CI/CD, kjer želimo projekt zgraditi čim hitreje.

\subsection{Node Package Manager ali NPM}
\label{subsec:npm}

Node Package manager\footnote{\url{https://www.npmjs.com/}} je orodje za upravljanje paketov, ki so napisani v programskem jeziku JavaScript in se uporabljajo v izvajalnem okolju Node.js. Orodje je vključeno v namestitveni paket izvajalnega okolja Node.js in je avtomatsko nameščeno ob namestitvi Node.js, zato služi kot privzeti upravitelj paketov. Uporablja se v ukazni vrstici in razvijalcem omogoča enostavno namestitev, upravljanje in deljenje paketov. V datoteki \texttt{package.json} razvijalci določijo odvisnosti, ki jih projekt zahteva. Te odvisnosti nato enostavno namestijo z uporabo orodja NPM, ki avtomatsko poišče ustrezne pakete in jih prenese iz registra NPM ali iz drugih virov kot je npr. repozitorij Nexus. Prav tako omogoča upravljanje različič paketov, posodabljanje paketov, odstranjevanje paketov in objavljanje lastnih paketov v zasebne ali javne repozitorije. Poskrbi tudi za razreševanje odvisnosti in zagotavlja, da so vsi zahtevani paketi in njihove odvisnosti pravilno nameščeni in združljivi med seboj~\cite{Ali2013}.

\subsection{Performant Node Package Manager ali PNPM}
\label{subsec:pnpm}
Performant Node Package Manager\footnote{\url{https://pnpm.io/}} je alternativa orodju NPM, ki si prizadeva izboljšati efektivnost in učinkovitost namestitve in upravljanja paketov. Za razliko od orodja NPM, ki uporablja centralen predpomnilnik paketov in namešča odvisnosti projekta v mapo poimenovano \texttt{node\_modules}, PNPM uporablja virtualno shrambo. Namesto podvajanja odvisnosti za več projektov, jih shrani na centralno mesto na datotečnem sistemu, kar omogoča, da si različni projekti odvisnosti delijo. Na ta način izboljšuje hitrost namestitve in zmanjšuje prostor, ki ga posamezni projekti zasedajo. Prav tako omogoča hitrejše in učinkovitejše posodobitve paketov, saj je ob posodobitvi potrebno posodobiti samo eno kopijo paketa v centralni virtualni shrambi. Orodje je združljivo z obstoječimi projekti Node.js in ga lahko uporabimo kot zamenjavo za orodje NPM. Ponuja podobne ukaze in je zato tudi enostavno za uporabo~\cite{Jacobs2019, PnpmBenchmarks2023}.

\subsection{Pip Installs Packages ali PIP}
\label{subsec:pip}
Pip Installs Packages\footnote{\url{https://pypi.org/project/pip/}} je privzeto orodje za upravljanje paketov, ki so napisani v programskem jeziku Python. Omogoča enostavno namestitev, nadgradnjo in upravljanje s paketi, ki niso del osnovne knjižnice. Orodje omogoča razvijalcem enostaven dostop in namestitev paketov iz Python Package Index-a (PyPI), ki gostuje odprto-kodne pakete~\cite{Pip2023}.

Orodje PIP je enostavno za uporabo in s preprostimi ukazi omogoča uporabnikom namestitev in brisanje paketov ter kreiranje virtualnega okolja za izolacijo projektnih odvisnosti. Zaradi splošne razširjenosti uporabe orodja PIP, služi kot ključno orodje za deljenje kode in učinkovito upravljanje paketov v projektih, ki uporabljajo programski jezik Python~\cite{Pip2023}.

\subsection{Setuptools}
\label{subsec:setuptools}
Setuptools\footnote{\url{https://pypi.org/project/setuptools/}} je knjižnica, ki poenostavi razvoj in distribucijo paketov v programskem jeziku Python. Vsebuje zbirko uporabnih funkcij in razširitev, ki nam pomagajo pri pakiranju, distribuciji in namestitvi paketov. Glavne funkcionalnosti so~\cite{Younker2008}:
\begin{itemize}
    \item Podpora za kreiranje paketov: knjižnica za konfiguracijo gradnje paketov in meta podatkov o paketu, kot so ime, verzija, avtor, odvisnosti in vhodne točke uporablja datoteko \texttt{setup.py}. Na ta način razvijalcu omogoči, da definira strukturo projekta in določi vire in dodatne datoteke, ki so del projekta.
    \item Upravljanje odvisnosti: knjižnica s ključno besedo \texttt{requires} v konfiguracijski datoteki \texttt{setup.py} omogoča avtorjem paketa nastavitev odvisnosti, ki jih paket potrebuje. Knjižnica prav tako omogoča avtomatsko razrešitev in namestitev teh odvisnosti, kar poenostavi namestitev in uporabo paketa.
    \item Vstopne točke: knjižnica omogoča uporabo vhodnih točk, ki na standarden način definirajo, kako lahko ostali paketi in aplikacije razširijo funkcionalnosti paketa. Hkrati omogoča dinamično odkrivanje in namestitev vtičnikov brez navajanja eksplicitnih uvozov modulov s stavkom \texttt{import}.
    \item Integracija s PyPI: knjižnica je dobro integrirana s PyPI in vsebuje orodja za nalaganje paketov na PyPI.
    \item Razširitev osnovnih ukazov: osnovne ukaze razširja z ukazi, ki pomagajo pri gradnji različnih distribucij, ustvarjanju dokumentacije in zagonu testov.
\end{itemize}

\subsection{Docker}
\label{subsec:docker}
Docker\footnote{\url{https://www.docker.com/}} je odprtokodno orodje za razvoj, namestitev in zagon aplikacij z uporabo vsebnikov (angl. \textit{container}). Vsebniki so način pakiranja aplikacij in njihovih odvisnosti, ki zagotavljajo, da jih lahko zanesljivo zaženemo v različnih okoljih in vsebujejo vse potrebno za zagon aplikacije: kodo, izvajalno okolje, knjižnice in sistemska orodja. Takšen standardiziran način pakiranja in distribucije aplikacij znatno olajša proces razvoja, nameščanja in upravljanja z aplikacijami. Glavne prednosti orodja Docker so~\cite{Rad2017}:
\begin{itemize}
    \item Prenosljivost: vsebniki so neodvisni od operacijskega sistema, lahko jih zaženemo na kateremkoli sistemu, ki podpira Docker in zagotavljajo konsistentno obnašanje v različnih okoljih.
    \item Skalabilnost: orodje Docker omogoča hitro razširljivost aplikacij s podvajanjem vsebnikov, ki izvajajo enako aplikacijo in porazdelitvijo obremenitve med njimi. 
    \item Izolacija: vsak vsebnik deluje v svojem izoliranem okolju, kar zagotavlja izolacijo procesov in preprečuje konflikte med aplikacijami.
    \item Učinkovita raba virov: vsebniki si delijo računalniške vire sistema, na katerem delujejo.
    \item Upravljanje različic: orodje Docker z označevanjem vsebniških slik omogoča enostavno upravljanje različic in vračanje na starejše različice.
\end{itemize}

\subsection{Kaniko}
\label{subsec:kaniko}
Kaniko\footnote{\url{https://github.com/GoogleContainerTools/kaniko}} je odprtokodno orodje za gradnjo vsebniških slik Docker iz datotek Docker (angl. \textit{Dockerfile}) znotraj vsebnika ali na gruči Kubernetes. Za gradnjo vsebniških slik ne potrebuje privilegiranega dostopa do prikritega procesa Docker (angl. \textit{Docker daemon}), kar predstavlja glavno prednost pred načinom gradnje vsebniških slik, ki ga uporablja platforma Docker~\cite{Kaniko2023}.

Platforma Docker za gradnjo slik uporablja prikriti proces, ki potrebuje korenski dostop in visoke pravice na gostujočem sistemu. Uporabniki v nekaterih okoljih, kot je na primer Kubernetes ali cevovod za neprekinjeno integracijo in dostavo zaradi varnostnih zahtev nimajo zahtevanih pravic za zagon prikritega procesa. Kaniko to težavo reši z uporabo samostojnega orodja, ki se zažene znotraj vsebnika in zgradi sliko brez priviligiranega dostopa. Vse potrebne operacije, kot so: prenos osnovnih slik, izvajanje ukazov za gradnjo slike in kreiranje končne slike vsebnika, izvede s pravicami, ki jih ima navaden uporabnik gostiteljskega sistema. To nam omogoča, da lahko zgradimo slike na gruči Kubernetes ali drugih platformah za orkestracijo~\cite{Kaniko2023}.

\section{Gitlab}
\label{sec:gitlab}
Gitlab\footnote{\url{https://about.gitlab.com/}} je spletna platforma za razvijalce in sistemske inženirje, ki vsebuje velik nabor orodji za upravljanje in dostavo aplikacij. Osnovni namen platforme je porazdeljen sistem za nadzor različic (angl. \textit{distributed version control system - DVCS}), v katerem lahko razvijalci gostijo svoje repozitorije izvorne kode, na podoben način kot v sistemu Git (\ref{subsec:git}), obenem pa vsebuje še veliko funkcionalnosti~\cite{Gitlab2023}:
\begin{itemize}
    \item Orodja za vodenje projekta, ki olajšajo: organiziranje projekta, sledenje nalogam, dodeljevanje nalog razvijalcem in določanje mejnikov.
    \item Sistem za gradnjo in zagon cevovodov za neprekinjeno integracijo in dostavo aplikacij.
    \item Orodja za pregled kode in medsebojno sodelovanje, ki omogočajo: pregled kode, dodajanje komentarjev na posamezne vrstice v kodi, pregled zahtevkov za združitev in upravljanje vej.
    \item Vgrajen register za shranjevanje in upravljanje z vsebniki Docker in njihovo uporabo v delovnih procesih.
    \item Varnostna preverjanja in preverjanja skladnosti kode z industrijskimi standardi.
    \item Možnost namestitve na lastni infrastrukturi ali v oblaku, kar omogoča skalabilnost in visoko razpoložljivost.
\end{itemize}

\subsection{Neprekinjena integracija in dostava z Gitlab CI/CD}
\label{subsec:neprekinjena-integracija-in-dostava-z-gitlab-ci-cd}

Gitlab CI/CD je sistem za gradnjo in zagon cevovodov za neprekinjeno integracijo in dostavo aplikacij, ki je integriran v platformo Gitlab. Razvijalcem omogoča avtomatizacijo gradnje, testiranja in namestitve aplikacij in s tem olajša proces dostave programske opreme. Temelji na konceptu cevovodov, ki jih predstavljajo zaporedja nalog. Vsak cevovod ima lahko več faz v kateri je lahko več nalog, vsaka naloga pa ima lahko več akcij. Z razporejanjem nalog po fazah, zagotovimo pravilen vrstni red izvajanja nalog. Cevovode definiramo v konfiguracijski datoteki \texttt{.gitlab-ci.yml}, v formatu YAML, ki se mora nahajati v korenskem direktoriju projekta. V tej datoteki specificiramo tudi vse artefakte, ki jih naloge generirajo, odvisnosti in okoljske spremenljivke~\cite{Gitlab2023}.

Za izvajanje nalog skrbijo izvajalci nalog Gitlab. Izvajajo se lahko na lastni infrastrukturi ali na oblačnih virtualnih strojih in omogočajo paralelno izvajanje nalog, kar lahko znatno pohitri povprečen izvajalni čas cevovoda. Izvajalci nalog so različnih tipov glede na to, katerim projektom so na voljo~\cite{Gitlab2023}:
\begin{itemize}
    \item deljeni: na voljo vsem projektom, ki obstajajo na instanci platforme Gitlab,
    \item skupinski: na voljo samo projektom v specifični skupini in
    \item projektni: na voljo samo specifičnemu projektu.
\end{itemize}

Izvajalce nalog lahko prilagodimo svojim potrebam z omejitvijo števila vzporednih nalog, ki jih lahko izvajajo, omejitvami virov, ki so jim na voljo, in okoljskimi spremenljivkami. Lahko jim dodelimo tudi oznake, s katerimi lahko na nalogo natančno določimo njihovo izvajanje. Med posameznimi izvajalci nalog je zagotovljena izolacija, tako da se lahko naloge izvajajo neodvisno druga od druge. Hkrati je z nadzorom dostopa in ustreznimi metodami verodostojnosti, ki zagotavljajo integriteto celotnega procesa, poskrbljeno za varnost~\cite{Gitlab2023}.

\section{Kubernetes}
\label{sec:kubernetes}
Kubernetes\footnote{\url{https://kubernetes.io/}} je odprtokodna rešitev za orkestracijo vsebnikov, ki avtomatizira namestitev, skaliranje instanc in upravljanje aplikacij nameščenih v vsebnikih. Glavni gradniki platforme Kubernetes so~\cite{Burns2022}:
\begin{itemize}
    \item Vozlišča (angl. nodes) - stroji, na katerih so nameščeni vsebniki. Izvajalno okolje vsakega vozlišča mora vsebovati orodja, ki so potrebna za upravljanje vsebnikov, kot je npr. platforma Docker.
    \item Stroki (angl. pods) - vozlišča gostijo stroke, kjer tečejo vsebniki, ki so komponente aplikacije. Vsebniki, ki tečejo na istem stroku si delijo omrežje, imenski prostor in prostor za hrambo podatkov.
    \item Nabori podvajanj (angl. replica set) - njihova naloga je ohraniti stabilen nabor kopiranih strokov.
    \item Storitve - abstrakcije, ki predstavljajo nabor strokov in definirajo način dostopa do njih. Omogočajo mehanizme za upravljanje z obremenitvijo in odkrivanje storitev.
    \item Ingress - objekti programskega vmesnika, ki upravljajo zunanji dostop do storitev v gruči. Omogočajo združevanje pravil usmerjanja v en sam vir in lahko izpostavijo več storitev pod istim naslovom IP.
    \item Shrambe podatkov (angl. volumes) - rezervirani podatkovni prostor na datotečnem sistemu, ki je neodvisen od življenjskega cikla vsebnikov in na ta način omogočajo dolgotrajno shrambo podatkov, tudi ob zaustavitvi ali ponovnem zagonu vsebnikov.
    \item Oznake in selektorji - oznake predstavljajo objekti v obliki ključ-vrednost, ki so dodani gradnikom platforme Kubernetes. Selektorji pa se uporabljajo za identifikacijo objektov na podlagi njihovih oznak.
    \item Namestitive (angl. deployments) - omogočajo upravljanje namestitve in posodobitev vmesnikov. Definiramo jih z deklarativno konfiguracijsko datoteko in skrbijo za ustrezno skaliranje, posodobitve brez zaustavitve (angl. rolling updates) in vračanje na starejše različice (angl. rollback).
\end{itemize}

% \section{Konfiguracijski strežniki}
% \label{sec:konfiguracijski-strežniki}

% \subsection{Konfiguracijski strežnik Spring Boot}
% \label{subsec:spring-boot-configuration-server}

% \section{Zagotavljanje varnosti}
% \label{sec:zagotavljanje-varnosti}

% \subsection{Kubernetes v izoliranem produkcijskem okolju}
% \label{subsec:kubernetes-v-izoliranem-produkcijskem-okolju}

% \section{Beleženje dogodkov in zbiranje metrik}
% \label{subsec:belezenje-dogodkov-in-zbiranje-metrik}

% \subsection{Graylog}
% \label{subsec:graylog}

% \subsection{Elastic}
% \label{subsec:elastic}

% \subsection{Kibana}
% \label{subsec:kibana}

\section{Pomožne tehnologije}
\label{subsec:pomozne-tehnologije}

\subsection{Konfiguracijski strežnik Consul}
\label{subsec:konfiguracijski-strežnik-consul}

Consul\footnote{\url{https://www.consul.io/}} je platforma za upravljanje s porazdeljenimi storitvami, ki omogoča odkrivanje storitev, upravljanje konfiguracije in preverjanje zdravja storitev~\cite{Sabharwal2021}.

Ena izmed glavnih komponent platforme Consul je konfiguracijski strežnik Consul, ki omogoča hranjenje in upravljanje aplikacijske konfiguracije na centralen in lahko dostopen način. Uporablja hierarhično podatkovno shrambo, v kateri so podatki shranjeni v parih ključ-vrednost. To pomeni, da vsakemu ključu pripada vrednost, ki je lahko v poljubnem formatu. Dostop do teh vrednosti, pa je omogočen preko programskega vmesnika ali orodja za ukazno vrstico~\cite{Sabharwal2021}.

Konfiguracijski strežnik Consul v procesu neprekinjene integracije in dostave uporabljamo kot centralno mesto konfiguracije cevovodov Gitlab kot tudi izvajalne konfiguracije aplikacij. Na ta način je konfiguracijo lažje najti in ob posodobitvi konfiguracije aplikacije ni potrebno ponovno namestiti, saj jo lahko prenesemo preko programskega vmesnika.

\subsection{Jsonnet}
\label{subsec:jsonnet}

Jsonnet\footnote{\url{https://jsonnet.org/}} je funkcijski jezik za kreiranje in upravljanje kompleksnih podatkovnih struktur v formatu JSON. Omogoča bolj natančen in fleksibilen način definicije in organizacije struktur, kot osnoven format JSON. Največkrat se uporablja za definicijo dinamičnih ali parametriziranih struktur, saj omogoča uporabo spremenljivk, funkcij, pogojnih stavkov, aritmetike in zank~\cite{Jsonnet2023}.

Ena glavnih prednosti uporabe jezika Jsonnet je njegova modularnost in možnost ponovne uporabe z definicijo komponent, ki jih lahko večkrat uporabimo v različnih objektih. Na ta način lahko znatno povečamo doslednost konfiguracije skozi projekt. Prav tako nudi uvažanje definicij iz drugih datotek, kar omogoča sestavljanje kompleksnih objektov in komentarje, ki olajšajo dokumentiranje konfiguracije~\cite{Jsonnet2023}.

Pri implementaciji cevovoda platforme Gitlab je Jsonnet ključnega pomena, saj nam omogoča dinamično kreiranje nalog in s tem znatno zmanjša količino vrstic v konfiguraciji cevovoda, kar poveča preglednost in razumljivost konfiguracije.

% \subsection{Copier}
% \label{subsec:copier}

\subsection{Integrirano razvojno okolje IntelliJ IDEA}
\label{subsec:integrirano-razvojno-okolje-intellij-idea}
IntelliJ IDEA\footnote{\url{https://www.jetbrains.com/idea/}} je integrirano razvojno okolje za razvoj programske opreme, ki vsebuje veliko dodatnih orodji in funkcionalnosti, ki povečajo produktivnost in poenostavijo proces razvoja programske opreme. Podpira veliko število različnih programskih jezikov in nudi napredno pomoč pri pisanju kode, analizo kode ter pomoč pri refaktoriranju kode, ki razvijalcem pomaga pri pisanju bolj berljive in učinkovite kode. Okolje vsebuje tudi vgrajeno podporo za sisteme za upravljanje različic, upravljanje projekta in obsežno podporo za ogrodja za testiranje. Pomembna lastnost razvojnega okolja, ki je ključna za neprekinjeno integracijo in olajša upravljanje z odvisnostmi ter gradnjo projekta, pa predstavlja tudi dobra integracija z orodji za gradnjo kot sta Maven in Gradle~\cite{Krochmalski2014}.


%----------------------------------------------------------------
% Poglavje 4 - Implementacija
%----------------------------------------------------------------


\chapter{Implementacija}
\label{ch:implementacija}

\section{Potek integracije in dostave}
\label{sec:potek-integracije-in-dostave}

\subsection{Proces v podjetju}
\label{subsec:proces-v-podjetju}

\subsection{Proces pri stranki}
\label{subsec:proces-pri-stranki}

\subsection{Proces neprekinjene dostave stranki}
\label{subsec:proces-neprekinjene-dostave-stranki}

\section{Razvoj komponente za neprekinjeno dostavo in integracijsko testiranje - Medius CD}
\label{sec:razvoj-komponente-za-neprekinjeno-dostavo-in integracijsko-testiranje-medius-cd}

\section{Postavitev projekta}
\label{subsec:postavitev-projekta}

\subsection{Java in Maven}
\label{subsec:java-in-maven}

\subsection{Java in Gradle}
\label{subsec:java-in-gradle}

\subsection{JavaScript in NPM}
\label{subsec:javascript-in-npm}

\section{Vtičnik Gitlab Template Lint}
\label{sec:vtičnik-gitlab-template-lint}



%----------------------------------------------------------------
% Poglavje 5 - Rezultati
%----------------------------------------------------------------


\chapter{Rezultati}
\label{ch:rezultati}

\section{Predstavitev projekta}
\label{sec:predstavitev-projekta}

\section{Primer uporabe na projektu}
\label{sec:primer-uporabe-na-projektu}

\subsection{Java projekt}
\label{subsec:java-projekt}

\subsection{JavaScript projekt}
\label{subsec:javascript-projekt}

\subsection{Python projekt}
\label{subsec:python-projekt}

\subsection{Monorepo projekt}
\label{subsec:monorepo-projekt}


%----------------------------------------------------------------
% Poglavje 6 - Zaključek
%----------------------------------------------------------------


\chapter{Zaključek}
\label{ch:zaključek}
Aleluja

% ---------------------------------------------------------------
% Appendix
% ---------------------------------------------------------------
\appendix


%----------------------------------------------------------------
% SLO: bibliografija
% ENG: bibliography
%----------------------------------------------------------------
\bibliographystyle{elsarticle-num}

%----------------------------------------------------------------
% SLO: odkomentiraj za uporabo zunanje datoteke .bib (ne pozabi je potem prevesti!)
% ENG: uncomment to use .bib file (don't forget to compile it!)
%----------------------------------------------------------------
\bibliography{bibliography}


\end{document}
