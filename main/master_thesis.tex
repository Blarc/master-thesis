%======================================================================================================================
% SLO: definiraj strukturo dokumenta
% ENG: define file structure
%======================================================================================================================
\documentclass[a4paper, 12pt]{book}
\usepackage[T1]{fontenc}

%======================================================================================================================
% SLO: Odkomentiraj "\SLOtrue " za izbiro slovenskega jezika
% ENG: Uncomment "\SLOfalse" to chose English languagge
%======================================================================================================================
\newif\ifSLO
\newif\ifTRACKEXIST
\newif\ifTRACKCS
\newif\ifPROGRAMMM

% ---------------------------------------------------------------------------------------------------------------------
% IMPORTANT: Adjust the thesis language, your study program and course within this block
%  --------------------------------------------------------------------------------------------------------------------
% switch language
\SLOtrue % Enables Slovenian language
%\SLOfalse  % Enables English language

% switch programs: Computer science and Multimedia. Set to false if the program is in Multimedia
\PROGRAMMMfalse
%\PROGRAMMMtrue

% switch on if your program is divided into tracks CS and DS, otherwise leave it false
% CAUTION: if you were first enrolled into your program before school year 2019/2020, your program is not divided
% into tracks. In any case, be absolutely sure you select the correct variant. IF IN DOUBT, always contact the
% student office to advise you.
%
\TRACKEXISTfalse
%\TRACKEXISTtrue

% default course name is "Computer science" if your course name is "Data science", set the following switch to false
\TRACKCStrue % uncomment if the thesis is from course "Information science"
%\TRACKCSfalse % uncomment if the thesis is from course "Data Science"
% ---------------------------------------------------------------------------------------------------------------------
% End of language, program and course adjustment
% ---------------------------------------------------------------------------------------------------------------------


%======================================================================================================================
% SLO: vključi oblikovanje in pakete
% ENG: include design and packages
%======================================================================================================================
%----------------------------------------------------------------
% SLO: LaTeX paketi
% ENG: LateX packages
%----------------------------------------------------------------
% SLO: omogoča uporabo slovenskih (latinskih) črk kodiranih v formatu UTF-8
% ENG: enables the use of slovene (latin) caracters encoded in the UFT-8 format
\usepackage[utf8]{inputenc}
%\inputencoding{utf8}
% SLO: naloži, med drugim, slovenske delilne vzorce
% ENG: loads, among others, slovene dividing patterns
\usepackage[slovene,english]{babel}
% SLO: poskrbi za postavitev strani
% ENG: takes care of the page layout
\usepackage{fancyhdr}
% SLO: za vlaganje slik različnih formatov
% ENG: for loading figures of different formats
\usepackage{graphicx}
\usepackage{caption}
\captionsetup[figure]{labelfont=bf} % SLO: napis "Slika #" v krepkem tisku
									% ENG: wirte "Figure #" caption in bold
\captionsetup[table]{labelfont=bf} % SLO: napis "Tabela #" v krepkem tisku
								   % ENG: wirte "Table #" caption in bold
% SLO: za pisanje psevdokode
% ENG: for writing pseudocode
\usepackage{algorithm}
\usepackage{algorithmic}
\floatname{algorithm}{\footnotesize Algorithm} % SLO: napis "Algoritem #" v krepkem tisku
											   % ENG: write "Algorithm #" caption in bold
% SLO: poveže reference slik/tabel in slike/tabele znotraj dokumenta
% ENG: links image/table references with the images/tables within the document
\usepackage[pdfa]{hyperref}
% SLO: pri kliku na referenco slike/tabele se postavi na vrh slike/tabele
% ENG: when clicking the image/table reference, position the focus on top of the image/table
\usepackage[all]{hypcap}
% SLO: omogoča, med drugim, definicjo in uporebo barve
% ENG: enables, among others, the definition and use of colors
\usepackage{xcolor}
%----------------------------------------------------------------
% SLO: dodatni paketi
% ENG: additional packages
%----------------------------------------------------------------
% SLO: omogoča večjo manipulacijo nad tabelami
% ENG: allows for greater manipulation of tables
\usepackage{booktabs}
% SLO: naloži dodatne simbole
% ENG: loads additional symbols
\usepackage{amssymb}
% SLO: omogoča, med drugim, sklicevanje na formule z eqref
% ENG: enables, among others, equation referencing with eqref
\usepackage{amsmath}
% SLO: omogoča komentiranje večjega dela teksta
% ENG: enables the commenting of larger text parts
\usepackage{verbatim}
% SLO: omogoča rotacijo PDF strani v ležeč položaj
% ENG: enables the rotation of a PDF page to landscape
\usepackage{pdflscape}
% SLO: omogoča barvanje vrstic in stolpcev tabel
% ENG: enables coloring of table rows and columns
\usepackage{colortbl}
\usepackage{url}
\usepackage{tabularx}



%================================================================
% SLO: nastavitve dokumenta
% ENG: document properties
%================================================================
% SLO: prilagoditev robov za tisk
% ENG: margin adjustments for printing
\addtolength{\marginparwidth}{-20pt}
\addtolength{\oddsidemargin}{40pt}
\addtolength{\evensidemargin}{-40pt}
% SLO: razmik med vrsticami
% ENG: line spacing
\renewcommand{\baselinestretch}{1.3}
% SLO: postavitev strani
% ENG: page layout
\renewcommand{\chaptermark}[1]{\markboth{\MakeUppercase{\thechapter.\ #1}}{}}
\renewcommand{\sectionmark}[1]{\markright{\MakeUppercase{\thesection.\ #1}}}
\renewcommand{\headrulewidth}{0.5pt} % Header rule
\renewcommand{\footrulewidth}{0pt} % Footer rule
%
\fancypagestyle{frontmatter}{%
	\fancyhf{} % Clear all headers and footers first
	\fancyhead[LE, RO]{\sl \thepage}
	%\fancyhead[LO]{\sl \rightmark}
	%\fancyhead[RE]{\sl \leftmark}
}
\fancypagestyle{mainmatter}{%
  	\fancyhf{} % Clear all headers and footers first
	\fancyhead[LE,RO]{\sl \thepage}
	\fancyhead[LO]{\sl \rightmark}
	\fancyhead[RE]{\sl \leftmark}
}
% SLO: font za ime avtorja
% ENG: font for author name
\newcommand{\authorfont}{\Large}
% SLO: font za naslov diplomskega dela
% ENG: font for thesis title
\newcommand{\titlefont}{\LARGE\bf}
% SLO: globina kazala
% ENG: content depth
\setcounter{tocdepth}{1}
% SLO: definiraj ukaz za prazno stran
% ENG: define the command for empty page
\newcommand{\clearemptydoublepage}{\newpage{\pagestyle{empty}\cleardoublepage}}

% course title
\newcommand{\trackname}{!undefined!}

\newcommand{\BibTeX}{{\sc Bib}\TeX}




%----------------------------------------------------------------------------------------------------------------------
% |||||||||||||||||||||| USTREZNO POPRAVI |||||||||||||||||||||||
% |||||||||||||||||||||| EDIT ACCORDINGLY |||||||||||||||||||||||
%----------------------------------------------------------------------------------------------------------------------
\newcommand{\ttitle}{Neprekinjena integracija in dostava poslovno kritičnih aplikacij}
\newcommand{\ttitleEn}{Continuous integration and delivery for business critical applications}
\newcommand{\tsubject}{\ttitle}
\newcommand{\tsubjectEn}{\ttitleEn}
\newcommand{\tauthor}{Jakob Maležič}
\newcommand{\temail}{jm6421@student.uni-lj.si}
\newcommand{\myyear}{2023}
\newcommand{\tkeywords}{neprekinjena integracija, neprekinjena dostava, poslovno kritične aplikacije}
\newcommand{\tkeywordsEn}{continuous integration, continuous deployment, business critical applications}
\newcommand{\mysupervisor}{doc.~dr.\ Nejc Ilc}
\newcommand{\mycosupervisor}{dr.\ Tadej Justin}

% include formatted front pages

%----------------------------------------------------------------
% SLO: definiraj metapodatke za datoteko master_thesis.tex
% ENG: define metadata for the file master_thesis.tex
%----------------------------------------------------------------
%----------------------------------------------------------------
%	HYPERREF SETUP
% SLO: ustrezno popravi e-mail
% ENG: edit the e-mail accordingly
%----------------------------------------------------------------
\hypersetup{pdftitle={\ttitle}}
\hypersetup{pdfsubject=\ttitleEn}
\hypersetup{pdfauthor={\tauthor, \temail}}
\hypersetup{pdfkeywords=\tkeywordsEn}

%----------------------------------------------------------------
% define medatata
% SLO: ustrezno popravi e-mail
% ENG: edit the e-mail accordingly
%----------------------------------------------------------------
\def\Title{\ttitle}
\def\Author{\tauthor, \temail}
\def\Subject{\ttitleEn}
\def\Keywords{\tkeywordsEn}
\def\Org{Univerza v Ljubljani, Fakulteta za računalništvo in informatiko}

%%%%%%%%%%%%%%%%%%%%%%%%%%%%%%%%%%%%%%%%
% \convertDate converts D:20080419103507+02'00' to 2008-04-19T10:35:07+02:00
%%%%%%%%%%%%%%%%%%%%%%%%%%%%%%%%%%%%%%%%
\def\convertDate{%
    \getYear
}

{\catcode`\D=12
 \gdef\getYear D:#1#2#3#4{\edef\xYear{#1#2#3#4}\getMonth}
}
\def\getMonth#1#2{\edef\xMonth{#1#2}\getDay}
\def\getDay#1#2{\edef\xDay{#1#2}\getHour}
\def\getHour#1#2{\edef\xHour{#1#2}\getMin}
\def\getMin#1#2{\edef\xMin{#1#2}\getSec}
\def\getSec#1#2{\edef\xSec{#1#2}\getTZh}
\def\getTZh +#1#2{\edef\xTZh{#1#2}\getTZm}
\def\getTZm '#1#2'{%
    \edef\xTZm{#1#2}%
    \edef\convDate{\xYear-\xMonth-\xDay T\xHour:\xMin:\xSec+\xTZh:\xTZm}%
}

\expandafter\convertDate\pdfcreationdate


%%%%%%%%%%%%%%%%%%%%%%%%%%%%%%%%%%%%%%%%
% get pdftex version string
%%%%%%%%%%%%%%%%%%%%%%%%%%%%%%%%%%%%%%%%
\newcount\countA
\countA=\pdftexversion
\advance \countA by -100
\def\pdftexVersionStr{pdfTeX-1.\the\countA.\pdftexrevision}

%%%%%%%%%%%%%%%%%%%%%%%%%%%%%%%%%%%%%%%%
% XMP data
%%%%%%%%%%%%%%%%%%%%%%%%%%%%%%%%%%%%%%%%
\usepackage{xmpincl}
\includexmp{pdfa-1b}

%%%%%%%%%%%%%%%%%%%%%%%%%%%%%%%%%%%%%%%%
% pdfInfo
%%%%%%%%%%%%%%%%%%%%%%%%%%%%%%%%%%%%%%%%
\pdfinfo{%
    /Title    (\ttitle)
    /Author   (\tauthor, \temail)
    /Subject  (\ttitleEn)
    /Keywords (\tkeywordsEn)
    /ModDate  (\pdfcreationdate)
    /Trapped  /False
}

%================================================================
% SLO: razno
% ENG: other
%================================================================
% SLO: nastavitev sklicevanj
% ENG: hyper referencing setup
\definecolor{black}{rgb}{0,0,0}
\hypersetup{
	colorlinks = true,
	linkcolor = black,
	citecolor = black,
	urlcolor = black
}

%----------------------------------------------------------------
% SLO: dodaj poti do datotek s slikami
% ENG: add paths to files containing figures
%----------------------------------------------------------------
\graphicspath{
	{figures/}
	{tables/}
}
%----------------------------------------------------------------
% SLO: moji paketi
% ENG: my packages
%----------------------------------------------------------------
% ...
%----------------------------------------------------------------
% SLO: moji konstrukti
% ENG: my constructs
%----------------------------------------------------------------
\newtheorem{izrek}{Izrek}[chapter]
\newtheorem{trditev}{Trditev}[izrek]
\newenvironment{dokaz}{\emph{Dokaz.}\ }{\hspace{\fill}{$\Box$}}

\newcommand{\CcImageCc}[1]{%
	\includegraphics[scale=#1]{cc-licenca/cc_cc_30.pdf}%
}
\newcommand{\CcImageBy}[1]{%
	\includegraphics[scale=#1]{cc-licenca/cc_by_30.pdf}%
}
\newcommand{\CcImageSa}[1]{%
	\includegraphics[scale=#1]{cc-licenca/cc_sa_30.pdf}%
}

%================================================================
% SLO: začetne strani magistrskega dela
% ENG: fist pages of the master's thesis
%================================================================
\begin{document}
% SLO: prepreči težave s številkami strani v kazalu
% ENG: prevents problems with the page numbers in the contents page
\renewcommand{\thepage}{}

%----------------------------------------------------------------
% Language-dependent formatting
%----------------------------------------------------------------
\ifSLO
    % SLO: definiraj slovensko besedo za kazalo
    \renewcommand{\contentsname}{Kazalo}

    % SLO: naslovnica
    % select the course title if it exist
\ifTRACKEXIST
    \ifTRACKCS
        \renewcommand{\trackname}{Računalništvo in Informatika}
    \else
        \renewcommand{\trackname}{Podatkovne Vede}
    \fi
\fi

\ifPROGRAMMM
    \thispagestyle{empty}
    \begin{center}
        {\large\sc Univerza v Ljubljani\\Fakulteta za računalništvo in informatiko\\
            Fakulteta za elektrotehniko}
    	   \vskip 10em
    	   {\authorfont \tauthor \par}
    	   {\titlefont \ttitle \par}
        {\vskip 2em \textsc{MAGISTRSKO DELO\\[2mm]
        MAGISTRSKI ŠTUDIJSKI PROGRAM DRUGE STOPNJE\\MULTIMEDIJA
        }\par}
        \vfill\null
        {\large \textsc{Mentor}: \mysupervisor \par}
   	    {\large \textsc{Somentor}: \mycosupervisor \par}
        {\vskip 2em \large Ljubljana, \myyear \par}
   \end{center}
\else
    \thispagestyle{empty}
	\begin{center}
            {\large\sc Univerza v Ljubljani\\Fakulteta za računalništvo in informatiko}
    	   \vskip 10em
    	   {\authorfont \tauthor \par}
    	   {\titlefont \ttitle \par}
        {\vskip 2em \textsc{MAGISTRSKO DELO\\[2mm]
        ŠTUDIJSKI PROGRAM DRUGE STOPNJE\\RAČUNALNIŠTVO IN INFORMATIKA
        \ifTRACKEXIST
            \\Smer: \trackname
        \fi
        }\par}
        \vfill\null
        {\large \textsc{Mentor}: \mysupervisor \par}
   	    {\large \textsc{Somentor}: \mycosupervisor \par}
        {\vskip 2em \large Ljubljana, \myyear \par}
   \end{center}
\fi  \clearemptydoublepage
    % SLO: avtorske pravice
    \thispagestyle{empty}
\vspace*{\fill}
{\noindent\footnotesize
{\sc Avtorske pravice}. 
\ifPROGRAMMM
    Rezultati magistrskega dela so intelektualna lastnina avtorja, Fakultete za ra\-ču\-nal\-niš\-tvo in informatiko ter Fakultete za Elektrotehniko Univerze v Ljubljani. Za objavljanje ali izkoriščanje rezultatov ma\-gi\-str\-ske\-ga dela je potrebno pisno soglasje avtorja, Fakultete za ra\-ču\-nal\-niš\-tvo in informatiko, Fakultete za Elektrotehniko ter mentorja\footnote{V dogovorju z mentorjem lahko kandidat magistrsko delo s pripadajočo izvorno kodo izda tudi pod drugo licenco, ki ponuja določen del pravic vsem: npr. Creative Commons, GNU GPL. V tem primeru na to mesto vstavite opis licence, na primer tekst~\cite{licence}.}.
\else
    Rezultati magistrskega dela so intelektualna lastnina avtorja in Fakultete za ra\-ču\-nal\-niš\-tvo in informatiko Univerze v Ljubljani. Za objavljanje ali izkoriščanje rezultatov ma\-gi\-str\-ske\-ga dela je potrebno pisno soglasje avtorja, Fakultete za ra\-ču\-nal\-niš\-tvo in informatiko ter mentorja\footnote{V dogovorju z mentorjem lahko kandidat magistrsko delo s pripadajočo izvorno kodo izda tudi pod drugo licenco, ki ponuja določen del pravic vsem: npr. Creative Commons, GNU GPL. V tem primeru na to mesto vstavite opis licence, na primer tekst~\cite{licence}.}.
\fi    
}
\begin{center}
{\footnotesize{\sc \copyright \myyear\ \tauthor}}
\end{center}  \clearemptydoublepage
    % SLO: izjava o avtorstvu (ni več del vezane izdaje, ločena oddaja)
    % SLO: zahvala
    \thispagestyle{empty}

\begin{center}
{\Large \textbf{\sc Zahvala}}
\end{center}
\vspace{0.5cm}

{\it\noindent
Rad bi se zahvalil svoji družini, še posebej svojim staršem, za vzpodbudo in podporo tekom mojega študija, zaročenki Maši za potrpežljivost in pomoč pri pisanju, mentorjema doc. dr. Nejcu Ilcu in dr. Tadeju Justinu za mentorstvo in nasvete, sodelavcu Roku Koleša za vso pomoč pri reševanju izzivov, in podjetju Medius, da mi je omogočilo pisanje magistrske naloge ter priskrbelo zanimivo temo.

\vspace{0.5cm} \hfill \tauthor, \myyear
} \clearemptydoublepage
    % SLO: posvetilo
    \thispagestyle{empty}\mbox{}{\vskip0.20\textheight}\mbox{}\hfill\begin{minipage}{0.55\textwidth}%

Vsem rožicam tega sveta.\\\\
\textit{''The only reason for time is so that everything doesn't happen at once.''}
\flushright --- Albert Einstein
\normalfont\end{minipage} \clearemptydoublepage
\else

    % ENG: title page ENG
    % select the course title if it exist
\ifTRACKEXIST
    \ifTRACKCS
        \renewcommand{\trackname}{Computer and Information Science}
    \else
        \renewcommand{\trackname}{Data Science}
    \fi
\fi

\ifPROGRAMMM
    \thispagestyle{empty}
	\begin{center}
        {\large\sc University of Ljubljana\\Faculty of Computer and Information Science\\
        Faculty of Electrical Engineering}
    	\vskip 10em
    	{\authorfont \tauthor \par}
    	{\titlefont \ttitleEn \par}
        {\vskip 2em \textsc{MASTER'S THESIS\\[2mm]
        THE 2nd CYCLE MASTER'S STUDY PROGRAMME\\MULTIMEDIA
        }\par}
        \vfill\null
        {\large \textsc{Supervisor}: \mysupervisor \par}
   	    {\large \textsc{Co-supervisor}:  \mycosupervisor \par}
        {\vskip 2em \large Ljubljana, \myyear \par}
   \end{center}
\else
    \thispagestyle{empty}
	\begin{center}
        {\large\sc University of Ljubljana\\Faculty of Computer and Information Science}
    	\vskip 10em
    	{\authorfont \tauthor \par}
    	{\titlefont \ttitleEn \par}
        {\vskip 2em \textsc{MASTER'S THESIS\\[2mm]
        THE 2nd CYCLE MASTER'S STUDY PROGRAMME\\COMPUTER AND INFORMATION SCIENCE
        \ifTRACKEXIST
            \\Track: \trackname
        \fi
        }\par}
        \vfill\null
        {\large \textsc{Supervisor}: \mysupervisor \par}
   	    {\large \textsc{Co-supervisor}:  \mycosupervisor \par}
        {\vskip 2em \large Ljubljana, \myyear \par}
    \end{center}
\fi  \clearemptydoublepage
    % ENG: title page SLO
    % select the course title if it exist
\ifTRACKEXIST
    \ifTRACKCS
        \renewcommand{\trackname}{Računalništvo in Informatika}
    \else
        \renewcommand{\trackname}{Podatkovne Vede}
    \fi
\fi

\ifPROGRAMMM
    \thispagestyle{empty}
    \begin{center}
        {\large\sc Univerza v Ljubljani\\Fakulteta za računalništvo in informatiko\\
            Fakulteta za elektrotehniko}
    	   \vskip 10em
    	   {\authorfont \tauthor \par}
    	   {\titlefont \ttitle \par}
        {\vskip 2em \textsc{MAGISTRSKO DELO\\[2mm]
        MAGISTRSKI ŠTUDIJSKI PROGRAM DRUGE STOPNJE\\MULTIMEDIJA
        }\par}
        \vfill\null
        {\large \textsc{Mentor}: \mysupervisor \par}
   	    {\large \textsc{Somentor}: \mycosupervisor \par}
        {\vskip 2em \large Ljubljana, \myyear \par}
   \end{center}
\else
    \thispagestyle{empty}
	\begin{center}
            {\large\sc Univerza v Ljubljani\\Fakulteta za računalništvo in informatiko}
    	   \vskip 10em
    	   {\authorfont \tauthor \par}
    	   {\titlefont \ttitle \par}
        {\vskip 2em \textsc{MAGISTRSKO DELO\\[2mm]
        ŠTUDIJSKI PROGRAM DRUGE STOPNJE\\RAČUNALNIŠTVO IN INFORMATIKA
        \ifTRACKEXIST
            \\Smer: \trackname
        \fi
        }\par}
        \vfill\null
        {\large \textsc{Mentor}: \mysupervisor \par}
   	    {\large \textsc{Somentor}: \mycosupervisor \par}
        {\vskip 2em \large Ljubljana, \myyear \par}
   \end{center}
\fi  \clearemptydoublepage
    % ENG: copyright
    \thispagestyle{empty}
\vspace*{\fill}
{\noindent\footnotesize
{\sc Copyright}. 
\ifPROGRAMMM
    The results of this master's thesis are the intellectual property of the author, the Faculty of Computer and Information Science and the Faculty of Electrical Engineering, University of Ljubljana. For the publication or exploitation of the master's thesis results, a written consent of the author, the Faculty of Computer and Information Science, the Faculty of Electrical Engineering and the supervisor is necessary.
    \footnote{V dogovorju z mentorjem lahko kandidat magistrsko delo s pripadajočo izvorno kodo izda tudi pod drugo licenco, ki ponuja določen del pravic vsem: npr. Creative Commons, GNU GPL. V tem primeru na to mesto vstavite opis licence, na primer tekst~\cite{licence}.}
\else
    The results of this master's thesis are the intellectual property of the author and the Faculty of Computer and Information Science, University of Ljubljana. For the publication or exploitation of the master's thesis results, a written consent of the author, the Faculty of Computer and Information Science, and the supervisor is necessary.
    \footnote{V dogovorju z mentorjem lahko kandidat magistrsko delo s pripadajočo izvorno kodo izda tudi pod drugo licenco, ki ponuja določen del pravic vsem: npr. Creative Commons, GNU GPL. V tem primeru na to mesto vstavite opis licence, na primer tekst~\cite{licence}.}
\fi    
}
\begin{center}
{\footnotesize{\sc \copyright \myyear\ \tauthor}}
\end{center}  \clearemptydoublepage
    % ENG: declaration of authorship (not part of paper edition, turn in separately)
    % ENG: acknowledgements
    \thispagestyle{empty}

\begin{center}
{\Large \textbf{\sc Acknowledgments}}
\end{center}
\vspace{0.5cm}

{\it\noindent
Worth mentioning in the acknowledgment is everyone who contributed to your thesis.

\vspace{0.5cm} \hfill \tauthor, \myyear
} \clearemptydoublepage
    % ENG: dedication
    \thispagestyle{empty}\mbox{}{\vskip0.20\textheight}\mbox{}\hfill\begin{minipage}{0.55\textwidth}%

To all the flowers of this world.\\\\
\textit{''The only reason for time is so that everything doesn't happen at once.''}
\flushright --- Albert Einstein
\normalfont\end{minipage} \clearemptydoublepage
\fi

%----------------------------------------------------------------
% SLO: kazalo
% ENG: contents
%----------------------------------------------------------------
\begingroup
	\hypersetup{colorlinks=true,linkcolor=black}
	\def\thepage{}
	\tableofcontents{}
	\clearemptydoublepage
\endgroup


\ifSLO
    % SLO: seznam kratic
    \chapter*{Seznam uporabljenih kratic}

\begin{tabular}{l|l|l}
  {\bf kratica} & {\bf angleško} & {\bf slovensko} \\ \hline
  % after \\: \hline or \cline{col1-col2} \cline{col3-col4} ...
  {\bf CRM} & Customer Relationship Management & upravljanje odnosov s strankami \\
  {\bf SCM} & Supply Chain Management & upravljanje oskrbovalne verige \\
  {\bf ERP} & Enterprise Resource Planning & upravljanje podjetniških virov \\
  {\bf BI} & Business Intelligence & poslovna inteligenca \\
  {\bf REST} & Representational state transfer & predstavitveni prenos stanja \\
  {\bf IaC} & Infrastructure as Code & infrastuktura kot koda \\
  {\bf API} & Application Programming Interface & aplikacijski programski vmesnik \\
  {\bf NPM} & Node Package Manager & upravitelj paketov Node \\
  {\bf PNPM} & Performant Node Package Manager & učinkovit upravitelj paketov Node \\
  {\bf CI/CD} & Continuos Integration and Deployment & neprekinjena integracija in  dostava \\
  
\end{tabular} \clearemptydoublepage
    % SLO: glavne strani diplomskega dela
\else
    % ENG: list of acronmys
    \chapter*{List of used acronmys}

\begin{tabular}{l|l|l}
  {\bf acronym} & {\bf meaning}  \\ \hline
  % after \\: \hline or \cline{col1-col2} \cline{col3-col4} ...
  {\bf CA} & classification accuracy \\
  {\bf DBMS} & database management system \\
  {\bf SVM} & support vector machine \\
  ... & ... \\
\end{tabular} \clearemptydoublepage
\fi

\frontmatter
\pagestyle{frontmatter}
\setcounter{page}{1} %
\renewcommand{\thepage}{}       % preprecimo težave s številkami strani v kazalu

% Include extended abstract [Razširjeni povzetek v slovenščini-- le za dela pisana v angleščini]
\ifSLO
    % include Slovenian abstract
    %---------------------------------------------------------------
% SLO: slovenski povzetek
% ENG: slovenian abstract
%---------------------------------------------------------------
\selectlanguage{slovene} % Preklopi na slovenski jezik
\addcontentsline{toc}{chapter}{Povzetek}
\chapter*{Povzetek}

\noindent\textbf{Naslov:} \ttitle
\bigskip

V magistrski nalogi predstavimo proces neprekinjene integracije in dostave (ang. continuous integration and delivery - CI/CD), ki predstavlja pomemben del razvoja poslovno kritičnih aplikacij. V svetu obstaja veliko aplikacij in programov, ki so namenjeni prav tej nalogi. Vendar vsak izmed njih ni primeren za uporabo pri razvoju poslovno kritičnih aplikaciji. V tej nalogi smo najprej izpostavili tehnologije, ki se splošno uporabljajo v okviru procesa CI/CD. Le-te smo preverili in izbrali primerne za vpeljavo v projekte poslovno kritičnih aplikacij. Dodatno smo izpostavili njihove prednosti in pomanjkljivosti pri uporabi. Na podlagi ugotovitev smo razvili komponento, ki omogoča enostavno povezovanje vseh izbranih tehnologij in skuša odpraviti izpostavljene pomanjkljivosti.

Razvili smo komponento za vzpostavitev CI/CD z orodjem GitLab CI/CD imenovano Medius CD. S komponento smo poenotili cevovode CI/CD, zmanjšali podvojenost konfiguracije in kode ter olajšali vzdrževanje. Razvita komponenta je tudi zelo prilagodljiva različnim zahtevam naročnikov in omogoča enostavno vzpostavitev cevovoda CI/CD tako v razvojnem okolju kot tudi v okolju naročnika.

Med razvojem smo odkrili potrebo po vtičniku za integrirano okolje IntelliJ, ki bi preverjal konfiguracijske datoteke orodja GitLab CI/CD. Zato smo razvili odprtokoden vtičnik Gitlab Template Lint, ki v integriranem okolju prikazuje napake in združeno vsebino konfiguracijskih datotek. Razvit vtičnik smo objavili na tržnico vtičnikov JetBrains in na ta način omogočili njegovo uporabo velikemu številu razvijalcev DevOps po vsem svetu.

V sklopu naloge smo koncepte in funkcionalnosti komponente ponazorili s pomočjo študij primerov iz resničnih projektov poslovno kritičnih aplikacij, ki uporabljajo različne programske jezike. S tem smo pokazali praktično uporabnost razvite rešitve ter njeno sposobnost prilagajanja različnim scenarijem in programskim jezikom v poslovnem okolju.


% Opišemo poslovno kritične aplikacije in proces neprekinjene integracije in dostave. Predstavimo izbrane tehnologije in aplikacije, ki jih v nadaljevanju uporabimo za vzpostavitev cevovoda neprekinjene integracije in dostave. Poseben poudarek namenimu razvoju komponente Medius CD, ki olajšuje vzpostavitev cevovoda neprekinjene integracije in dostave z orodjem Gitlab CI/CD.

% Razvita komponenta je sestavljena iz dveh ključnih delov: prvi del vključuje pripravljene predloge za uporabo v okviru projektov GitLab, medtem ko drugi del obsega skripte v programskem jeziku Bash. Te skripte se uporabljajo za izvajanje bolj kompleksnih nalog v okviru opredeljenih predlog. Ena izmed glavnih prednosti razvite komponente je njena visoka prilagodljivost, saj omogoča prilagajanje različnim zahtevam projektov. Dodana konfiguracijska datoteka igra ključno vlogo pri določanju specifičnih zahtev za posamezen projekt.

% Razvijemo komponento Medius CD, ki olajša vzpostavitev cevovoda neprekinjene integracije in dostave na platformi Gitlab. Komponento razdelimo na dva dela: predloge Gitlab, ki jih lahko enostavno pouporabimo v projektih in skripte v programskem jeziku Bash, ki jih predloge uporabljajo za izvajanje bolj kompleksnih nalog. Razvita komponenta omogoča visoko prilagodljivost različnim projektnim zahtevam, ki jih je mogoče definirati v konfiguracijski datoteki. Slednje pokažemo na primerih projektov iz resničnega sveta. 

% V vzorcu je predstavljen postopek priprave magistrskega dela z uporabo okolja \LaTeX. Vaš povzetek mora sicer vsebovati približno 100 besed, ta tukaj je odločno prekratek. Dober povzetek vključuje: (1) kratek opis obravnavanega problema, (2) kratek opis vašega pristopa za reševanje tega problema in (3) (najbolj uspešen) rezultat ali prispevek magistrske naloge.

\subsection*{Ključne besede}
\textit{\tkeywords}
\clearemptydoublepage

    % include English abstract
     %---------------------------------------------------------------
% SLO: angleški povzetek
% ENG: english abstract
%---------------------------------------------------------------
\selectlanguage{english} % Preklopi na angleški jezik
\addcontentsline{toc}{chapter}{Abstract}
\chapter*{Abstract}

\noindent\textbf{Title:} \ttitleEn
\bigskip

In this master's thesis we present the process of continuous integration and delivery (CI/CD), which is a crucial part of developing business-critical applications. There are many applications and tools worldwide designed for this purpose. However, not all of them are suitable for developing business-critical applications. We first highlighted the technologies commonly used within the CI/CD process. We examined them and selected those suitable for implementation in business-critical application projects. Additionally, we emphasized their advantages and drawbacks in their usage. Based on these findings, we developed a component that facilitates the seamless integration of the selected technologies and aims to address the identified shortcomings.

We developed a component for establishing CI/CD using the GitLab CI/CD tool, named Medius CD. With this component, we standardized CI/CD pipelines, reduced configuration and code duplication, and simplified maintenance. The developed component is highly adaptable to various customer requirements and enables easy setup of CI/CD pipelines in both the development environment and the customer's environment.

During development, we identified the need for a plugin for the integrated development environment IntelliJ IDEA that verifies the configuration files of the GitLab CI/CD tool. Therefore, we developed an open-source plugin called Gitlab Template Lint, which displays errors and consolidated content of configuration files in the integrated environment. We published the plugin on the JetBrains plugin marketplace, thus enabling its use to a large number of DevOps developers worldwide.

As part of the assignment, we illustrated the concepts and functionalities of the component through case studies from real projects of business-critical applications that use different programming languages. This demonstrated the practical usability of the developed solution and its ability to adapt to various scenarios and programming languages in the business environment.

% We describe business-critical applications and their characteristics. We present the process of continuous integration and delivery, which is an important part of developing business-critical applications. We describe the technologies that allow us to introduce best development practices and use them within the CI/CD process. 

% We develop a component for establishing CI/CD with the GitLab CI/CD tool, called Medius CD. With this component, we standardize CI/CD pipelines, reduce configuration and code duplication, and simplify maintenance. The developed component is highly adaptable to various customer requirements and enables easy setup of CI/CD pipelines in both the development environment and the customer's environment.

% As part of the task, we illustrate the concepts and functionalities of the component with case studies from real projects of business-critical applications that use different programming languages. This demonstrates the practical usability of the developed solution and its ability to adapt to various scenarios in the business environment.

% During development, we discover the need for a plugin for the integrated development environment IntelliJ IDEA that checks the configuration files of the GitLab CI/CD tool. Therefore, we develop an open-source plugin called Gitlab Template Lint, which displays errors and consolidated content of configuration files in the integrated development environment. We publish the developed plugin on the JetBrains plugin marketplace, thus enabling its use by a large number of DevOps developers worldwide.

\subsection*{Keywords}
\textit{\tkeywordsEn}
\clearemptydoublepage
\else
    % include English abstract
     %---------------------------------------------------------------
% SLO: angleški povzetek
% ENG: english abstract
%---------------------------------------------------------------
\selectlanguage{english} % Preklopi na angleški jezik
\addcontentsline{toc}{chapter}{Abstract}
\chapter*{Abstract}

\noindent\textbf{Title:} \ttitleEn
\bigskip

In this master's thesis we present the process of continuous integration and delivery (CI/CD), which is a crucial part of developing business-critical applications. There are many applications and tools worldwide designed for this purpose. However, not all of them are suitable for developing business-critical applications. We first highlighted the technologies commonly used within the CI/CD process. We examined them and selected those suitable for implementation in business-critical application projects. Additionally, we emphasized their advantages and drawbacks in their usage. Based on these findings, we developed a component that facilitates the seamless integration of the selected technologies and aims to address the identified shortcomings.

We developed a component for establishing CI/CD using the GitLab CI/CD tool, named Medius CD. With this component, we standardized CI/CD pipelines, reduced configuration and code duplication, and simplified maintenance. The developed component is highly adaptable to various customer requirements and enables easy setup of CI/CD pipelines in both the development environment and the customer's environment.

During development, we identified the need for a plugin for the integrated development environment IntelliJ IDEA that verifies the configuration files of the GitLab CI/CD tool. Therefore, we developed an open-source plugin called Gitlab Template Lint, which displays errors and consolidated content of configuration files in the integrated environment. We published the plugin on the JetBrains plugin marketplace, thus enabling its use to a large number of DevOps developers worldwide.

As part of the assignment, we illustrated the concepts and functionalities of the component through case studies from real projects of business-critical applications that use different programming languages. This demonstrated the practical usability of the developed solution and its ability to adapt to various scenarios and programming languages in the business environment.

% We describe business-critical applications and their characteristics. We present the process of continuous integration and delivery, which is an important part of developing business-critical applications. We describe the technologies that allow us to introduce best development practices and use them within the CI/CD process. 

% We develop a component for establishing CI/CD with the GitLab CI/CD tool, called Medius CD. With this component, we standardize CI/CD pipelines, reduce configuration and code duplication, and simplify maintenance. The developed component is highly adaptable to various customer requirements and enables easy setup of CI/CD pipelines in both the development environment and the customer's environment.

% As part of the task, we illustrate the concepts and functionalities of the component with case studies from real projects of business-critical applications that use different programming languages. This demonstrates the practical usability of the developed solution and its ability to adapt to various scenarios in the business environment.

% During development, we discover the need for a plugin for the integrated development environment IntelliJ IDEA that checks the configuration files of the GitLab CI/CD tool. Therefore, we develop an open-source plugin called Gitlab Template Lint, which displays errors and consolidated content of configuration files in the integrated development environment. We publish the developed plugin on the JetBrains plugin marketplace, thus enabling its use by a large number of DevOps developers worldwide.

\subsection*{Keywords}
\textit{\tkeywordsEn}
\clearemptydoublepage
    % include Slovenian abstract
    %---------------------------------------------------------------
% SLO: slovenski povzetek
% ENG: slovenian abstract
%---------------------------------------------------------------
\selectlanguage{slovene} % Preklopi na slovenski jezik
\addcontentsline{toc}{chapter}{Povzetek}
\chapter*{Povzetek}

\noindent\textbf{Naslov:} \ttitle
\bigskip

V magistrski nalogi predstavimo proces neprekinjene integracije in dostave (ang. continuous integration and delivery - CI/CD), ki predstavlja pomemben del razvoja poslovno kritičnih aplikacij. V svetu obstaja veliko aplikacij in programov, ki so namenjeni prav tej nalogi. Vendar vsak izmed njih ni primeren za uporabo pri razvoju poslovno kritičnih aplikaciji. V tej nalogi smo najprej izpostavili tehnologije, ki se splošno uporabljajo v okviru procesa CI/CD. Le-te smo preverili in izbrali primerne za vpeljavo v projekte poslovno kritičnih aplikacij. Dodatno smo izpostavili njihove prednosti in pomanjkljivosti pri uporabi. Na podlagi ugotovitev smo razvili komponento, ki omogoča enostavno povezovanje vseh izbranih tehnologij in skuša odpraviti izpostavljene pomanjkljivosti.

Razvili smo komponento za vzpostavitev CI/CD z orodjem GitLab CI/CD imenovano Medius CD. S komponento smo poenotili cevovode CI/CD, zmanjšali podvojenost konfiguracije in kode ter olajšali vzdrževanje. Razvita komponenta je tudi zelo prilagodljiva različnim zahtevam naročnikov in omogoča enostavno vzpostavitev cevovoda CI/CD tako v razvojnem okolju kot tudi v okolju naročnika.

Med razvojem smo odkrili potrebo po vtičniku za integrirano okolje IntelliJ, ki bi preverjal konfiguracijske datoteke orodja GitLab CI/CD. Zato smo razvili odprtokoden vtičnik Gitlab Template Lint, ki v integriranem okolju prikazuje napake in združeno vsebino konfiguracijskih datotek. Razvit vtičnik smo objavili na tržnico vtičnikov JetBrains in na ta način omogočili njegovo uporabo velikemu številu razvijalcev DevOps po vsem svetu.

V sklopu naloge smo koncepte in funkcionalnosti komponente ponazorili s pomočjo študij primerov iz resničnih projektov poslovno kritičnih aplikacij, ki uporabljajo različne programske jezike. S tem smo pokazali praktično uporabnost razvite rešitve ter njeno sposobnost prilagajanja različnim scenarijem in programskim jezikom v poslovnem okolju.


% Opišemo poslovno kritične aplikacije in proces neprekinjene integracije in dostave. Predstavimo izbrane tehnologije in aplikacije, ki jih v nadaljevanju uporabimo za vzpostavitev cevovoda neprekinjene integracije in dostave. Poseben poudarek namenimu razvoju komponente Medius CD, ki olajšuje vzpostavitev cevovoda neprekinjene integracije in dostave z orodjem Gitlab CI/CD.

% Razvita komponenta je sestavljena iz dveh ključnih delov: prvi del vključuje pripravljene predloge za uporabo v okviru projektov GitLab, medtem ko drugi del obsega skripte v programskem jeziku Bash. Te skripte se uporabljajo za izvajanje bolj kompleksnih nalog v okviru opredeljenih predlog. Ena izmed glavnih prednosti razvite komponente je njena visoka prilagodljivost, saj omogoča prilagajanje različnim zahtevam projektov. Dodana konfiguracijska datoteka igra ključno vlogo pri določanju specifičnih zahtev za posamezen projekt.

% Razvijemo komponento Medius CD, ki olajša vzpostavitev cevovoda neprekinjene integracije in dostave na platformi Gitlab. Komponento razdelimo na dva dela: predloge Gitlab, ki jih lahko enostavno pouporabimo v projektih in skripte v programskem jeziku Bash, ki jih predloge uporabljajo za izvajanje bolj kompleksnih nalog. Razvita komponenta omogoča visoko prilagodljivost različnim projektnim zahtevam, ki jih je mogoče definirati v konfiguracijski datoteki. Slednje pokažemo na primerih projektov iz resničnega sveta. 

% V vzorcu je predstavljen postopek priprave magistrskega dela z uporabo okolja \LaTeX. Vaš povzetek mora sicer vsebovati približno 100 besed, ta tukaj je odločno prekratek. Dober povzetek vključuje: (1) kratek opis obravnavanega problema, (2) kratek opis vašega pristopa za reševanje tega problema in (3) (najbolj uspešen) rezultat ali prispevek magistrske naloge.

\subsection*{Ključne besede}
\textit{\tkeywords}
\clearemptydoublepage


  %  \cleardoublepage
    \let\oldthesection=\thesection %Special section numbering for this chapter - remember default one
    \let\oldthesubsection=\thesubsection
    \renewcommand{\thesection}{\Roman{section}} %Special section numbering for this chapter
    \renewcommand{\thesubsection}{\thesection.\Roman{subsection}}

    % set roman page numbering
    \pagenumbering{roman}
    % set slovene language
    \selectlanguage{slovene}
    % insert extended abstract
     \chapter{Razširjeni povzetek}
 
 To je primer razširjenega povzetka v slovenščini, ki je obvezen za naloge pisane v angleščini. Razširjeni povzetek mora vsebovati vse glavne elemente dela napisanega v angleščini skupaj s kratkim uvodom in povzetkom glavnih elementov metode, glavnih eksperimentalnih rezultatov in glavnih ugotovitev. Razširjeni povzetek naj bo strukturiran v podpoglavja (spodaj je naveden le okvirni primer in je nezavezujoč).
 Čez palec navadno razširjeni povzetek nanese okoli 10 odstotkov obsega celotnega dela. 
 
 \section{Kratek pregled sorodnih del}
 
 \section{Predlagana metoda}
 
 \section{Eksperimentalna evaluacija}
 
 \section{Sklep}
 
poljuben tekst  poljuben tekst  poljuben tekst  poljuben tekst  poljuben tekst  poljuben tekst  poljuben tekst  poljuben tekst  poljuben tekst  poljuben tekst  poljuben tekst  poljuben tekst  poljuben tekst  poljuben tekst  poljuben tekst  poljuben tekst  poljuben tekst  poljuben tekst  poljuben tekst  poljuben tekst  poljuben tekst  poljuben tekst  poljuben tekst  poljuben tekst  poljuben tekst  poljuben tekst  poljuben tekst  poljuben tekst  poljuben tekst  poljuben tekst  poljuben tekst  poljuben tekst  poljuben tekst  poljuben tekst  poljuben tekst  poljuben tekst  poljuben tekst  poljuben tekst  poljuben tekst  poljuben tekst  poljuben tekst  poljuben tekst  poljuben tekst  poljuben tekst  poljuben tekst  poljuben tekst  poljuben tekst  poljuben tekst  poljuben tekst  poljuben tekst  poljuben tekst  poljuben tekst  poljuben tekst  poljuben tekst  poljuben tekst  poljuben tekst  poljuben tekst  poljuben tekst  poljuben tekst  poljuben tekst  poljuben tekst  poljuben tekst  poljuben tekst  poljuben tekst  poljuben tekst  poljuben tekst  poljuben tekst  poljuben tekst  poljuben tekst  poljuben tekst  poljuben tekst  poljuben tekst  poljuben tekst  poljuben tekst  poljuben tekst  poljuben tekst  poljuben tekst  poljuben tekst  poljuben tekst  poljuben tekst  poljuben tekst  poljuben tekst  poljuben tekst  poljuben tekst  poljuben tekst  poljuben tekst  poljuben tekst  poljuben tekst  poljuben tekst  poljuben tekst 


    \let\thesection=\oldthesection % Restore default section numbering
    \let\thesubsection=\oldthesubsection
\fi

%----------------------------------------------------------------
% SLO: Preklopi izbrani jezik
% ENG: Switch to chosen language
%----------------------------------------------------------------
\ifSLO
    \selectlanguage{slovene} % Preklopi na slovenski jezik
\else
    \selectlanguage{english}  % Switch to english language
\fi

% SLO: vklopi številčenje poglavji, ponastavi številčenje strani in uporabi arabske številkami za številčenje strani
% ENG: turns on chapter numbering, resets page numbering and uses arabic numerals for page numbers
\mainmatter
\pagestyle{mainmatter}
\setcounter{page}{1}
\pagestyle{fancy}


%================================================================
% ENG: main pages of the thesis
%================================================================

%----------------------------------------------------------------
% Poglavje 1 - Uvod in motivacija
%----------------------------------------------------------------


\chapter{Uvod}
\label{ch:uvod}

Pri razvoju večje programske opreme ali aplikacije je programje navadno razdeljeno na več zaključenih enot, ki jih
različne razvojne skupine ali razvojna podjetja neodvisno razvijajo.\ Za slednje je pred delitvijo nalog vnaprej
izbrano in specificirano tudi produkcijsko okolje.\ Na ta način si lahko razvojne skupine, za preverjanje realizacije
izvedenega dela, pripravijo razvojno okolje, ki ima čim bolj podobno infrastrukturo, kot je na voljo v produkcijskem
okolju.\ Samo tako so posamezne skupine lahko dovolj samozavestne, da bo njihovo programje pravilno delovalo tudi v
produkcijskem okolju.

\section{Motivacija}
\label{sec:motivacija}

Poseben primer produkcijskega okolja predstavlja zaprto produkcijsko okolje stranke, ki se velikokrat pojavi pri
poslovno kritičnih aplikacijah. Poslovno kritične aplikacije so tiste aplikacije, ki so ključne za delovanje
poslovnega procesa podjetja~\cite{Hinchey2010, Syng2016}. V takšnem primeru je zaradi varnosti produkcijsko okolje
pogosto razvijalcem nedostopno, kot tudi celotno omrežje stranke. S tem je dostava programske kode v omrežje
stranke s strani razvijalcev nemogoča. Za takšen primer razvijanja programske opreme je potrebno celoten proces
neprekinjene dostave prilagoditi in upoštevati morebitne dodatne zaplete pri izdaji programske opreme v produkcijsko
okolje.

\section{Cilji}
\label{sec:cilji}

V magistrski nalogi bomo predstavili strategije in tehnologije, ki nam omogočajo postavitev takšnega sistema. Slednje
bomo opisali na realnih projektih in pripravili orodja za izvajanje integracijskega ter dostavnega procesa. Bolj
podrobno bomo opisali tudi proces neprekinjene integracije in dostave v okolje, kjer nimamo dostopa do nobenega izmed
programskih okolij in nimamo možnosti odriva programske kode v strankino omrežje.


%----------------------------------------------------------------
% Poglavje 2 - Teorija
%----------------------------------------------------------------


\chapter{Teoretična osnova}
\label{ch:teoretična-osnova}

\section{Poslovno kritične aplikacije}
\label{sec:poslovno-kritične-aplikacije}
Poslovno kritične aplikacije so aplikacije, ki so nujne za delovanje podjetja. To so tiste aplikacije, ki bi v primeru odpovedi ali prekinitve delovanja močno vplivale na poslovanje podjetja in potencialno povzročile veliko finančne škode ali škodovale ugledu podjetja \cite{Shen2015}. Kritičnost aplikacije, ki ga neko podjetje uporablja, je odvisno od podjetja in narave dela, ki ga to podjetje opravlja. Primeri poslovno kritičnih aplikacij so \cite{Shen2015}:
\begin{itemize}
    \item Aplikacije za elektronsko poslovanje: Spletne aplikacije, ki omogočajo podjetjem da tržijo svoje izdelke in storitve preko spleta.
    \item Aplikacije za upravljanje s strankami (CRM): Aplikacije, ki pomagajo podjetjem upravljati poslovanje s strankami in hranijo podatke o strankah.
    \item Dobavni sistemi (SCM): Sistemi, ki pomagajo podjetju upravljati z dobavo in prodajo, kot tudi z beleženjem inventarja in logistiko.
    \item Sistem za načrtovanje virov (ERP): Sistem za nadzor poslovanja podjetja.
    \item Sistemi poslovne inteligence: To so aplikacije ki pomagajo podjetju zbirati, hraniti in analizirati podatke za sklepanje boljših poslovnih odločitev.
    \item Orodja za komunikacijo: Aplikacije, ki zaposlenim omogočajo komunikacijo kot na primer: elektronska pošta, sporočilni sistemi in aplikacije za video konference.
\end{itemize}

Pri razvoju poslovno kritičnih aplikacij so pomembne naslednje lastnosti \cite{Sarkis2004}:
\begin{itemize}
    \item Zanesljivost: Aplikacija mora biti vedno na voljo in delovati konsistentno brez napak in zaustavitev.
    \item Skalabilnost (razširljivost?): Ob rasti podjetja se količina prometa, ki jih mora aplikacija obdelati, lahko poveča. Aplikacija mora zato biti sposobna obvladovati velike količine prometa in podpreti rastoče zahteve podjetja.
    \item Varnost: Poslovno kritične aplikacije pogosto hranijo občutljive podatke, na primer podatke o uporabnikih ali finančne podatke. Aplikacija mora zato biti varna in zaščitena pred kibernetskimi napadi, nepooblaščenim dostopom in drugim tveganji, da prepreči izlive takšnih podatkov.
    \item Zmogljivost: Počasna ali neodzivna aplikacija negativno vpliva na produktivnosti. Aplikacija mora zato biti odzivna in delovati dobro tudi pod velikimi obremenitvami.
    \item Vzdrževanje: Dobro vzdrževanje aplikacije je pomembno, da aplikacija ostane zanesljiva in zmogljiva. Aplikacija mora zato biti lahka za vzdrževanje in posodabljanje kot tudi jasno in dobro dokumentirana.
    \item Uporabniška izkušnja: Dober uporabniški vmesnik in izkušnja lahko uporabnikom olajšata razumevanje in uporabo aplikacije, kar poveča produktivnost.
    % \item Integracija: Aplikacija mora nuditi dober programski vmesnik, da se lahko druge aplikacije nanj integrirajo.
    \item Prilagodljivost: Različna podjetja imajo različne potrebe, zato je pomembno, da je aplikacija prožna in prilagodljiva.
\end{itemize}

% Podjetja pogosto razdelijo svoje aplikacije v skupine, glede na velikost posledic, ki bi jih njihova odpoved prinesla. V večini primerov se delijo na: aplikacije s kritično nalogo, poslovno kritične aplikacije in ne-kritične aplikacije. Ločimo jih po škodi, ki jih povzroči njihov izpad.

% \subsection{Aplikacije s kritično nalogo}
% Odpovedi aplikacije s kritično nalogo, največkrat privede do napake pri doseganju nekega pomembnega cilja. Na primer: reševanje življenj, preprečevanje resnih poškodb, transport nujnih stvari ipd.

% Primerjava poslovno kritičnih z aplikacijami s kritično nalogo.

\section{Neprekinjena integracija}
\label{sec:neprekinjena-integracija}
Neprekinjena integracija je praksa razvoja programske opreme, pri kateri programerji redno združujejo svoje spremembe kode v centralni repozitorij. Ta koda se nato avtomatsko zgradi, preizkusi in objavi \cite{Travassos2016}. Cilj neprekinjene integracije je čim prej avtomatsko odkriti napake, tako da lahko programerji posvečajo več časa pisanju kode kot iskanju napak \cite{Fowler2006}. Avtomatska gradnja, preizkušanje in objavljanje kode tudi pomaga ekipam, da objavijo posodobitve pogosteje in z večjim zaupanjem, kar je še posebej pomembno v hitro spreminjajočih se okoljih razvoja \cite{Fowler2006}.

Za sistem neprekinjene integracije in dostave so potrebni trije glavni elementi: centralni repozitorij, orodje za avtomatizacijo gradnje in orodje za preizkušanje \cite{Travassos et al., 2016}. Centralni repozitorij je mesto, kamor razvijalci oddajo spremembe svoje kode. Orodje za avtomatizacijo gradnje je odgovorno za avtomatsko gradnjo in objavljanje posodobljene kode. Orodje za preizkušanje pa izvaja avtomatsko testiranje na posodobljeni kodi, s čimer zagotovi, da posodobljena koda deluje pravilno in dosega zahtevane standarde kakovosti \cite{Fowler2006}.

% Travassos, G. H., Fernandes, E. B., de O. Costa, R. F., & Maldonado, J. C. (2016). Continuous integration: Best practices, patterns, and anti-patterns. Journal of Systems and Software, 120, 1-16.

% Neprekinjena integracija pri razvoju programske opreme predstavlja večkratno dnevno združevanje delovnih različič programerjev v skupno celoto \cite{Fowler2006}.
% Fowler, Martin (1 May 2006). "Continuous Integration". Retrieved 9 January 2014.

\subsection{Razlogi za neprekinjeno integracijo}
Ko razvijalec želi dodati nekaj novega v programsko kodo aplikacije, si v svojem okolju ustvari kopijo temeljne programske kode, ki je v tistem trenutku aktualna. Medtem ko razvijalec v svojem okolju razvija nove funkcionalnosti ali pripravlja popravke, lahko ostali razvijalci spremenijo temeljno programsko kodo. Zato se lokalna kopija razvijalca čedalje bolj razlikuje od temeljne programske kode. Še več, razvijalci lahko v temeljno programsko kodo dodajo nove funkcionalnosti, nove knjižnice ali druge vire, ki lahko ustvarijo dodatne odvisnosti in potencialne konflikte.

Dalj časa kot razvijalec svoje lokalne kopije ne združi s temeljno programsko kodo, večje je tveganje integracijskih konfliktov in napak pri združevanju kode \cite{Duvall2007}. Preden razvijalec svoje spremembe doda v glavno vejo, mora najprej posodobiti svojo lokalno kopijo, da pridobi vse spremembe, ki so bile v vmesnem času dodane v glavno vejo. Več kot je bilo sprememb dodanih v vmesnem času, več dela ima razvijalec, preden svoje spremembe doda v glavno vejo.


Če razvijalec predolgo odlaša z oddajo kode v glavno vejo, se njegova kopija lahko začne zelo razlikovati od kode v glavni veji in za integracijo svoje kode v glavno vejo porabi več časa, kot ga je za razvoj sprememb. Temu rečemo tudi "integracijski pekel" \cite{Cunningham2009}.

Implementacija neprekinjene integracije lahko prinese številne koristi pri razvoju programske opreme, vključno z izboljšano kakovostjo kode, hitrejšim objavljanjem posodobitev in zmanjšanjem tveganja napak pri integraciji in objavi \cite{Travassos et al., 2016}. Vendar pa implementacija neprekinjene integracije predstavlja tudi nekaj izzivov, kot je potreba po namestitvi in vzdrževanju infrastrukture za neprekinjeno integracijo in potreba po zagotavljanju učinkovitega postopka preizkušanja programske kode \cite{Rios in da Costa, 2017}. 

\section{Neprekinjena dostava}
\label{sec:neprekinjena-dostava}

Neprekinjena dostava je metoda razvoja in izdelave, ki temelji na stalnem izboljševanju procesov in izdelkov ter zagotavljanju neprekinjene dostave izdelkov ali storitev. To pomeni, da se procesi in izdelki izboljšujejo in nadgrajujejo neprekinjeno ter dostava izdelkov ali storitev ne povzroči prekinitev ali zastoja v procesu. 

Cilj neprekinjene dostave je zmanjšanje časa med pisanjem kode in njene dostave končnim uporabnikom, obenem pa zagotavljanje visoke kakovosti in zanesljivosti \cite{Humble2010}.

Neprekinjena dostava je zato širši pomen, ki vključuje neprekinjeno integracijo in avtomatično testiranje, obenem pa tudi izdajanje novih verzij in avtomatično objavo aplikacije v testno ali produkcijsko okolje. Pomemben del pa je tudi sistem za spremljanje napak in težav z zmogljivostjo, ki razvojni ekipi pošilja povratne informacije o delovanju aplikacije, ki jim pomaga identificirati in odpraviti napake \cite{Humble2010}.

\subsection{Dostava izvorne kode}
\label{subsec:dostava-izvorne-kode}
Pomemben del neprekinjene dostave je dostava izvorne kode stranki ali končnim uporabnikom. Proces prenosa kode mora biti dobro zasnovan, da zagotovimo, da se izvorna koda pravilno in učinkovito prenese.

Dostava izvorne kode stranki je lahko koristna tako za stranko kot tudi za podjetje, ki je naredilo izvorno kodo. Za stranko je koristno, ker ni več odvisno od podjetja iz vidika posodobitev, vzdrževanja in podpore, vendar za to lahko najame drugo podjetje ali pa ta del sami prevzamejo. To je lahko zelo koristno predvsem takrat, ko podjetje, ki je ustvarilo izvorno kodo, preneha s svojim poslovanjem ali pa ni zmožno zagotoviti podpore zaradi kakšnega drugega razloga. 

% Skupaj lahko dostava izvorne kode stranki zagotovi večjo neodvisnost in prilagodljivost pri uporabi programske opreme, pa tudi priložnost prilagoditve programske opreme lastnim specifičnim potrebam in zahtevam (Humble et al., 2010).

Dostava izvorne kode pa je koristna tudi za podjetje, ki je izvorno kodo naredilo predvsem iz vidika, da podjetje ni več odgovorno za izvorno kodo:
\begin{itemize}
    \item Prenos odgovornosti: z dostavo izvorne kode stranki lahko podjetje prenese del odgovornosti programske opreme na stranko. To je lahko še posebej koristno, če stranka načrtuje prilagoditve ali spremembe programske opreme, saj podjetje ne bi odgovorno za morebitne težave ali napake, ki bi lahko nastali zaradi teh sprememb \cite{Humble2010}.
    \item Zmanjšanje bremena vzdrževanja in podpore: dostava izvorne kode lahko zmanjša breme vzdrževanja in podpore podjetja. Če ima stranka dostop do izvorne kode, lahko sama odpravi težave in jih popravi, namesto da bi se zanašala na podjetje za podporo \cite{Humble2010}.
\end{itemize}

\subsection{Neprekinjena namestitev}
\label{subsec:neprekinjena-namestitev}
Najširši obseg pa predstavlja neprekinjena namestitev, ki je praska razvoja programske opreme, pri kateri se spremembe kode avtomatsko zgradijo, preizkusijo, oddajo naročniku in tudi objavijo v produkcijsko okolje brez potrebe po človeškem posredovanju \cite{Travassos2016}. Predstavlja še en korak naprej od neprekinjene dostave, pri kateri se koda zgradi in preizkusi, vendar jih mora človek ročno objaviti v produkcijo \cite{Bass2015}.

Cilj neprekinjene namestitve je, da se nove funkcionalnosti in popravki čim hitreje dostavijo končnim uporabnikom \cite{Bass2015}. Da bi to dosegli, uporabljamo neprekinjeno integracijo, neprekinjeno dostavo in specifične prakse neprekinjene namestitve, kot je proces upravljanja in rezervacije računalniških virov z uporabo datotek, ki jih računalnik zna prebrati \cite{Wittig2016} - infrastruktura kot koda\cite{Humble2014}.

Neprekinjena namestitev lahko pomaga podjetju, da pogosteje objavi nove funkcionalnosti in posodobitve \cite{Jiang2016} in izboljša učinkovitost in hitrost razvojnega procesa aplikacije \cite{Travassos2016}.

% Programerjem omogoča testiranje aplikacije v živo? Quality assurance?

\section{Struktura projekta}
\label{sec:struktura-projekta}
Struktura projekta opisuje kako je projekt organiziran in razmerja med različnimi deli projekta. Dobro strukturiran projekt pripomore k boljšemu razumevanju, vzdrževanju in razvijanju programske kode projekta. Obstaja veliko načinov strukturiranja projektov, seveda pa je struktura projekta odvisna od specifičnih potreb in ciljev projekta. Pri strukturiranju projekta so ključni deli:
\begin{itemize}
    \item Direktorijska struktura: Opisuje kako so datoteke in direktoriji razporejeni na datotečnem sistemu. Dobra direktorijska struktura združuje odvisne datoteke in vire ter olajša iskanje datotek.
    \item Moduli in odvisnosti: V večini projektov, je programska koda logično razdeljena na več modulov ali komponent, kjer vsak izmed modulov zadolžen za en del projekta. Ti moduli so lahko tudi odvisni med sabo, zato jih je potrebno skrbno organizirati in poskrbeti da ne pride do cikličnih odvisnosti.
    \item Skripti za gradnjo in namestitev projekta: Skripti opisujejo kako projekt zgraditi in namestiti na testna in produkcijsko okolje. Ti skripti so po navadi del neprekinjene namestitve in morajo zato biti dobro organizirani in enostavni za razumevanje.
    \item Dokumentacija: Dobra dokumentacija je pomemben del vsakega projekta. Ta lahko vključuje uporabniška navodila, dokumentacijo aplikacijskega programskega vmesnika (API), in druge dokumente, ki pomagajo razvijalcem pri razvoju in uporabnikom pri razumevanju delovanja projekta.
\end{itemize}

Na projektno strukturo močno vpliva tudi sistem za verzioniranje kode, ki ga uporabljamo \cite{Wiki_Architectural_pattern}, saj lahko kodo shranjujemo v enem skupnem repozitoriju ali pa jo razdelimo v več repozitorijev. Izbira repozitorijske strukture je ključna za strukturo projekta, saj ima vsaka repozitorijska struktura različne prednosti in slabosti \cite{Kokrehel2022}. Najbolj pogosti repozitorijski strukturi sta: monorepo in polyrepo.

\subsection{Monorepo}
\label{subsec:monorepo}
Monorepo ali monolitni repozitorij je poimenovanje organizacije in verzioniranja izvorne kode z enim repozitorijem. Ta lahko vsebuje več projektov, paketov ali modulov in razvijalcem omogoča širok dostop do izvorne kode, skupnih orodji in skupni množici odvisnosti na enem mestu \cite{Jaspan2018}.

Prednosti monolitnega repozitorija so:
\begin{itemize}
    \item Boljša preglednost izvorne kode: Razvijalci lažje najdejo relevantno dokumentacijo, primere implementacij in uporabe, kar pozitivno vpliva na hitrost in kvaliteto kode \cite{Jaspan2018, Shakikhanli2022}. 
    \item Enostavne odvisnosti: Vsi paketi in moduli v repozitoriju imajo lahko enako verzijo in ni potrebno skrbeti katere verzije so med seboj kompatibilne.
    \item Lažje spreminjanje odvisnih delov kode: Pri popravljanju in posodabljanju kode lahko popravimo tudi vse dele kode, ki so odvisni od spremenjene kode in vse skupaj oddamo v repozitorij kot eno spremembo.
\end{itemize}

Slabosti monolitnega repozitorija so:
\begin{itemize}
    \item Velikost repozitorija: Monolitni repozitorij lahko skozi razvoj zasede veliko prostora na podatkovnem sistemu. To lahko oteži prenos izvorne kode iz repozitorija in ostale operacije sistema za verzioniranje \cite{Harry2017}.
    \item Kompleksen cevovod za neprekinjeno dostavo: Kompleksnost konfiguracije cevovoda za neprekinjeno dostavo se poveča, saj je potrebno vse korake izvesti na vseh modulih in projektih. Da ohranimo učinkovitost, pa nočemo vedno izvajati vseh korakov za vse module, saj to podaljša izvajanje cevovoda.
    \item Veliki zahtevki za združitev vej: Ker repozitorij vsebuje vse odvisne dele kode, je ob posodobitvah lahko posodobljenih veliko vrstic izvorne kode. To lahko povzroči slabo preglednost sprememb v pregledu sprememb zahtevka za združitev \cite{Harry2017}.
    \item Orodja za upravljanje: Za učinkovito upravljanje monolitnega repozitorija so velikokrat potrebna dodatna programska orodja kot so orodja za gradnjo aplikacije in upravljalci paketov. To lahko poveča kompleksnost razvijanja aplikacije.
\end{itemize}

\subsection{Polirepo}
\label{subsec:polirepo}
Polirepo je poimenovanje organizacije in verzioniranja izvorne kode z več repozitoriji. Ti vsebujejo vse pakete in komponente izvorne kode, ki skupaj tvorijo celotno aplikacijo \cite{Shakikhanli2022}.

Prednosti uporabe več repozitorijev:
\begin{itemize}
    \item Poenostavljeno upravljanje: S posameznimi moduli ali paketi v ločenih repozitorijih je lažje slediti spremembam in ugotoviti, katere spremembe spadajo k posameznim projektom.
    \item Ločitev odgovornosti: Shranjevanje posameznih projektov v ločenih repozitorijih lahko pomaga izolirati kodo in zmanjša tveganje za nastanek sporov med različnimi deli projekta.
    \item Fleksibilnost: S posameznimi projekti v ločenih repozitorijih je lažje, da se različne ekipe ali razvijalci lahko posvečajo različnim delom projekta brez motenj od drugih ekip.
    \item Velikost repozitorijev: Ker je celoten projekt razdeljen na več repozitorijev, so ti repozitoriji manjši in zato lažji za prenos in delo na lokalnih računalnikih, še posebej, če razvijalec dela samo na enem delu projekta.
    \item Potencialno hitrejše gradnje: S posameznimi moduli ali paketi v ločenih repozitorijih jih je morda mogoče graditi neodvisno, kar lahko skrajša skupni čas gradnje.
\end{itemize}

Slabosti uporabe več repozitorijev:
\begin{itemize}
    \item Verzioniranje in določevanje odvisnosti: Potrebno je določiti kako se bodo moduli ali paketi v posameznih repozitorijih verzionirali in kako bodo določene odvisnosti med njimi.
    \item Oteženo spreminjanje odvisnih delov kode: Po spremembi nekega dela kode, se velikokrat zgodi, da je potrebno popraviti še kodo v odvisnih modulih ali paketih. To pomeni, da je treba v vsakem izmed odvisnih repozitorijev dodati spremembo in po možnosti popraviti verzijo.
\end{itemize}

Oba pristopa imata prednosti in slabosti in tisto, kar en razvijalec šteje za prednost, lahko drugemu predstavlja slabost. Na primer, nekatere osebe lahko vidijo enostavnost upravljanja enega samega repozitorija kot prednost uporabe monolitnega repozitorija, medtem ko druge lahko zaradi potencialne povečane zapletenosti vidijo to kot slabost. Podobno lahko nekatere osebe vidijo sposobnost enostavnega deljenja kode in virov med projekti kot prednost uporabe več repozitorijev, medtem ko druge lahko vidijo potrebo po upravljanju večih repozitorijev kot slabost. Zato je izbira med uporabo monolitnega repozitorija ali več repozitorijev največkrat stvar osebnega okusa in je odvisna od specifičnih potreb ter narave projekta kot tudi delovnih navad podjetja.


%----------------------------------------------------------------
% Poglavje 3 - Tehnologije
%----------------------------------------------------------------


\chapter{Uporabljene tehnologije in aplikacije}
\label{ch:uporabljene-tehnologije-in-aplikacije}
V tem poglavju bomo predstavili uporabljene tehnologije in aplikacije, ki so bile uporabljene za postavitev cevovoda neprekinjene integracije in dostave (CI/CD). Opisali bomo repozitorije za nadzor različic in orodja za gradnjo, ki so bila uporabljena, pa tudi aplikacijo Gitlab in konfiguracijske strežnike. Prav tako bomo podrobneje obravnavali kako smo zagotovili varnost aplikacije in opisali pomožna orodja in njihov namen.
 
\section{Repozitoriji za nadzor različic}
\label{sec:repozitoriji-za-nadzor-različic}


\subsection{Nexus}
\label{subsec:nexus}
Apache Nexus\footnote{https://www.sonatype.com/products/nexus-repository} je eden od najbolj priljubljenih odprtokodnih upravljalcev repozitorijev, ki se uporablja za gostovanje in upravljanje artefaktov, kot so datoteke tipa JAR in ostale binarne komponente, ki jih potrebujejo projekti za svojo gradnjo in delovanje. Pogosto se uporablja kot namestniski strežnik za zunanje repozitorije, kot je na primer centralni repozitorij Maven, in za gostovanje interno razvitih artefaktov. To lahko pomaga izboljšati hitrost gradnje aplikacije, ker se zmanjša število zunanjih odvisnosti, ki jih je potrebno prenesti iz spleta, hkrati pa omogoča dodatno varnost, tako da omogoča nadzor nad tem, kateri artefakti so na voljo za gradnjo \cite{Varanasi2019}.

Poleg gostovanja artefaktov Nexus omogoča tudi upravljanje in omejevanje uporabe artefaktov z pravili, ki jih lahko nastavimo. Na primer, da morajo biti vsi artefakti digitalno podpisani ali pa da morajo biti odobreni s strani administratorja, preden jih lahko uporabimo ob gradnji aplikacije.

Nexus ponuja tudi uporabniški vmesnik, s katerim je enostavno poiskati in prenesti artefakte kot tudi naložiti nove artefakte v repozitorij ročno. Prav tako izpostavlja aplikacijski programski vmesnik z arhitekturnim slogom REST, ki se lahko uporablja za avtomatizacijo opravil upravljanja repozitorijev.

\subsection{Git}
\label{subsec:git}

\subsection{Register slik Docker}
\label{subsec:register-slik-docker}

\section{Orodja za gradnjo}
\label{sec: orodja-za-gradnjo}

\subsection{Maven}
\label{subsec:maven}

\subsection{Gradle}
\label{subsec:gradle}

\subsection{Npm}
\label{subsec:npm}

\subsection{Pnpm}
\label{subsec:pnpm}

\subsection{Docker}
\label{subsec:docker}

\subsection{Kaniko}
\label{subsec:kaniko}

\section{Gitlab}
\label{sec:gitlab}

\subsection{Neprekinjena integracija in dostava v Gitlab-u}
\label{subsec:gitlab-neprekinjena-integracija-in-dostava}

\subsection{Vzorci gitlab}
\label{subsec:vzorci-gitlab}

\section{Kubernetes}
\label{sec:kubernetes}

\section{Konfiguracijski strežniki}
\label{sec:konfiguracijski-strežniki}

\subsection{Consul}
\label{subsec:consul}

\subsection{Konfiguracijski strežnik Spring Boot}
\label{subsec:spring-boot-configuration-server}

\section{Zagotavljanje varnosti}
\label{sec:zagotavljanje-varnosti}

\subsection{Kubernetes v izoliranem produkcijskem okolju}
\label{subsec:kubernetes-v-izoliranem-produkcijskem-okolju}

\section{Pomožne tehnologije}
\subsection{subsec:pomozne-tehnologije}

\subsection{Jsonnet}
\label{subsec:jsonnet}

\subsection{Copier}
\label{subsec:copier}

%----------------------------------------------------------------
% Poglavje 4 - Implementacija
%----------------------------------------------------------------


\chapter{Implementacija}
\label{ch:implementacija}

\section{Potek integracije in dostave}
\label{sec:potek-integracije-in-dostave}

\subsection{Proces v podjetju}
\label{subsec:proces-v-podjetju}

\subsection{Proces pri stranki}
\label{subsec:proces-pri-stranki}

\subsection{Proces neprekinjene dostave stranki}
\label{subsec:proces-neprekinjene-dostave-stranki}

\section{Razvoj komponente za neprekinjeno dostavo in integracijsko testiranje - Medius CD}
\label{sec:razvoj-komponente-za-neprekinjeno-dostavo-in integracijsko-testiranje-medius-cd}

\section{Postavitev projekta}
\label{subsec:postavitev-projekta}

\subsection{Java in Maven}
\label{subsec:java-in-maven}

\subsection{Java in Gradle}
\label{subsec:java-in-gradle}

\subsection{JavaScript in NPM}
\label{subsec:javascript-in-npm}



%----------------------------------------------------------------
% Poglavje 5 - Rezultati
%----------------------------------------------------------------


\chapter{Rezultati}
\label{ch:rezultati}

\section{Predstavitev projekta}
\label{sec:predstavitev-projekta}

\section{Primer uporabe na projektu}
\label{sec:primer-uporabe-na-projektu}

\subsection{Java projekt}
\label{subsec:java-projekt}

\subsection{JavaScript projekt}
\label{subsec:javascript-projekt}

\subsection{Python projekt}
\label{subsec:python-projekt}

\subsection{Monorepo projekt}
\label{subsec:monorepo-projekt}


%----------------------------------------------------------------
% Poglavje 6 - Zaključek
%----------------------------------------------------------------


\chapter{Zaključek}
\label{ch:zaključek}


% ---------------------------------------------------------------
% Appendix
% ---------------------------------------------------------------
\appendix


%----------------------------------------------------------------
% SLO: bibliografija
% ENG: bibliography
%----------------------------------------------------------------
\bibliographystyle{elsarticle-num}

%----------------------------------------------------------------
% SLO: odkomentiraj za uporabo zunanje datoteke .bib (ne pozabi je potem prevesti!)
% ENG: uncomment to use .bib file (don't forget to compile it!)
%----------------------------------------------------------------
\bibliography{bibliography}


\end{document}
