%======================================================================================================================
% SLO: definiraj strukturo dokumenta
% ENG: define file structure
%======================================================================================================================
\documentclass[a4paper, 12pt]{book}
\usepackage[T1]{fontenc}

%======================================================================================================================
% SLO: Odkomentiraj "\SLOtrue " za izbiro slovenskega jezika
% ENG: Uncomment "\SLOfalse" to chose English languagge
%======================================================================================================================
\newif\ifSLO
\newif\ifTRACKEXIST
\newif\ifTRACKCS
\newif\ifPROGRAMMM

% ---------------------------------------------------------------------------------------------------------------------
% IMPORTANT: Adjust the thesis language, your study program and course within this block
%  --------------------------------------------------------------------------------------------------------------------
% switch language
\SLOtrue % Enables Slovenian language
%\SLOfalse  % Enables English language

% switch programs: Computer science and Multimedia. Set to false if the program is in Multimedia
\PROGRAMMMfalse
%\PROGRAMMMtrue

% switch on if your program is divided into tracks CS and DS, otherwise leave it false
% CAUTION: if you were first enrolled into your program before school year 2019/2020, your program is not divided
% into tracks. In any case, be absolutely sure you select the correct variant. IF IN DOUBT, always contact the
% student office to advise you.
%
\TRACKEXISTfalse
%\TRACKEXISTtrue

% default course name is "Computer science" if your course name is "Data science", set the following switch to false
\TRACKCStrue % uncomment if the thesis is from course "Information science"
%\TRACKCSfalse % uncomment if the thesis is from course "Data Science"
% ---------------------------------------------------------------------------------------------------------------------
% End of language, program and course adjustment
% ---------------------------------------------------------------------------------------------------------------------


%======================================================================================================================
% SLO: vključi oblikovanje in pakete
% ENG: include design and packages
%======================================================================================================================
\input{style/thesis_style}

%----------------------------------------------------------------------------------------------------------------------
% |||||||||||||||||||||| USTREZNO POPRAVI |||||||||||||||||||||||
% |||||||||||||||||||||| EDIT ACCORDINGLY |||||||||||||||||||||||
%----------------------------------------------------------------------------------------------------------------------
\newcommand{\ttitle}{Neprekinjena integracija in dostava poslovno kritičnih aplikacij}
\newcommand{\ttitleEn}{Continuous integration and delivery for business critical applications}
\newcommand{\tsubject}{\ttitle}
\newcommand{\tsubjectEn}{\ttitleEn}
\newcommand{\tauthor}{Jakob Maležič}
\newcommand{\temail}{jm6421@student.uni-lj.si}
\newcommand{\myyear}{2022}
\newcommand{\tkeywords}{neprekinjena integracija, neprekinjena dostava, poslovno kritične aplikacije}
\newcommand{\tkeywordsEn}{continuous integration, continuous deployment, business critical applications}
\newcommand{\mysupervisor}{doc.~dr.\ Nejc Ilc}
\newcommand{\mycosupervisor}{dr.\ Tadej Justin}

% include formatted front pages
\input{style/thesis_front_pages}

%================================================================
% ENG: main pages of the thesis
%================================================================

%----------------------------------------------------------------
% Poglavje 1
%----------------------------------------------------------------


\chapter{Uvod}
\label{ch:uvod}

Pri razvoju večje programske opreme ali aplikacije je programje navadno razdeljeno na več zaključenih enot, ki jih
različne razvojne skupine ali razvojna podjetja neodvisno razvijajo.\ Za slednje je pred delitvijo nalog vnaprej
izbrano in specificirano tudi produkcijsko okolje.\ Na ta način si lahko razvojne skupine, za preverjanje realizacije
izvedenega dela, pripravijo razvojno okolje, ki ima čim bolj podobno infrastrukturo, kot je na voljo v produkcijskem
okolju.\ Samo tako so posamezne skupine lahko dovolj samozavestne, da bo njihovo programje pravilno delovalo tudi v
produkcijskem okolju.

\section{Motivacija}
\label{sec:motivacija}

Poseben primer produkcijskega okolja predstavlja zaprto produkcijsko okolje naročnika, ki se velikokrat pojavi pri
poslovno kritičnih aplikacijah. Poslovno kritične aplikacije so tiste aplikacije, ki so ključne za delovanje
poslovnega procesa podjetja~\cite{Hinchey2010, Syng2016}. V takšnem primeru je zaradi varnosti produkcijsko okolje
pogosto razvijalcem nedostopno, kot tudi celotno omrežje naročnika. S tem je dostava programske kode v omrežje
naročnika s strani razvijalcev nemogoča. Za takšen primer razvijanja programske opreme je potrebno celoten proces
neprekinjene dostave prilagoditi in upoštevati morebitne dodatne zaplete pri izdaji programske opreme v produkcijsko
okolje.

\section{Cilji}
\label{sec:cilji}

V magistrski nalogi bomo predstavili strategije in tehnologije, ki nam omogočajo postavitev takšnega sistema. Slednje
bomo opisali na realnih projektih in pripravili orodja za izvajanje integracijskega ter dostavnega procesa. Bolj
podrobno bomo opisali tudi proces neprekinjene integracije in dostave v okolje, kjer nimamo dostopa do nobenega izmed
programskih okolij in nimamo možnosti odriva programske kode v naročnikovo omrežje.


%----------------------------------------------------------------
% Poglavje 2
%----------------------------------------------------------------


\chapter{Teoretična osnova}
\label{ch:teoretična-osnova}

\section{Poslovno kritične aplikacije}
\label{sec:poslovno-kritične-aplikacije}

\section{Neprekinjena integracija}
\label{sec:neprekinjena-integracija}

\section{Neprekinjena dostava}
\label{sec:neprekinjena-dostava}

\subsection{Dostava izvorne kode}
\label{subsec:dostava-izvorne-kode}

\subsection{Neprekinjena namestitev}
\label{subsec:neprekinjena-namestitev}

\section{Organizacija projekta}
\label{sec:organizacija-projekta}

\subsection{Monorepo}
\label{subsec:monorepo}

\subsection{Polyrepo}
\label{subsec:polyrepo}

%----------------------------------------------------------------
% Poglavje 3
%----------------------------------------------------------------

\chapter{Uporabljene tehnologije in aplikacije}
\label{ch:uporabljene-tehnologije-in-aplikacije}
 
\section{Repozitoriji programske opreme}
\label{sec:repozitoriji-programske-opreme}

\subsection{Nexus}
\label{subsec:nexus}

\subsection{Git}
\label{subsec:git}

\subsection{Register slik Docker}
\label{subsec:register-slik-docker}

\section{Gitlab}
\label{sec:gitlab}

\subsection{Gitlab neprekinjena integracija in dostava}
\label{subsec:gitlab-neprekinjena-integracija-in-dostava}

\subsection{Vzorci gitlab}
\label{subsec:vzorci-gitlab}

\section{Kubernetes}
\label{sec:kubernetes}

\section{Konfiguracijski strežniki}
\label{sec:konfiguracijski-strežniki}

\subsection{Consul}
\label{subsec:consul}

\section{Zagotavljanje varnosti}
\label{sec:zagotavljanje-varnosti}

\subsection{Kubernetes v izoliranem produkcijskem okolju}
\label{subsec:kubernetes-v-izoliranem-produkcijskem-okolju}

%----------------------------------------------------------------
% Poglavje 4
%----------------------------------------------------------------


\chapter{Implementacija}
\label{ch:implementacija}

\section{Potek integracije in dostave}
\label{sec:potek-integracije-in-dostave}

\subsection{Proces v podjetju}
\label{subsec:proces-v-podjetju}

\subsection{Proces pri naročniku}
\label{subsec:proces-pri-naročniku}

\subsection{Proces neprekinjene dostave naročniku}
\label{subsec:proces-neprekinjene-dostave-naročniku}

\section{Razvoj komponente za neprekinjeno dostavo in integracijsko testiranje - Medius CD}
\label{sec:razvoj-komponente-za-neprekinjeno-dostavo-in integracijsko testiranje-medius-cd}

\section{Postavitev projekta}
\label{subsec:postavitev-projekta}

\subsection{Java in Maven}
\label{subsec:java-in-maven}

\subsection{Java in Gradle}
\label{subsec:java-in-gradle}

\subsection{JavaScript in NPM}
\label{subsec:javascript-in-npm}



%----------------------------------------------------------------
% Poglavje 5
%----------------------------------------------------------------


\chapter{Rezultati}
\label{ch:rezultati}

\section{Predstavitev projekta}
\label{sec:predstavitev-projekta}

\section{Primer uporabe na projektu}
\label{sec:primer-uporabe-na-projektu}

\subsection{Java projekt}
\label{subsec:java-projekt}

\subsection{JavaScript projekt}
\label{subsec:javascript-projekt}

\subsection{Python projekt}
\label{subsec:python-projekt}

\subsection{Monorepo projekt}
\label{subsec:monorepo-projekt}


%----------------------------------------------------------------
% Poglavje 5
%----------------------------------------------------------------


\chapter{Zaključek}
\label{ch:zaključek}


% ---------------------------------------------------------------
% Appendix
% ---------------------------------------------------------------
\appendix


%----------------------------------------------------------------
% SLO: bibliografija
% ENG: bibliography
%----------------------------------------------------------------
\bibliographystyle{elsarticle-num}

%----------------------------------------------------------------
% SLO: odkomentiraj za uporabo zunanje datoteke .bib (ne pozabi je potem prevesti!)
% ENG: uncomment to use .bib file (don't forget to compile it!)
%----------------------------------------------------------------
\bibliography{bibliography}


\end{document}
