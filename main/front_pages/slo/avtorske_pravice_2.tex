\thispagestyle{empty}
\vspace*{\fill}
{\noindent\footnotesize


\vspace*{5cm}
{\small \noindent
{\sc Avtorske pravice}.
Rezultati magistrskega dela z izjemo vtičnika Gitlab Template Lint so intelektualna lastnina avtorja in podjetja Medius d.o.o. Za objavljanje ali izkoriščanje rezultatov ma\-gi\-str\-ske\-ga dela je potrebno pisno soglasje avtorja, podjetja Medius d.o.o., Fakultete za ra\-ču\-nal\-niš\-tvo in informatiko, ter mentorja.
}


% \vspace*{1.5cm}
% {\small \noindent

% Vtičnik Gitlab Template Lint, ki je eden izmed rezultatov, je ponujen pod licenco \textit{Creative Commons Priznanje avtorstva-Deljenje pod enakimi pogoji 2.5 Slovenija} (ali novej\v so razli\v cico).
% To pomeni, da se tako besedilo, slike, grafi in druge sestavine dela kot tudi rezultati magistrskega dela lahko prosto distribuirajo,
% reproducirajo, uporabljajo, priobčujejo javnosti in predelujejo, pod pogojem, da se jasno in vidno navede avtorja in naslov tega
% dela in da se v primeru spremembe, preoblikovanja ali uporabe tega dela v svojem delu, lahko distribuira predelava le pod
% licenco, ki je enaka tej.
% Podrobnosti licence so dostopne na spletni strani \href{http://creativecommons.si}{creativecommons.si} ali na Inštitutu za
% intelektualno lastnino, Streliška 1, 1000 Ljubljana.

% \begin{center}% 0.66 / 0.89 = 0.741573033707865
% \CcImageCc{0.741573033707865}\hspace*{1ex}\CcImageBy{1}\hspace*{1ex}\CcImageSa{1}%
% \end{center}
% }

\vspace*{1.5cm}
{\small \noindent
Izvorna koda vtičnika Gitlab Template Lint je ponujena pod licenco GNU General Public License,
različica 3 (ali novejša). To pomeni, da se lahko prosto distribuira in/ali predeluje pod njenimi pogoji.
Podrobnosti licence so dostopne na spletni strani \url{http://www.gnu.org/licenses/}.
}


}
\begin{center}
{\footnotesize{\sc \copyright \myyear\ \tauthor}}
\ \\ \vfill
{\em
Besedilo je oblikovano z urejevalnikom besedil \LaTeX.}
\end{center}