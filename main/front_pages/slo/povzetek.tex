%---------------------------------------------------------------
% SLO: slovenski povzetek
% ENG: slovenian abstract
%---------------------------------------------------------------
\selectlanguage{slovene} % Preklopi na slovenski jezik
\addcontentsline{toc}{chapter}{Povzetek}
\chapter*{Povzetek}

\noindent\textbf{Naslov:} \ttitle
\bigskip

V magistrski nalogi predstavimo proces neprekinjene integracije in dostave (ang. \textit{continuous integration and delivery} – CI/CD), ki predstavlja pomemben del razvoja poslovno kritičnih aplikacij. V svetu obstaja veliko aplikacij in programov, ki so namenjeni prav tej nalogi. Vendar vsak izmed njih ni primeren za uporabo pri razvoju poslovno kritičnih aplikaciji. V tej nalogi smo najprej izpostavili tehnologije, ki se splošno uporabljajo v okviru procesa CI/CD. Le-te smo preverili in izbrali primerne za vpeljavo v projekte poslovno kritičnih aplikacij. Dodatno smo izpostavili njihove prednosti in pomanjkljivosti pri uporabi. Na podlagi ugotovitev smo razvili komponento, ki omogoča enostavno povezovanje vseh izbranih tehnologij in skuša odpraviti izpostavljene pomanjkljivosti.

Razvili smo komponento za vzpostavitev CI/CD z orodjem GitLab CI/CD, imenovano Medius CD. S komponento smo poenotili cevovode CI/CD, zmanjšali podvojenost konfiguracije in kode ter olajšali vzdrževanje. Razvita komponenta je tudi zelo prilagodljiva različnim zahtevam naročnikov in omogoča enostavno vzpostavitev cevovoda CI/CD tako v razvojnem okolju kot tudi v okolju naročnika.

V procesu razvoja smo odkrili potrebo po vtičniku za integrirano okolje IntelliJ, ki bi preverjal konfiguracijske datoteke orodja GitLab CI/CD. Zato smo razvili odprtokodni vtičnik Gitlab Template Lint, ki v integriranem okolju prikazuje napake in združeno vsebino konfiguracijskih datotek. Razvit vtičnik smo objavili na tržnico vtičnikov JetBrains in na ta način omogočili njegovo uporabo velikemu številu razvijalcev DevOps po vsem svetu.

V sklopu naloge smo koncepte in funkcionalnosti komponente ponazorili s pomočjo študij primerov iz resničnih projektov poslovno kritičnih aplikacij, ki uporabljajo različne programske jezike. S tem smo pokazali praktično uporabnost razvite rešitve ter njeno sposobnost prilagajanja različnim scenarijem in programskim jezikom v poslovnem okolju.


% Opišemo poslovno kritične aplikacije in proces neprekinjene integracije in dostave. Predstavimo izbrane tehnologije in aplikacije, ki jih v nadaljevanju uporabimo za vzpostavitev cevovoda neprekinjene integracije in dostave. Poseben poudarek namenimu razvoju komponente Medius CD, ki olajšuje vzpostavitev cevovoda neprekinjene integracije in dostave z orodjem Gitlab CI/CD.

% Razvita komponenta je sestavljena iz dveh ključnih delov: prvi del vključuje pripravljene predloge za uporabo v okviru projektov GitLab, medtem ko drugi del obsega skripte v programskem jeziku Bash. Te skripte se uporabljajo za izvajanje bolj kompleksnih nalog v okviru opredeljenih predlog. Ena izmed glavnih prednosti razvite komponente je njena visoka prilagodljivost, saj omogoča prilagajanje različnim zahtevam projektov. Dodana konfiguracijska datoteka igra ključno vlogo pri določanju specifičnih zahtev za posamezen projekt.

% Razvijemo komponento Medius CD, ki olajša vzpostavitev cevovoda neprekinjene integracije in dostave na platformi Gitlab. Komponento razdelimo na dva dela: predloge Gitlab, ki jih lahko enostavno pouporabimo v projektih in skripte v programskem jeziku Bash, ki jih predloge uporabljajo za izvajanje bolj kompleksnih nalog. Razvita komponenta omogoča visoko prilagodljivost različnim projektnim zahtevam, ki jih je mogoče definirati v konfiguracijski datoteki. Slednje pokažemo na primerih projektov iz resničnega sveta. 

% V vzorcu je predstavljen postopek priprave magistrskega dela z uporabo okolja \LaTeX. Vaš povzetek mora sicer vsebovati približno 100 besed, ta tukaj je odločno prekratek. Dober povzetek vključuje: (1) kratek opis obravnavanega problema, (2) kratek opis vašega pristopa za reševanje tega problema in (3) (najbolj uspešen) rezultat ali prispevek magistrske naloge.

\subsection*{Ključne besede}
\textit{\tkeywords}
\clearemptydoublepage
