%---------------------------------------------------------------
% SLO: slovenski povzetek
% ENG: slovenian abstract
%---------------------------------------------------------------
\selectlanguage{slovene} % Preklopi na slovenski jezik
\addcontentsline{toc}{chapter}{Povzetek}
\chapter*{Povzetek}

\noindent\textbf{Naslov:} \ttitle
\bigskip

Opišemo poslovno kritične aplikacije in proces neprekinjene integracije in dostave. Predstavimo izbrane tehnologije in aplikacije, ki jih v nadaljevanju uporabimo za vzpostavitev cevovoda neprekinjene integracije in dostave. Poseben poudarek je namenjen razvoju komponente Medius CD, ki olajšuje vzpostavitev cevovoda neprekinjene integracije in dostave v okviru platforme Gitlab.

Razvita komponenta je sestavljena iz dveh ključnih delov: prvi del vključuje pripravljene predloge za uporabo v okviru projektov GitLab, medtem ko drugi del obsega skripte v programskem jeziku Bash. Te skripte se uporabljajo za izvajanje bolj kompleksnih nalog v okviru opredeljenih predlog. Ena izmed glavnih prednosti razvite komponente je njena visoka prilagodljivost, saj omogoča prilagajanje različnim zahtevam projektov. Dodana konfiguracijska datoteka igra ključno vlogo pri določanju specifičnih zahtev za posamezen projekt.

V sklopu naloge koncepte in funkcionalnosti komponente ponazorimo s pomočjo študij primerov iz realnih projektov. S tem pokažemo praktično uporabnost razvite rešitve ter njeno sposobnost prilagajanja različnim scenarijem v poslovnem okolju.

% Razvijemo komponento Medius CD, ki olajša vzpostavitev cevovoda neprekinjene integracije in dostave na platformi Gitlab. Komponento razdelimo na dva dela: predloge Gitlab, ki jih lahko enostavno pouporabimo v projektih in skripte v programskem jeziku Bash, ki jih predloge uporabljajo za izvajanje bolj kompleksnih nalog. Razvita komponenta omogoča visoko prilagodljivost različnim projektnim zahtevam, ki jih je mogoče definirati v konfiguracijski datoteki. Slednje pokažemo na primerih projektov iz resničnega sveta. 

% V vzorcu je predstavljen postopek priprave magistrskega dela z uporabo okolja \LaTeX. Vaš povzetek mora sicer vsebovati približno 100 besed, ta tukaj je odločno prekratek. Dober povzetek vključuje: (1) kratek opis obravnavanega problema, (2) kratek opis vašega pristopa za reševanje tega problema in (3) (najbolj uspešen) rezultat ali prispevek magistrske naloge.

\subsection*{Ključne besede}
\textit{\tkeywords}
\clearemptydoublepage
