%----------------------------------------------------------------
% SLO: LaTeX paketi
% ENG: LateX packages
%----------------------------------------------------------------
% SLO: omogoča uporabo slovenskih (latinskih) črk kodiranih v formatu UTF-8
% ENG: enables the use of slovene (latin) caracters encoded in the UFT-8 format
\usepackage[utf8]{inputenc}
%\inputencoding{utf8}
% SLO: naloži, med drugim, slovenske delilne vzorce
% ENG: loads, among others, slovene dividing patterns
\usepackage[slovene,english]{babel}
% SLO: poskrbi za postavitev strani
% ENG: takes care of the page layout
\usepackage{fancyhdr}
% SLO: za vlaganje slik različnih formatov
% ENG: for loading figures of different formats
\usepackage{graphicx}
\usepackage{caption}
\captionsetup[figure]{labelfont=bf} % SLO: napis "Slika #" v krepkem tisku
									% ENG: wirte "Figure #" caption in bold
\captionsetup[table]{labelfont=bf} % SLO: napis "Tabela #" v krepkem tisku
								   % ENG: wirte "Table #" caption in bold
% SLO: za pisanje psevdokode
% ENG: for writing pseudocode
\usepackage{algorithm}
\usepackage{algorithmic}
\floatname{algorithm}{\footnotesize Algorithm} % SLO: napis "Algoritem #" v krepkem tisku
											   % ENG: write "Algorithm #" caption in bold
% SLO: poveže reference slik/tabel in slike/tabele znotraj dokumenta
% ENG: links image/table references with the images/tables within the document
\usepackage[pdfa]{hyperref}
% SLO: pri kliku na referenco slike/tabele se postavi na vrh slike/tabele
% ENG: when clicking the image/table reference, position the focus on top of the image/table
\usepackage[all]{hypcap}
% SLO: omogoča, med drugim, definicjo in uporebo barve
% ENG: enables, among others, the definition and use of colors
\usepackage{xcolor}
%----------------------------------------------------------------
% SLO: dodatni paketi
% ENG: additional packages
%----------------------------------------------------------------
% SLO: omogoča večjo manipulacijo nad tabelami
% ENG: allows for greater manipulation of tables
\usepackage{booktabs}
% SLO: naloži dodatne simbole
% ENG: loads additional symbols
\usepackage{amssymb}
% SLO: omogoča, med drugim, sklicevanje na formule z eqref
% ENG: enables, among others, equation referencing with eqref
\usepackage{amsmath}
% SLO: omogoča komentiranje večjega dela teksta
% ENG: enables the commenting of larger text parts
\usepackage{verbatim}
% SLO: omogoča rotacijo PDF strani v ležeč položaj
% ENG: enables the rotation of a PDF page to landscape
\usepackage{pdflscape}
% SLO: omogoča barvanje vrstic in stolpcev tabel
% ENG: enables coloring of table rows and columns
\usepackage{colortbl}
\usepackage{url}
\usepackage{tabularx}



%================================================================
% SLO: nastavitve dokumenta
% ENG: document properties
%================================================================
% SLO: prilagoditev robov za tisk
% ENG: margin adjustments for printing
\addtolength{\marginparwidth}{-20pt}
\addtolength{\oddsidemargin}{40pt}
\addtolength{\evensidemargin}{-40pt}
% SLO: razmik med vrsticami
% ENG: line spacing
\renewcommand{\baselinestretch}{1.3}
% SLO: postavitev strani
% ENG: page layout
\renewcommand{\chaptermark}[1]{\markboth{\MakeUppercase{\thechapter.\ #1}}{}}
\renewcommand{\sectionmark}[1]{\markright{\MakeUppercase{\thesection.\ #1}}}
\renewcommand{\headrulewidth}{0.5pt} % Header rule
\renewcommand{\footrulewidth}{0pt} % Footer rule
%
\fancypagestyle{frontmatter}{%
	\fancyhf{} % Clear all headers and footers first
	\fancyhead[LE, RO]{\sl \thepage}
	%\fancyhead[LO]{\sl \rightmark}
	%\fancyhead[RE]{\sl \leftmark}
}
\fancypagestyle{mainmatter}{%
  	\fancyhf{} % Clear all headers and footers first
	\fancyhead[LE,RO]{\sl \thepage}
	\fancyhead[LO]{\sl \rightmark}
	\fancyhead[RE]{\sl \leftmark}
}
% SLO: font za ime avtorja
% ENG: font for author name
\newcommand{\authorfont}{\Large}
% SLO: font za naslov diplomskega dela
% ENG: font for thesis title
\newcommand{\titlefont}{\LARGE\bf}
% SLO: globina kazala
% ENG: content depth
\setcounter{tocdepth}{1}
% SLO: definiraj ukaz za prazno stran
% ENG: define the command for empty page
\newcommand{\clearemptydoublepage}{\newpage{\pagestyle{empty}\cleardoublepage}}

% course title
\newcommand{\trackname}{!undefined!}

\newcommand{\BibTeX}{{\sc Bib}\TeX}


