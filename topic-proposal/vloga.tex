\documentclass[a4paper, 12pt]{article}
\usepackage[slovene]{babel}
\usepackage{lmodern}
\usepackage[T1]{fontenc}
\usepackage[utf8]{inputenc}
\usepackage{url}
\usepackage{xcolor}

\definecolor{munsell}{rgb}{0.0, 0.5, 0.69}
\newcommand\cmnt[1]{\textcolor{munsell}{#1}}


\topmargin=0cm
\topskip=0cm
\textheight=25cm
\headheight=0cm
\headsep=0cm
\oddsidemargin=0cm
\evensidemargin=0cm
\textwidth=16cm
\parindent=0cm
\parskip=12pt

\renewcommand{\baselinestretch}{1.2}

\begin{document}

%%%%%%%%%%%%%%%%%%%%%%%%%% Izpolni kandidat! %%%%%%%%%%%%%%%%%%%%%%%%%%
\newcommand{\ImeKandidata}{Jakob} % Ime
\newcommand{\PriimekKandidata}{Maležič} % Priimek
\newcommand{\VpisnaStevilka}{63170191} % vpisna številka
\newcommand{\StudijskiProgram}{Računalništvo in informatika, MAG} % Študijski program/smer
\newcommand{\NaslovBivalisca}{ Cesta v log 18, 1330 Kočevje, Slovenija } % kandidatov naslov
\newcommand{\SLONaslov}{Neprekinjena integracija in dostava poslovno kritičnih aplikacij} % naslov dela v slovenščini
\newcommand{\ENGNaslov}{Continuous integration and delivery for business critical applications} % naslov dela v angleščini
%%%%%%%%%%%%%%%%%%%%%%%%%% Konec izpolnjevanja %%%%%%%%%%%%%%%%%%%%%%%%%%


\newcommand{\Kandidat}{\ImeKandidata~\PriimekKandidata}
\noindent
\Kandidat\\
\NaslovBivalisca \\
Študijski program: \StudijskiProgram \\
Vpisna številka: \VpisnaStevilka
\bigskip

{\bf Komisija za študijske zadeve}\\
Univerza v Ljubljani, Fakulteta za računalništvo in informatiko\\
Večna pot 113, 1000 Ljubljana\\

{\Large\bf
{\centering
    Vloga za prijavo teme magistrskega dela \\%[2mm]
\large Kandidat: \Kandidat \\[10mm]}}


\Kandidat, študent magistrskega programa na Fakulteti za računalništvo in informatiko, zaprošam Komisijo za študijske zadeve, da odobri predloženo temo magistrskega dela z naslovom:

%\hfill\begin{minipage}{\dimexpr\textwidth-2cm}
Slovenski: {\bf \SLONaslov}\\
Angleški: {\bf \ENGNaslov}
%\end{minipage}

Tema je bila že potrjena lani in je ponovno vložena: {\bf \textit{DA}}

Izjavljam, da so spodaj navedeni mentorji predlog teme pregledali in odobrili ter da se z oddajo predloga strinjajo.

Magistrsko delo nameravam pisati v slovenščini. % In case you would like to write the thesis in English, comment this line out, and use the following template to explain your request:
%Komisijo zaprošam, da odobri pisanje magistrskega dela v angleškem jeziku z obrazložitvijo ... .

Za mentorja/mentorico predlagam:

%%%%%%%%%%%%%%%%%%%%%%%%%% Izpolni kandidat! %%%%%%%%%%%%%%%%%%%%%%%%%%
\hfill\begin{minipage}{\dimexpr\textwidth-2cm}
Ime in priimek, naziv: Nejc Ilc, dr. doc. \\
Ustanova: Fakulteta za Računalništvo in Informatiko \\
Elektronski naslov: nejc.ilc@fri.uni-lj.si
\end{minipage}

Za somentorja/somentorico predlagam:

\hfill\begin{minipage}{\dimexpr\textwidth-2cm}
Ime in priimek, naziv: Tadej Justin, dr. \\
Ustanova: Medius \\
Elektronski naslov:  tadej.justin@medius.si \\
\end{minipage}



%%%%%%%%%%%%%%%%%%%%%%%%%% Konec izpolnjevanja %%%%%%%%%%%%%%%%%%%%%%%%%%

\bigskip


\hfill V Ljubljani, \today.
%V Ljubljani, dne …………………………
%
% Podpis mentorja: \hspace{180px} Podpis kandidata/kandidatke:




\clearpage
\section*{PREDLOG TEME MAGISTRSKEGA DELA}

\section{Področje magistrskega dela}

slovensko: razvijalci in operacije \\
angleško: developers and operations


\section{Ključne besede}

slovensko: neprekinjena integracija, neprekinjena dostava, poslovno kritične aplikacije   \\
angleško: continuous integration, continuous deployment, business critical applications


\section{Opis teme magistrskega dela}

% Navodilo (pobrišite v končnem izdelku):
%\cmnt{
%\textbf{Briši iz končnega izdelka:}
%Dolžina teme je zelo odvisna od zgoščenosti teksta in jasnosti podajanja argumentov. Zato je zelo težko predpisati natančno dolžino vsakega podpoglavja brez da bi ob tem posegali preveč v stil pisanja. Splošno vodilo naj bo, da naj bo iz teme: (i) jasno razviden problem in relevantnost problema, (ii) izpostavljene potencialne pomanjkljivosti sorodnih rešitev, (iii) novosti/prispevki naloge naj bodo jasni in v relaciji do sorodnih rešitev, (iv) jasne naj bodo pričakovane uporabljene metode za razvoj vaše rešitve, evalvacijo uspešnosti in primerjavo s sorodnimi deli.
%\\
%\\
%Kljub temu v vsakem podpoglavju podajamo okvirni obseg v številu besed. Vodilo pa naj bo vseeno vsebinska kvaliteta.}
% V nadaljevanju opredelite izhodišča magistrskega dela in utemeljite znanstveno ali strokovno relevantnost predlagane teme.

\textbf{Pretekle potrditve predložene teme:}\\
Predložena tema je bila oddana in potrjena v preteklih letih ter se od lanske ne razlikuje.
% Tu ne gre za to, da bi morali pogledati vse pretekle teme, če se slučajno ujemajo z vašo. Pač pa je ta del namenjen tistim, ki letos ponovno oddajate temo, ki vam jo je KŠZ potrdila že lani.
% v kolikor gre za temo, ki je bila že oddana v preteklem letu in je bila takrat potrjena, prosim to napišite. Prav tako napišite, če se v nečem tema razlikuje od lanske (ste kaj dodali, odvzeli).

\subsection{Uvod in opis problema}
%Navodilo:
%\cmnt{Pojasnite, kaj je problem, ki ga želite reševati, in podajte motivacijo za delo. Pri opisu motivacije se navežite na literaturo in nerešene probleme, ki jih bo naslavljala vaša magistrska naloga. Delo umestite v ožje področje dela. Okvirni obseg: ~300 besed (1/2 strani A4).}

Pri razvoju večje programske opreme ali aplikacije je programje navadno razdeljeno na več zaključenih enot, ki jih različne razvojne skupine ali razvojna podjetja neodvisno razvijajo. Za slednje je pred delitvijo nalog vnaprej izbrano in specificirano tudi produkcijsko okolje. Na ta način si lahko razvojne skupine, za preverjanje realizacije izvedenega dela, pripravijo razvojno okolje, ki ima čim bolj podobno infrastrukturo, kot je na voljo v produkcijskem okolju. Samo tako so posamezne skupine lahko dovolj samozavestne, da bo njihovo programje pravilno delovalo tudi v produkcijskem okolju.

Poseben primer produkcijskega okolja predstavlja zaprto produkcijsko okolje naročnika, ki se velikokrat pojavi pri poslovno kritičnih aplikacijah. Poslovno kritične aplikacije so tiste aplikacije, ki so ključne za delovanje poslovnega procesa podjetja \cite{Hinchey2010, Syng2016}. V takšnem primeru je zaradi varnosti produkcijsko okolje pogosto razvijalcem nedostopno, kot tudi celotno omrežje naročnika. S tem je dostava programske kode v omrežje naročnika s strani razvijalcev nemogoča. Za takšen primer razvijanja programske opreme je potrebno celoten proces neprekinjene dostave prilagoditi in upoštevati morebitne dodatne zaplete pri izdaji programske opreme v produkcijsko okolje.

V magistrski nalogi bomo predstavili strategije in tehnologije, ki nam omogočajo postavitev takšnega sistema. Slednje bomo opisali na realnih projektih in pripravili orodja za izvajanje integracijskega ter dostavnega procesa. Bolj podrobno bomo opisali tudi proces neprekinjene integracije in dostave v okolje, kjer nimamo dostopa do nobenega izmed programskih okolij in nimamo možnosti odriva programske kode v naročnikovo omrežje.


\subsection{Pregled sorodnih del}

%Navodilo:
%\cmnt{Opišite pregled sorodnih del na ožjem področju, na katerem nameravate opravljati magistrsko nalogo. Vsako delo naj bo na kratko opisano v nekaj stavkih, besedilo pa naj poudari njegove glavne prednosti, slabosti ali posebnosti. Sklicujte se na dela, navedena v razdelku \ref{literatura} Literatura in viri. Pregled naj bo fokusiran.  Okvirni obseg: 1/2 - 2/3 strani A4.}

Zanimanje in motivacijo za vpeljevanje strategij razvoja in operacij smo našli v \cite{smith_2020}, kjer so predstavljene tehnike vpeljevanja strategij v realnih delovnih okoljih. Avtor daje velik poudarek motivaciji in razlogom zakaj je vpeljava takšnih strategij potrebna in kako ugotoviti, katera orodja zares potrebujemo. Predstavljen je tudi psihološki vidik uporabe takšnih orodji na zaposlene.

V \cite{Ebert2016} so opisani posamezni koraki neprekinjene integracije in dostave ter orodja, ki se v posameznih korakih uporabljajo. Opisan je namen posameznih orodji in kako pripomorejo k bolj učinkovitemu delovanju integracijskega procesa. 

V \cite{Chen2016} je podrobno opisana neprekinjena dostava po posameznih korakih v cevovodu na primeru velikega podjetja. V prispevku so opisane prednosti in izzivi vzpostavitve takšnega sistema.

V \cite{KOCEVAR_2019} so predstavljene raziskave na področju oblačnega računalništva in uporaba principov razvoja ter operacij na oblačni programski rešitvi. Na osnovi tega, so opisane dve arhitekturni zasnovi. V rezultatih se izkaže, da je proces namestitve nove programske opreme veliko lažji in hitrejši, če ga avtomatiziramo in upoštevamo principe razvoja in operacij. Opiše tudi slabosti in prednosti tehnologije Kubernetes, ki jo bomo pri implementaciji tudi sami uporabljali.


\subsection{Predvideni prispevki magistrske naloge}

%Navodilo:
%\cmnt{Opišite predvidene prispevke magistrske naloge s področja računalništva in informatike, ki so lahko strokovni ali znanstveni. Poudarite in opišite predvideni napredek ali novost vašega dela v primerjavi z obstoječim stanjem na strokovnem (ali znanstvenem) področju.  Okvirni obseg: 70 besed.}

V delu bomo predstavili izzive, prednosti in rešitve razvoja poslovno kritičnih aplikacij z uporabo neprekinjene integracije in dostave programske opreme. Opisali bomo različne tehnologije, ki so del tega procesa in njihovo uporabo predstavili na realnem primeru. Za potrebe delovanja takšnega cevovoda bomo razvili splošna orodja, ki nam omogočajo uporabo pripravljenega cevovoda na projektih s poljubno arhitekturo.  Vsa orodja, ki jih bomo uporabljali, bodo dostopna zgolj znotraj lokalnega omrežja. S tem bomo zagotovili visoko stopnjo varnosti, ki je pomembna pri poslovno kritičnih aplikacijah. Največji prednosti razvitega sistema bosta možnost prilagoditve sistema za skoraj poljuben projekt in možnost izvajanja celotnega procesa integracije v naročnikovem okolju, neodvisno od podjetja, ki je aplikacijo razvilo.

\subsection{Metodologija}

%Navodilo:
%\cmnt{Na kratko opredelite metodologijo, ki jo nameravate uporabiti pri svojem delu. Metodologija vsebuje metode, ki jih nameravate uporabiti (npr. razvoj v izbranem programskem jeziku, izdelava strojne opreme itd.), postopek analize, postopek evalvacije vašega prispevka in primerjavo s sorodnimi deli.  Okvirni obseg: 1/4 - 1/3 A4 strani.}

Pri pripravi cevovoda za neprekinjeno integracijo in neprekinjeno dostavo bomo uporabljali različne repozitorije programske opreme: Nexus, Git in register Docker. Za upravljanje s cevovodom in organizacijo dela bomo uporabljali platformo GitLab, delovanje vsebovalnikov Docker bomo upravljali s sistemom Kubernetes, konfiguracijo posameznih aplikacij pa s konfiguracijskim strežnikom Consul. Vsa orodja bomo medseboj povezali z uporabo skriptov napisanih v jeziku Bash. Izdelan cevovod in pripadajoča orodja bomo predstavili na realnem projektu in opisali njegove prednosti in slabosti.

\subsection{Literatura in viri}
\label{literatura}

%Navodilo:
%\cmnt{Tu navedite vse vire, ki jih citirate v predlogu teme. Citiranje naj bo v skladu z znanstveno-strokovnimi standardi citiranja, na primer, \cite{Zivkovic2004}. Seznam naj vsebuje vsaj nekaj del, objavljenih v zadnjih petih letih. Prednostno naj bodo navedene objave s konferenc, revij, oziroma drugih priznanih virov.}

\renewcommand\refname{}
\vspace{-50px}
\bibliographystyle{elsarticle-num}
\bibliography{./bibliografija/bibliography}


%\bigskip
%
%Ljubljana, \today .

\end{document}
