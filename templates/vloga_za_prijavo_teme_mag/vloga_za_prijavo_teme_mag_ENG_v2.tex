\documentclass[a4paper, 12pt]{article}
\usepackage[slovene]{babel}
\usepackage{lmodern}
\usepackage[T1]{fontenc}
\usepackage[utf8]{inputenc}
\usepackage{url}
\usepackage{xcolor}

\definecolor{munsell}{rgb}{0.0, 0.5, 0.69}
\newcommand\cmnt[1]{\textcolor{munsell}{#1}}

\topmargin=0cm
\topskip=0cm
\textheight=25cm
\headheight=0cm
\headsep=0cm
\oddsidemargin=0cm
\evensidemargin=0cm
\textwidth=16cm
\parindent=0cm
\parskip=12pt

\renewcommand{\baselinestretch}{1.2}

\begin{document}

%%%%%%%%%%%%%%%%%%%%%%%%%% Filled out by the candidate! %%%%%%%%%%%%%%%%%%%%%%%%%%
\newcommand{\ImeKandidata}{Name} % Name
\newcommand{\PriimekKandidata}{Surname} % Surname
\newcommand{\VpisnaStevilka}{60606060 } % Enrollment number
\newcommand{\StudyProgramme}{Computer and information science, MAG} % Study programme
\newcommand{\NaslovBivalisca}{ Cesta 1, 1234 Mesto, Slovenija } % the candidate address
\newcommand{\SLONaslov}{Naslov dela v slovenščini} % Slovenian title
\newcommand{\ENGNaslov}{Title of the thesis in English} % English title
%%%%%%%%%%%%%%%%%%%%%%%%%% End of filling in  %%%%%%%%%%%%%%%%%%%%%%%%%%


\newcommand{\Kandidat}{\ImeKandidata~\PriimekKandidata}
\noindent
\Kandidat\\
\NaslovBivalisca \\
Study programme: \StudyProgramme \\
Enrollment number: \VpisnaStevilka
\bigskip

{\bf Committee for Student Affairs}\\
Univerza v Ljubljani, Fakulteta za računalništvo in informatiko\\
Večna pot 113, 1000 Ljubljana\\

{\Large\bf
{\centering
    The master’s thesis topic proposal \\%[2mm]
\large Kandidat: \Kandidat \\[10mm]}}


I, \Kandidat, a student of the 2nd cycle study programme at the Faculty of computer and information science, am submitting a thesis topic proposal to be considered by the Committee for Student Affairs with the following title:

%\hfill\begin{minipage}{\dimexpr\textwidth-2cm}
Slovenian: {\bf \SLONaslov}\\
English: {\bf \ENGNaslov}
%\end{minipage}

This topic was already approved last year: {\bf \textit{YES} , \textit{NO} (select an answer) }
 

I declare that the mentors listed below have approved the submission of the thesis topic proposal described in the remainder of this document.

I would like to write the thesis in English with the following reason ... <fill this in!>.

I propose the following mentor:

%%%%%%%%%%%%%%%%%%%%%%%%%% Filled in by the candidate! %%%%%%%%%%%%%%%%%%%%%%%%%%
\hfill\begin{minipage}{\dimexpr\textwidth-2cm}
Name and surname, title: \\
Institution: \\
E-mail:
\end{minipage}

I propose the following co-mentor:

\hfill\begin{minipage}{\dimexpr\textwidth-2cm}
Name and surname, title: \\
Institution: \\
E-mail:
\end{minipage}
%%%%%%%%%%%%%%%%%%%%%%%%%% End of filling in %%%%%%%%%%%%%%%%%%%%%%%%%%

\bigskip

\hfill Ljubljana, \today.


\clearpage
\section*{Proposal of the masters thesis topic}

\section{The narrow field of the thesis topic}

English: e.g. computer science, computer architecture


\section{Key-words}

English:


\section{Detailed thesis proposal}
\cmnt{
% Navodilo (pobrišite v končnem izdelku):
The thesis topic length depends alot on the style of writing and is difficult to specify the expected length by a rule. Therefore please follow the general rule: (i) clearly specify the probem and its relevance, (ii) expose potential drawbacks of existing solutions, (iii) clearly specify the contributions/novelties in relation to previous works, (iv) clearly specify the methods used in your solution and your plan of evaluation and quantitative comparison with the related works.
\\
\\
Nevertheless we specify in each section an approximate expected length -- this should serve as a guideline only. Quality should be the main guideline.}
% V nadaljevanju opredelite izhodišča magistrskega dela in utemeljite znanstveno ali strokovno relevantnost predlagane teme.

\textbf{Past approvements of the proposed thesis topic:}\\
The proposed thesis has not been submitted nor approved in previous years.

% In case you are submitting a topic that has already been approved in the previous years, please state this here and explain the potential differences to that topic.

\subsection{Introduction and problem formulation}

\cmnt{
Guideline (delete from the final version):
Explain the problem which you plan to tackle and motivate your work. In motivation, relate to the literature and unsolved problems that your thesis will address. Position your work within the narrow field of research.  Approximate length: ~300 words (1/2 A4 page).}


\subsection{Related work}

\cmnt{
Guideline (delete from the final version):
Prepare an overview of the related work that is directly related to your problem. Describe the main highlights of each related work in a few sentences. Point out pros and drawbacks. Reference the works listed in the References section. The related work should be focused. Approximate length: 1/2 - 2/3 A4 page.}

\subsection{Expected contributions}

\cmnt{
Guideline (delete from the final version):
Describe the expected contributions of your masters thesis in the field of computer science. The contributions can be either scientific or technical. Describe the novelties of your contribution in relation to the related work and state-of-the-art (scientific or technical-wise). Approximate length: 70 words.}

\subsection{Methodology}

\cmnt{
Guideline (delete from the final version):
Briefly describe the methodology that you intend to apply in your work. This section should describe the methods you intend to use (e.g., the framework used for development, theoretical frameworks), methods that will be applied for analysis and evaluation of your approach, and comparison with the most related works. Approximate length: 1/4 - 1/3 A4 page.}


\subsection{References}
\label{literatura}

\cmnt{
Guideline (delete from the final version):
List all the references that you cite in the proposal. Use the scientific standard of citing, e.g.,
\cite{Zivkovic2004}. The list should contain at least a few works published in the recent years. Preferably, the references should include publications from conferences, journals or other well-accepted sources in your field.
}

\renewcommand\refname{}
\vspace{-50px}
\bibliographystyle{elsarticle-num}
\bibliography{./bibliografija/bibliography}




%\bigskip
%
%Ljubljana, \today .

\end{document}
